\documentclass{article}
% \usepackage[T2A]{fontenc}
\usepackage[utf8]{inputenc}
\usepackage[english, russian]{babel}
\usepackage[top=1cm,bottom=1cm,left=2cm,right=2cm]{geometry}
\usepackage{graphicx}
\usepackage{amsmath}
\usepackage{amsfonts}
\usepackage{xcolor}
\title{ВШБ Бизнес-информатика: ТВиМС 2025. \\ Лист задач для самостоятельного решения \#5. \\ Распределения Бернулли, биномиальное, Пуассона.}
\date{}
\author{}

\newcommand{\problemspace}{8cm}

\begin{document}
\maketitle

\begin{enumerate}
    \item Преподаватель высшей математики случайным образом выставляет оценки за сданное дз. С равной вероятностью он может поставить любую оценку от 5 до 10.
В одной из групп дз сдали 20 человек.
\begin{enumerate}
    \item Найти вероятность того, что ровно четверть студентов получит 10.
    \item Найти вероятность того, что половина студентов получит удовлетворительно.
    \item Найти наивероятнейшее число студентов, которые получат 6, найти вероятность того, что ровно такое количество студентов получит 6.
    \item Найти вероятность того, что преподаватель выставит не менее чем, две отличные оценки.
\end{enumerate}

    \vspace{\problemspace}

    \item Круг от вертолетной площадки вписан в условный квадрат.
    Вы смотрите боевик, в котором с вертолета, пролетающего над этой площадкой, три человека спрыгивают в квадрат.
    Найдите вероятность того, что ровно один человек приземлился именно в круг посадочной площадки.

    \newpage

    \item Инвестор владеет акциями $7$ предприятий одной отрасли.
    Известно, что вероятность роста цены акций по каждому из предприятий равна $0.4$, вероятность падения равна $0.3$.
    (будем считать, что акции ведут себя независимо)
    \begin{enumerate}
        \item Найти вероятность того, что изменится цена акций шести предприятий.
        \item Найти вероятность того, цена акций вырастет более чем у двух предприятий. 
        \item Найти наивероятнейшее число предприятий, цена на акции у которых уменьшится. Найти соответствующую вероятность.
        \item Найти наивероятнейшее число предприятий, цена на акции у которых не уменьшится. Найти соответствующую вероятность.
    \end{enumerate}

    \vspace{\problemspace}
    
    \item В забеге участвуют $12$ лошадей (равной силы). Каждый из $30$ зрителей пытается составить свой прогноз для трех призовых мест,
    и отмечает случайным образом трех участников забега.
    Какова вероятность того, что хотя бы один из них окажется прав, если:
    \begin{enumerate}
        \item надо угадать победителей и их места.
        \item достаточно указать тройку лидеров в любом порядке.
    \end{enumerate}

    \textbf{Решение:}

    \begin{enumerate}
        \item В этом варианте условия вероятность правильного угадывания $P(W) = p = \frac{1}{P^3_{12}}$, где
        $$
        P^3_{12} = \frac{12!}{(12-3)!} = 12 \cdot 11 \cdot 10 = 1320
        $$
        Соответственно, вероятность неправильного угадывания $P(\overline{W}) = q = 1 - p = 1 - \frac{1}{P^3_{12}}$.
        Тогда вероятность того, что хотя бы один из зрителей окажется прав, является комплементарным событием (англ. \textit{complementary event}) к вероятности того, что ни один из зрителей не угадает.
        Получаем:
        \begin{equation*}
        P(A) = 1 - P(\overline{A}) = 1 - \left(1 - \frac{1}{P^3_{12}}\right)^{30} \approx 0.0225.
        \end{equation*}
        \item В данном случае вероятность правильного угадывания $P(W) = p = \frac{1}{C^3_{12}}$.
        Вероятность того, что хотя бы один из зрителей окажется прав получаем аналогичным способом:
        \begin{equation*}
        P(A) = 1 - P(\overline{A}) = 1 - \left(1 - \frac{1}{C^3_{12}}\right)^{30} \approx 0.128.
        \end{equation*}
    \end{enumerate}

    \newpage
    
    \item Подбрасывается кубик, а затем монетка подбрасывается столько раз, сколько очков выпало на кубике. 
    Известно, что орел выпал ровно $4$ раза. Какова вероятность того, что на кубике выпала «6»?

    \textbf{Решение:} Из информации, что орел выпал ровно $4$ раза, мы можем сделать вывод, что на кубике выпало как минимум $4$ очка.

    Пусть случайная величина $X$ - число очков на кубике, а случайная величина $T$ - число орлов при подбрасывании монетки.

    Тогда вероятность того, что орел выпал ровно $4$ раза, равна:
    \begin{gather*}
    P(T = 4) = P(T = 4 | X = 4) P(X = 4) + P(T = 4 | X = 5) P(X = 5) + P(T = 4 | X = 6) P(X = 6)\\
    P(T = 4) = \left( C_4^4 \left(\frac{1}{2}\right)^4 + C_5^4 \left(\frac{1}{2}\right)^5 + C_6^4 \left(\frac{1}{2}\right)^6 \right) \cdot \frac{1}{6}\\
    P(T = 4) \approx 0.0755
    \end{gather*}

    Искомая же вероятность $P(X = 6 | T = 4)$ равна:
    \begin{gather*}
    P(X = 6 | T = 4) = \frac{P(T = 4 | X = 6) P(X = 6)}{P(T = 4)}\\
    P(X = 6 | T = 4) = \frac{C_6^4 \left(\frac{1}{2}\right)^6 \frac{1}{6}}{0.0755} \approx 0.517
    \end{gather*}

    \item В гостинице $35$ номеров. Управляющий знает, что клиент, забронировавший номер, с вероятностью $0.1$ не приедет.
    Но на каждом пустом номере гостиница теряет деньги, так что управляющий бронирует номера для $38$ клиентов, с запасом - «все равно кто-нибудь не приедет».
    Найти вероятность того, что у него возникнут проблемы – количество приехавших окажется больше количества номеров.

    \textbf{Решение:}
    Схема Бернулли с $n=38$, $p=0.9$, $k>35$, $P(k>35)=P(k=36)+P(k=37)+P(k=38)=0.254$.
    Очень похоже на overbooking в авиаперевозках :)
    
    \item Студент Антон опаздывает на занятие с вероятностью $0.65$, студентка Валерия опаздывает с вероятностью $0.75$.
    Вероятность того, что они опоздают оба, равна $0.55$.
    \begin{enumerate}
        \item Антон пришел на занятие вовремя. Найти вероятность того, что Валерия тоже пришла вовремя.
        \item На занятие опоздал только один из них. Найти вероятность того, что это был Антон.
        \item На следующей неделе будет 20 пар. Найти наивероятнейшее число пар, на которое опоздает ровно один из этих студентов.
    Найти соответствующую вероятность.
    \end{enumerate}

    \textbf{Решение:}

(Схема Бернулли + действия над событиями, условные вероятности)
УКАЗАНИЕ: Пусть событие $A$ заключается в том, что Антон опоздал на занятие, событие $B$ заключается в том, Валерия опоздала. Зная вероятности $P(A)$, $P(B)$ и $P(A \cap B)$, мы можем найти все остальное.
\begin{enumerate}
    \item Вероятность $P(\bar{A} \cap \bar{B})$ можно найти, например, используя закон де Моргана:
    $$
    P(\bar{A} \cap \bar{B}) = P(\overline{A \cup B}) = 1 - P(A) - P(B) + P(A \cap B)
    $$
    В итоге получаем:
    $$
    P\left( \bar{B} \mid \bar{A} \right) = \frac{P(\bar{A} \cap \bar{B})}{P(\bar{A})} = \frac{1 - P(A) - P(B) + P(A \cap B)}{1 - P(A)} = \frac{1 - 0.65 - 0.75 + 0.55}{1 - 0.65} \approx 0.429
    $$
    \item Пусть $T$ - событие, что опоздал только один из них. Мы можем его записать как $T = (\bar{A} \cap B) \cup (A \cap \bar{B})$. Тогда нам нужна $P(A \mid T)$.
    \begin{gather*}
    P(T) = P(\bar{A} \cap B) + P(A \cap \bar{B}) = (P(B) - P(A \cap B)) + (P(A) - P(A \cap B)) =\\
    = P(B) + P(A) - 2P(A \cap B) = 0.75 + 0.65 - 2 \cdot 0.55 = 0.3
    \end{gather*}
    Тогда $P(A \mid T) = \frac{P(A \cap T)}{P(T)} = \frac{P(A) - P(A \cap B)}{P(T)} = \frac{0.65 - 0.55}{0.3} = \frac{1}{3}$.
    \item Успех - опоздание только одного из них. Эту вероятность мы уже нашли, это $P(T)$. Тогда получаем схеме Бернулли: $n=20, p=0.3 \Rightarrow k_0=6, P(k_0) \approx 0.192$.
\end{enumerate}

\newpage

    \item Студент Николай прогуливает пару с вероятностью $0.25$. Если он прогуляет, то студент Андрей прогуляет эту пару с вероятностью $0.6$.
    Если же Николай не прогуляет, то и Андрей не прогуляет с вероятностью $0.7$.
    \begin{enumerate}
        \item Найти вероятность того, что Андрей прогуляет очередную пару.
        \item Известно, что Андрей пришел на пару. Найти вероятность того, что Николай тоже пришел.
        \item Всего на прошлой неделе было 20 пар. Найти наивероятнейшее число пар, прогулянных Андреем, найти вероятность того, что он прогулял именно такое количество пар.
        \item На прошлой неделе было 7 пар, которые Андрей прогулял. Найти наивероятнейшее число пар из этих 7, прогулянных и Николаем, найти вероятность того, что он прогулял именно такое количество пар.
    \end{enumerate}

    \vspace{\problemspace}
    
    \item На дополнительный курс «Английский для менеджеров» ходят студенты первых трех курсов, причем состав слушателей такой: $30\%$ - 1 курс, $20\%$ - 2 курс, $50\%$ - третий.
    Ходят, правда, так себе – вероятность прогула первокурсника – $30\%$, второкурсника – $40\%$, третьекурсника – $50\%$.
    \begin{enumerate}
        \item Найти вероятность того, что случайно выбранный студент, пришедший на занятие – со второго курса.
        \item На занятие пришли 20 человек. Найти вероятность того, что из них ровно шесть второкурсников.
        \item На занятие пришли 20 человек. Найти наивероятнейшее число второкурсников, пришедших на занятие, найти вероятность того, что именно такое число второкурсников придет на занятие.
    \end{enumerate}
    \textit{(Внимательный читатель заметит, что тут нужна соответствующая оговорка – мы считаем курс очень большим, то есть если мы взяли студента, и он оказался первокурсником, то вероятность того, что следующий студент тоже первокурсник, не меняется)}

    \newpage
    
    \item В команде 10 хороших стрелков, попадающих в цель при одном выстреле с вероятностью $0.8$, и 3 плохих, попадающих с вероятностью $0.5$.
    Один стрелок производит $5$ выстрелов. Чему равна вероятность того, что это хороший стрелок, если он попал более двух раз?

    \vspace{\problemspace}

    \item Иван живет на 12 этаже и обычно спускается вниз на лифте. Ему известно, что в будний день вероятность того, что лифт остановится на любом промежуточном этаже, одинакова для всех этажей и равна $0.4$.
    В выходные эта вероятность равна $0.2$.
    В некоторый произвольный день Иван ехал со своего этажа на первый, лифт остановился по дороге $2$ раза.
    Найти вероятность того, что это было в будний день.

    \textit{(Считаем, что все остановки на каждом из этажей независимы, праздничных дней не существует)}

    \newpage
    
    \item Каждый понедельник в промежуток времени от 12-00 до 20-00 в порт заходят два корабля (в случайные моменты и независимо друг от друга).
    Первый из кораблей разгружается два часа, второй – три, причем в один момент времени может разгружаться только один корабль.
    \begin{enumerate}
        \item Найти вероятность того, что ни одному из кораблей не придется ждать разгрузки.
        \item Найти вероятность того, что за ближайшие два месяца (8 недель) будет ровно 4 дня, когда ни одному из кораблей не придется ждать разгрузки.
    \end{enumerate}

    \vspace{\problemspace}

    \item Для случайной величины $X$, распределенной по закону Пуассона с матожиданием $2$, найти вероятность того, что она примет значение:
    \begin{enumerate}
        \item равное 1,
        \item не больше 2,
        \item больше 2,
        \item вероятность того, что случайная величина $X$ отклонится от матожидания более чем на половину стандартного отклонения.
    \end{enumerate}

    \newpage

    \item В поселке $1000$ домов, каждый из которых застрахован от пожара в одной страховой компании на сумму $100,000$ рублей. Страховой взнос составляет $300$ рублей за год.
    Для данного поселка вероятность пожара в доме в течение года равна $0.002$.
    Какова вероятность того, что в течение года страховая компания по данному виду полиса выплатит больше, чем соберет?
    \textit{(Указание: при решении использовать распределение Пуассона)}

    \vspace{\problemspace}

    \item Всего на одном курсе учится $397$ студентов. Вероятность того, что студент придет на экзамен, равна $0.99$.
    \begin{enumerate}
        \item Найти вероятность того, что на экзамене будет не менее 395 студентов. Использовать биномиальное распределение.
        \item Найти вероятность того, что на экзамене будет не менее 395 студентов. Использовать приближение пуассоновским распределением.
    \end{enumerate}

    \newpage

    \item Сессия состоит из $5$ экзаменов, для каждого из студентов вероятность сдать на хорошо любой из экзаменов одинакова, равна $0.4$, и не зависит от других студентов и экзаменов,
    вероятность сдать на отлично равна $0.1$.
    \begin{enumerate}
        \item Найти вероятность того, что в группе из 20 студентов ровно 7 получит по два «хорошо» за эту сессию. 
        \item Найти вероятность того, что из 200 первокурсников ровно у двоих человек будет ровно по три отличные оценки. (Использовать приближение пуассоновским распределением)
    \end{enumerate}

    \vspace{\problemspace}

    \item Заказы на доставку еды из ресторана образуют простейший поток с интенсивностью $6$ заказов в час. Найти вероятность того, что:
    \begin{enumerate}
        \item за очередные 20 минут поступит хотя бы 2 заказа.
        \item за очередные 5 минут – хотя бы один.
    \end{enumerate}
    Указание: Задача на простейший поток, число событий – распределение Пуассона.
    Пусть единица времени час, тогда $\lambda=6$. Пусть случайная величина $X$ – число заказов за час, тогда $X$ распределена по закону Пуассона

    \newpage

    \item В пожарную часть поступает в среднем $8$ вызовов в сутки, вызовы образуют простейший поток.
    \begin{enumerate}
        \item Найти вероятность того, что за одну смену (12 часов) поступит 5 вызовов.
        \item Найти вероятность того, что за последние 20 смен было ровно три таких смены, за которые поступило по 5 вызовов.
    \end{enumerate}
    Указание: так как в задаче идет речь о потоке событий во времени, о простейшем потоке, то проще всего сначала ввести какую-то удобную единицу времени, чтобы все встречающиеся интервалы времени удобно выражались через эту единицу.
    Например, пусть единица времени – это сутки. Тогда интенсивность потока $\lambda=8$ событий/ед. времени.
\end{enumerate}

\end{document}
