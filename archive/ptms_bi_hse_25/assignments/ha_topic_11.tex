\documentclass{article}
% \usepackage[T2A]{fontenc}
\usepackage[utf8]{inputenc}
\usepackage[english, russian]{babel}
\usepackage[top=1cm,bottom=1cm,left=2cm,right=2cm]{geometry}
\usepackage{graphicx}
\usepackage{amsmath}
\usepackage{amsfonts}
\usepackage{xcolor}
\title{ВШБ Бизнес-информатика: ТВиМС 2025. \\ Лист задач для самостоятельного решения \#11. \\ Точечные оценки. \\ Метод моментов. Метод максимального правдоподобия. \\ Интервальные оценки.}
\date{}
\author{}

\begin{document}
\maketitle

\section*{Основные формулы}

\subsection*{Свойства точечных оценок}
\begin{itemize}
    \item Несмещенность: $E[\hat{\theta}] = \theta$
    \item Смещение: $\text{Bias}(\hat{\theta}) = E[\hat{\theta}] - \theta$
    \item Среднеквадратичная ошибка: $\text{MSE}(\hat{\theta}) = E[(\hat{\theta} - \theta)^2] = \text{Var}(\hat{\theta}) + [\text{Bias}(\hat{\theta})]^2$
\end{itemize}

\subsection*{Доверительные интервалы}
\begin{itemize}
    \item Среднее $\mu$, дисперсия известна:
    $$\mu \in \left(\bar{x} - z_{\alpha/2} \frac{\sigma}{\sqrt{n}}, \; \bar{x} + z_{\alpha/2} \frac{\sigma}{\sqrt{n}} \right)$$
    
    \item Среднее $\mu$, дисперсия неизвестна:
    $$\mu \in \left(\bar{x} - t_{(n-1, \alpha/2)} \frac{s}{\sqrt{n}}, \; \bar{x} + t_{(n-1, \alpha/2)} \frac{s}{\sqrt{n}} \right)$$
    
    \item Доля $p$:
    $$p \in \left(\tilde{p} - z_{\alpha/2} \sqrt{\frac{\tilde{p}(1-\tilde{p})}{n}}, \; \tilde{p} + z_{\alpha/2} \sqrt{\frac{\tilde{p}(1-\tilde{p})}{n}} \right)$$
    
    \item Разность долей $p_1 - p_2$:
    $$p_1 - p_2 \in \left(\tilde{p}_1 - \tilde{p}_2 - z_{\alpha/2} \sqrt{\frac{\tilde{p}_1(1-\tilde{p}_1)}{n} + \frac{\tilde{p}_2(1-\tilde{p}_2)}{m}}, \; \tilde{p}_1 - \tilde{p}_2 + z_{\alpha/2} \sqrt{\frac{\tilde{p}_1(1-\tilde{p}_1)}{n} + \frac{\tilde{p}_2(1-\tilde{p}_2)}{m}} \right)$$
    
    \item Разность средних $\mu_X - \mu_Y$, дисперсии $\sigma_X^2$, $\sigma_Y^2$ известны:
    $$\mu_X - \mu_Y \in \left(\bar{x} - \bar{y} - z_{\alpha/2} \sqrt{\frac{\sigma_X^2}{n} + \frac{\sigma_Y^2}{m}}, \; \bar{x} - \bar{y} + z_{\alpha/2} \sqrt{\frac{\sigma_X^2}{n} + \frac{\sigma_Y^2}{m}} \right)$$
    
    \item Разность средних $\mu_X - \mu_Y$, дисперсии неизвестны, но предполагаются равными:
    $$\mu_X - \mu_Y \in \left(\bar{x} - \bar{y} - t_{(n+m-2, \alpha/2)} s_p \sqrt{\frac{1}{n} + \frac{1}{m}}, \; \bar{x} - \bar{y} + t_{(n+m-2, \alpha/2)} s_p \sqrt{\frac{1}{n} + \frac{1}{m}} \right)$$
    где $s_p^2 = \frac{(n-1)s_X^2 + (m-1)s_Y^2}{n+m-2}$ — объединённая выборочная дисперсия.
\end{itemize}

\clearpage

\begin{enumerate}
    \item Независимые случайные величины $X_1, X_2$ и $X_3$ имеют одинаковое матожидание $\mu$ но разные стандартные отклонения $\sigma, 2 \sigma$ и $3 \sigma$ соответственно.
    В качестве оценки матожидания мы рассматриваем три варианта:
    $\hat{\theta}_1=\frac{1}{3} X_1+\frac{1}{3} X_2+\frac{1}{3} X_3$,
    $\hat{\theta}_2=\frac{1}{6} X_1+\frac{1}{5} X_2+\frac{1}{4} X_3$,
    $\hat{\theta}_3=\frac{1}{6} X_1+\frac{1}{3} X_2+\frac{1}{2} X_3$. Какая из этих оценок лучше?
    Указание - проверить несмещенность, у несмещенных сравнить дисперсии.

    % Ответ: из этих трех оценок $\hat{X}_1$ лучше.
    % $$
    % E \hat{X}_1=E\left(\frac{1}{3} X_1+\frac{1}{3} X_2+\frac{1}{3} X_3\right)=E X, \quad E \hat{X}_2=E\left(\frac{1}{6} X_1+\frac{1}{5} X_2+\frac{1}{4} X_3\right) \neq E X, \quad E \hat{X}_3=E\left(\frac{1}{6} X_1+\frac{1}{3} X_2+\frac{1}{2} X_3\right)=E X
    % $$

    \item Пусть $X_1$, $X_2$ — случайная выборка из распределения с математическим ожиданием $\mu$ и дисперсией $\sigma^2$.

    Рассмотрим следующую оценку дисперсии $\sigma^2$:
    $$
    \hat{\theta} = c (X_1 - X_2)^2.
    $$
    Найти константу $c$ такую, что $\hat{\theta}$ является несмещенной оценкой для $\sigma^2$.
    
    \item Случайные величины $X_1$ и $X_2$ распределены по одному закону и независимы.
    Среди всех несмещенных оценок матожидания вида $c_1 X_1+c_2 X_2$ найти оценку с наименьшей дисперсией.
    % Ответ: $c_1=0.5, c_2=0.5$
    % УКАЗАНИЕ: $E\left(c_1 X_1+c_2 X_2\right)=c_1 E X_1+c_2 E X_2=c_1 m+c_2 m=m \Rightarrow c_1+c_2=1$
    % $D\left(c_1 X_1+c_2 X_2\right)=c_1^2 D X+c_2^2 D X=\sigma^2\left(c_1^2+c_2^2\right) \rightarrow \min$ по всем $c_1$ и $c_2$, таким, что $c_1+c_2=1$ т.е решаем задачу на условный экстремум: $\left\{\sigma^2\left(c_1^2+c_2^2\right) \rightarrow \min c_1+c_2=1\right.$ - либо методом Лагранжа, либо подстановкой:
    % $$
    % \sigma^2\left(c_1^2+\left(1-c_1\right)^2\right) \rightarrow \min \Rightarrow c_1=\frac{1}{2}, c_2=\frac{1}{2} .
    % $$

    \item Количество покупок, совершаемых клиентами в интернет-магазине за день, подчиняется распределению Пуассона.
    У нас есть выборка данных по количеству покупок за несколько дней, результаты записаны в таблице.
    Необходимо определить параметр $\lambda$ по этой выборке, используя метод моментов.

    \begin{tabular}{|l|l|l|l|l|l|l|}
        \hline Количество покупок за день $x_i$ & $0$ & $1$ & $2$ & $3$ & $4$ & $5$ \\
        \hline Количество дней с количеством покупок $x_i$ & $10$ & $37$ & $38$ & $22$ & $12$ & $6$ \\
        \hline
    \end{tabular}

    \item Найти методом моментов по выборке $x_1, x_2, \, \ldots, \, x_n$ точечную оценку параметра $p$ биномиального распределения.

    \item При условии равномерного распределения случайной величины $X$ произведена выборка:
    
    \begin{center}
    \begin{tabular}{|l|l|l|l|l|l|l|l|l|l|l|l|l|l|l|l|l|l|l|l|}
        \hline 3 & 5 & 7 & 9 & 11 & 13 & 15 & 17 & 19 & 21 & 21 & 16 & 15 & 26 & 22 & 14 & 21 & 22 & 18 & 25 \\
        \hline
    \end{tabular}
    \end{center}
    
    Найти оценку параметров $a$ и $b$ по методу моментов.

    \item Тренер и ученик стреляют в цель до первого попадания каждый.
    Известно, что тренер попадает в цель с вероятностью в два раза большей, чем ученик.
    Методом максимального правдоподобия оценить вероятность попадания учеником в цель при единичном выстреле,
    если известно, что тренер попал со второго раза, а ученик – с пятого.
    При решении задачи использовать логарифмическую функцию правдоподобия.

    \item В магазине работают три кассы. Время обслуживания покупателей каждым из кассиров распределено по показательному закону.
    При этом первый из кассиров самый опытный – среднее время обслуживания покупателя у него в два раза меньше чем у оставшихся двух.
    Первый кассир обслужил очередного покупателя за минуту, второй – за две минуты, третий – за полторы.
    Методом наибольшего правдоподобия оценить параметр $\lambda$ – интенсивность для первого кассира.

    \item Андрей и Борис независимо друг от друга играют в покер в интернете.
    Выигрыш каждого из них за день – это случайная величина, распределенная по нормальному закону,
    причем известно, что у них одинаковое матожидание выигрыша $m$ (тысяч рублей), но разные стандартные отклонения – у Андрея $1$ а у Бориса $2$ тысячи рублей.
    За последний день они выиграли по 2 и 3 тысячи рублей соответственно.
    Методом наибольшего правдоподобия оценить значение параметра $m$.

    \item Для определения среднего возраста своих клиентов крупный производитель одежды провёл случайную выборку из $50$ клиентов и получил $\bar{x} = 36$. Известно, что стандартное отклонение генеральной совокупности $\sigma = 12$:
    \begin{enumerate}
        \item Постройте $98\%$ доверительный интервал для среднего возраста $\mu$ всех клиентов.
        \item Предположим, что требуется, чтобы $92\%$ доверительный интервал был строго равен $[\bar{X} - 2, \bar{X} + 2]$. Какой размер выборки для этого потребуется?
    \end{enumerate}

    \item Проведена случайная выборка из $200$ студентов. $30$ из них говорят, что им "очень нравится" статистика.
    \begin{enumerate}
        \item Вычислите долю студентов в этой выборке, которые говорят, что им "очень нравится" статистика, и затем постройте $95\%$ доверительный интервал для истинной доли.
        \item Теперь предположим, что вы решили спросить снова, но уже других студентов того же возраста. На этот раз в выборке $40$ студентов, и $16$ из них говорят, что им "очень нравится" статистика.
        Постройте $95\%$ доверительный интервал для истинной доли в этом случае.
        Подумайте, почему два одинаковых по уровню доверия интервала для "поимки" одного и того же параметра могут отличаться.
    \end{enumerate}

    % src: McKean, Craig; Exercise 4.2.1
    \item Рассмотрим случайную выборку размера $20$ из распределения $\mathcal{N}(\mu, \sigma^2)$.
    Наблюдаемые значения выборочного среднего и выборочной дисперсии равны $\bar{x} = 81.2$ и $s^2 = 26.5$.
    Найдите соответственно $90\%$, $95\%$ и $99\%$ доверительные интервалы для среднего генеральной совокупности $\mu$.
    Отметьте и прокомментируйте, как увеличивается ширина доверительных интервалов при увеличении уровня доверия.

    \item Есть опасения по поводу скорости автомобилей, проезжающих по определённому участку шоссе.
    Для случайной выборки из $7$ автомобилей радар зафиксировал следующие скорости (в милях в час):
    $$79 \quad 73 \quad 68 \quad 77 \quad 86 \quad 71 \quad 69$$
    \begin{enumerate}
        \item Найдите выборочное среднее и выборочную дисперсию.
        \item Указав все необходимые предположения, постройте $90\%$ доверительный интервал для средней скорости всех автомобилей, проезжающих по этому участку шоссе.
    \end{enumerate}

    \item Рассмотрим оригинальное клиническое исследование вакцины Солка от полиомиелита, проведённое в 1954 году.
    Случайным образом одна группа детей получила вакцину (группа лечения), а другая группа получила плацебо (контрольная группа).
    Пусть $p_c$ и $p_T$ обозначают истинные доли заболевших полиомиелитом в контрольной группе и группе лечения соответственно. Результаты исследования представлены в таблице:

    \begin{center}
    \begin{tabular}{ |c|c|c| } 
    \hline
    Группа & Количество детей & Количество случаев полиомиелита  \\
     \hline
    Лечение & $200,745$ & $57$  \\
    \hline
    Контроль & $201,229$ & $199$  \\ 
     \hline
    \end{tabular}
    \end{center}

    Постройте $95\%$ доверительный интервал для разности $(p_c - p_T)$. (Не округляйте слишком сильно, оставьте хотя бы $4-5$ знаков в дробной части).
    Что можно сказать об эффективности вакцины на основе построенного доверительного интервала?

    \item Пусть $\bar{x}$ и $\bar{y}$ — выборочные средние двух независимых случайных выборок $X_1, \ldots, X_8$ и $Y_1, \ldots, Y_{10}$ из распределений $\mathcal{N}(\mu_X, \sigma^2)$ и $\mathcal{N}(\mu_Y, \sigma^2)$ соответственно, где общая дисперсия неизвестна.
    Известны собранные данные: $\bar{x} = 5$, $\bar{y} = 3$, $\sum \limits_{i=1}^{8}x_i^2 = 215.75$, $\sum \limits_{i=1}^{10}y_i^2 = 107.64$.

    \begin{enumerate}
        \item Вычислите $95\%$ доверительные интервалы для $\mu_X$ и $\mu_Y$.
        \item Вычислите $90\%$ доверительный интервал для $\mu_X - \mu_Y$.
        \item Получите теоретическую формулу, а потом на имеющихся данных вычислите $95\%$ доверительный интервал для $\theta = \frac{1}{3}\mu_X + \frac{2}{3}\mu_Y$.
    \end{enumerate}
\end{enumerate}

\end{document}
