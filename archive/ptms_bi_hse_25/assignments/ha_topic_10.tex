\documentclass{article}
% \usepackage[T2A]{fontenc}
\usepackage[utf8]{inputenc}
\usepackage[english, russian]{babel}
\usepackage[top=1cm,bottom=1cm,left=2cm,right=2cm]{geometry}
\usepackage{graphicx}
\usepackage{amsmath}
\usepackage{amsfonts}
\usepackage{xcolor}
\title{ВШБ Бизнес-информатика: ТВиМС 2025. \\ Лист задач для самостоятельного решения \#10. \\ Мат. статистика. Выборки. Выборочные распределения и характеристики.}
\date{}
\author{}

\begin{document}
\maketitle

Медиана (середина ранжированного ряда):
$$
\text{Median} (x) = \begin{cases}
x_{m+1}, & \text{если } n=2m+1 \\
\frac{x_m + x_{m+1}}{2}, & \text{если } n=2m
\end{cases}
$$

Мода: $\text{Mode} (x)$ – значение, встречающееся в выборке чаще всего.

\vspace{0.5cm}

\textbf{Выборочное среднее $\bar{X}$:}

Определение: $\bar{X} = \frac{1}{n}\sum\limits_{i=1}^{n} X_i$

Характеристики: $E[\bar{X}] = \mu$, $\text{Var}(\bar{X}) = \frac{\sigma^2}{n}$

Распределение:
\begin{itemize}
    \item Если $X_i \sim \mathcal{N}(\mu, \sigma^2)$, то $\bar{X} \sim \mathcal{N}\left(\mu, \frac{\sigma^2}{n}\right)$
    \item По центральной предельной теореме (ЦПТ): при $n \geq 30$ выполняется приближенно $\bar{X} \sim \mathcal{N}\left(\mu, \frac{\sigma^2}{n}\right)$
\end{itemize}

\vspace{0.5cm}

\textbf{Выборочная дисперсия $S^2$:}

Определение: $S^2 = \frac{1}{n-1}\sum\limits_{i=1}^{n}(X_i - \bar{X})^2$

Характеристики: $E[S^2] = \sigma^2$, $\text{Var}(S^2) = \frac{2\sigma^4}{n-1}$

Распределение: Если $X_i \sim \mathcal{N}(\mu, \sigma^2)$, то $\frac{(n-1)S^2}{\sigma^2} \sim \chi^2_{n-1}$ (хи-квадрат с $n-1$ степенями свободы)

\vspace{0.5cm}

\textbf{Выборочная доля $\hat{p}$:}

Пусть $X_1, \ldots, X_n$ — выборка из распределения Бернулли с $P(X_i = 1) = p$, $P(X_i = 0) = 1-p$.

Определение: $\hat{p} = \frac{Y}{n}$, где $Y = \sum\limits_{i=1}^{n} X_i$ — количество успешных исходов (имеет биномиальное распределение)

Характеристики: $E[\hat{p}] = p$, $\text{Var}(\hat{p}) = \frac{p(1-p)}{n}$

Распределение:
\begin{itemize}
    \item По интегральной теореме Муавра-Лапласа (ИТМЛ): при $n > 30$ выполняется приближенно $\hat{p} \sim \mathcal{N}\left(p, \frac{p(1-p)}{n}\right)$
\end{itemize}

\vspace{0.5cm}

\begin{enumerate}

    \item В таблице приведены данные по 10 салонам сотовой связи: первая строка - расстояние от метро, вторая - дневные продажи смартфонов.
    Для продаж найдите $\text{Median } (Y)$, $\text{Mode } (Y)$, выборочное среднее $\bar{y}$, выборочное стандартное отклонение $s$.
    
    \begin{tabular}{|c|l|l|l|l|l|l|l|l|l|l|}
    \hline$x_i$ & 30 & 50 & 40 & 40 & 60 & 50 & 70 & 100 & 20 & 50 \\
    \hline$y_i$ & 5 & 7 & 8 & 9 & 4 & 5 & 3 & 5 & 6 & 8 \\
    \hline
    \end{tabular}

    

    \item Предположим, что случайная выборка размера $n = 64$ будет выбрана из генеральной совокупности со средним $40$ и стандартным отклонением $5$.
    \begin{enumerate}
    \item Каковы математическое ожидание и стандартное отклонение выборочного распределения $\bar{X}$? Опишите форму выборочного распределения.
    \item Какова приближённая вероятность того, что $\bar{X}$ будет находиться в пределах $0.5$ от среднего генеральной совокупности $\mu$?
    \item Какова приближённая вероятность того, что $\bar{X}$ будет отличаться от $\mu$ более чем на $0.7$?
    \end{enumerate}

    
    \item Производственный процесс предназначен для изготовления
    деталей диаметром $0.5$ дюйма. Каждый день выбирается случайная выборка из $36$ деталей и записываются их диаметры.
    Если полученное выборочное среднее меньше $0.49$ дюйма или
    больше $0.51$ дюйма, процесс останавливается для настройки.
    Стандартное отклонение диаметра составляет $0.02$ дюйма. Какова
    вероятность того, что производственная линия будет остановлена
    без необходимости? 

    \item Сообщается, что в крупном
    исследовании, проведённом в штате Нью-Йорк, примерно
    $30\%$ участников исследования жили в пределах 1 мили от опасного
    места захоронения отходов. Пусть $p$ обозначает долю всех жителей
    Нью-Йорка, которые живут в пределах 1 мили от такого места, и
    предположим, что $p = 0.3$.
    \begin{enumerate}
    \item Каковы математическое ожидание и стандартное отклонение выборочной доли $\hat{p}$ на основе случайной выборки размера $n = 400$?
    \item При $n = 400$ чему равно $P(0.25 \leq \hat{p} \leq 0.35)$?
    \item Будет ли вероятность, вычисленная в предыдущем пункте, больше или меньше, чем в случае, если $n = 500$?
    \end{enumerate}

    \item Производитель компьютерных принтеров закупает
    пластиковые картриджи с чернилами у поставщика. Когда поступает крупная партия,
    выбирается случайная выборка из $200$ картриджей, и каждый картридж проверяется.
    Если выборочная доля бракованных картриджей больше $0.02$, вся
    партия возвращается поставщику.
    \begin{enumerate}
    \item Какова приближённая вероятность того, что партия
    будет возвращена, если истинная вероятность брака картриджа равна $0.05$?
    \item Какова приближённая вероятность того, что партия
    не будет возвращена, если истинная вероятность брака картриджа равна $0.10$?
    \end{enumerate}

    \item Кабельная компания решает, стоит ли
    заменить канал "магазин на диване" новой местной телепрограммой. Будет проведён опрос
    $100$ абонентов. Кабельная компания решила
    оставить канал "магазин на диване", если выборочная доля
    окажется больше $0.25$. Какова приближённая вероятность того,
    что кабельная компания все-таки его оставит,если истинная доля тех,
    кто его смотрит, составляет только $0.20$?

    \item Предположим, что случайная выборка будет взята из
    нормального распределения с неизвестным средним $\mu$ и стандартным
    отклонением $\sigma = 2$.
    Какого размера должна быть случайная выборка, чтобы $P(|\bar{X} - \mu| \leq 0.1) \geq 0.95$ для любого возможного значения $\mu$?

    \item Предположим, что случайная выборка будет взята из
    распределения Бернулли с неизвестным параметром $p$. Предположим также,
    что считается, что значение $p$ находится в окрестности 0.2. Найдите приближённо размер
    случайной выборки, который должен быть взят, чтобы $P(|\bar{X} - p| \leq 0.1) \geq 0.95$ при $p = 0.2$.
\end{enumerate}

\end{document}
