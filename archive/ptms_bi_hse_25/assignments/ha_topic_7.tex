\documentclass{article}
% \usepackage[T2A]{fontenc}
\usepackage[utf8]{inputenc}
\usepackage[english, russian]{babel}
\usepackage[top=1cm,bottom=1cm,left=2cm,right=2cm]{geometry}
\usepackage{graphicx}
\usepackage{amsmath}
\usepackage{amsfonts}
\usepackage{xcolor}
\title{ВШБ Бизнес-информатика: ТВиМС 2025. \\ Лист задач для самостоятельного решения \#7. \\ Нормальное распределение.}
\date{}
\author{}

\begin{document}
\maketitle

\begin{enumerate}
\item Для случайной величины $Z \sim \mathcal{N}(0, 1)$ найдите вероятности следующих событий:
\begin{enumerate}
    \item $P(-1.5 < Z < 0.5)$,
    \item $P(Z < 1.25)$,
    \item $P(Z > 0.5)$,
    \item $P(Z < -0.25)$,
    \item $P(-2 < Z < -1)$,
    \item Найдите такую точку $z_{\alpha}$, что $P(Z > z_{\alpha}) = 0.025$.
    
    Для подобных вопросов, когда наоборот нам дана вероятность и надо найти точку, мы тоже используем таблицу стандартного нормального распределения.
    Если в таблице нет точного значения вероятности, то мы выбираем ближайшее значение и используем его.
    \item Найдите такую точку $z_{\alpha}$, что $P(Z > z_{\alpha}) = 0.898$.
\end{enumerate}

\item Дневная выручка торговой точки распределена по нормальному закону с математическим ожиданием $80,000$ рублей и стандартным отклонением $16,000$ рублей. Найдите вероятность того, что выручка окажется:
\begin{enumerate}
    \item в пределах от $56,000$ рублей до $88,000$ рублей.
    \item более $88,000$ рублей.
    \item в пределах от $48,000$ рублей до $64,000$ рублей.
    \item менее $76,000$ рублей.
    \item менее $100,000$ рублей.
    \item Какой должна быть выручка за день, чтобы можно было сказать, что этот день попал в $2.5\%$ дней с наибольшей выручкой?
    \item Какой должна быть выручка за день, чтобы можно было сказать, что этот день попал в $10.2\%$ дней с наименьшей выручкой?
\end{enumerate}

У этой и предыдущей задач ответы совпадают, но перемешаны.
Это прекрасная начальная иллюстрация того, что нет разницы, какое именно перед нами нормальное распределение:
любое нормальное распределение мы можем свести к стандартному нормальному распределению.

%  source: Bertsekas, Introduction to Probability, section 3.3, Problem 12
\item Пусть $X$ — нормальная случайная величина с нулевым математическим ожиданием и стандартным отклонением $\sigma$.
Используя таблицу стандартного нормального распределения, вычислите вероятности событий $\{ X \geq k \sigma \}$ 
и $\{ |X| \leq k \sigma \}$ для $k = 1, \;2, \; 3$.

% source: Ross, 5.18
\item Предположим, что $X$ — нормальная случайная величина с математическим ожиданием $5$.

Если $P\{X > 9\} = 0.2$, то чему приблизительно равна $Var(X)$?
    
% source: Ross, 5.19
\item Пусть $X$ — нормальная случайная величина с математическим ожиданием $12$ и дисперсией $4$.
Найдите такое значение $c$, что $P\{ X > c \} = 0.1$.

\item 
\begin{enumerate}
    \item Для случайной величины $Y \sim \mathcal{N}(\mu, 5^2)$ известно, что $P(Y \geq 1) = 0.0668$. Найдите $P(Y \leq 0)$.

    \item Найдите такой симметричный относительно математического ожидания промежуток,
    в который случайная величина $U \sim \mathcal{N}(1000, 200^2)$ попадет с вероятностью $0.95$.

    Иными словами, найдите такое $a$, что $P\left(1000 - a \leq U \leq 1000 + a\right) = 0.95$.
\end{enumerate}

\item Предположим, что число посетителей кафе за день – случайная величина, распределенная по нормальному закону.
Управляющий кафе знает, что с вероятность $0.2$ число посетителей окажется больше $900$,
а с вероятностью $0.3$ меньше $800$. 
Найдите среднее число посетителей за день и стандартное отклонение числа посетителей за день.
Запишите функцию плотности, изобразите график, покажите с помощью графика указанные в задаче вероятности.

\item Предположим, что срок службы телефона - нормальная случайная величина со средним значением $24$ месяца и стандартным отклонением $3$ месяца.
Какой максимальный срок гарантии можно выставить, если мы готовы ремонтировать по гарантии не более $10\%$ телефонов?
Запишите функцию плотности, изобразите график, покажите с помощью графика указанную в задаче вероятность.

% src: ICEF Examination, 2016 FR Marking, 
\item Вес дынь, продаваемых в супермаркете, имеет нормальное распределение с математическим ожиданием $3$ кг и стандартным отклонением $500$ г.
\begin{enumerate}
    \item Опишите функцию вероятности дискретной случайной величины $X$ — числа дынь тяжелее $4$ кг среди $4$ дынь.
    \item Чему равна вероятность того, что среди $4$ дынь хотя бы одна тяжелее $4$ кг? 
\end{enumerate}

% src: UoL, 2006, zone A
\item Машина, заполняющая банки, подает в каждую банку объем $X$ фруктов и объем $Y$ сока.
Известно, что $X$ имеет нормальное распределение с математическим ожиданием $260$ и стандартным отклонением $17$, 
тогда как $Y$ имеет нормальное распределение с математическим ожиданием $150$ и стандартным отклонением $10$.
Эти случайные величины можно считать независимыми.

\begin{enumerate}
    \item Вычислите вероятность того, что объем фруктов, загруженных машиной в банку, больше $290$ единиц.
    \item Найдите вероятность того, что объем поданных фруктов более чем в два раза превышает объем сока.
    \item Если вместимость банки составляет $400$ единиц, чему равна вероятность того,
    что после заполнения банка окажется недозаполнена?
\end{enumerate}

\item Предположим, что $V_i \sim \mathcal{N}(500, 45^2)$, $W_j \sim \mathcal{N}(300, 70^2)$ и все случайные величины независимы.

\textit{Это хороший показательный пример. Перед решением, например, можно подумать о том, какое распределение имеют случайные величины: $2 V_1$, $V_1 + V_2$, $V_1 - V_2$?
Если распределения отличаются, в том числе по параметрам, то как и почему?}

\begin{enumerate}
    \item Найдите $P(9 \cdot V_1 + 11 \cdot W_1 < 8000)$.
    \item Найдите $P(V_1 + \ldots + V_9 + W_1 + \ldots + W_{11} < 8000)$.
\end{enumerate}

\item На курсе из $180$ человек обучается $10$ сильных студентов, их оценка по предмету - $X \sim \mathcal{N}(7.5, \, (0.8)^2)$,
оценка для обычных студентов - $Y \sim \mathcal{N}(6, \, 1)$.
\begin{enumerate}
    \item Найдите вероятность того, что работу, получившую $8$ или больше (без округлений), написал сильный студент.
    \item Найдите вероятность того, что случайный сильный студент получит более высокую оценку, чем случайный обычный студент.
\end{enumerate}

\item В группе 32 студента. Известно, что оценка студента за экзамен распределена по нормальному закону со средним $7.1$ и стандартным отклонением $1.2$.
Учебная часть требует у старосты следующие данные: суммарный балл группы и средний балл по группе.
Учитывая, что конечно же студенты не списывают, то есть сдают экзамен независимо друг от друга, найдите вероятность того, что:
\begin{enumerate}
    \item суммарный балл группы (сумма всех оценок) превзойдет $220$ баллов,
    \item средний балл по группе будет выше, чем $6.9$.
\end{enumerate}

\end{enumerate}

\newpage
\includegraphics[scale=0.9]{../files/normal_distr_table.pdf} 

\end{document}
