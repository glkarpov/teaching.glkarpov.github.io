\documentclass{article}
% \usepackage[T2A]{fontenc}
\usepackage[utf8]{inputenc}
\usepackage[english, russian]{babel}
\usepackage[top=1cm,bottom=1cm,left=2cm,right=2cm]{geometry}
\usepackage{graphicx}
\usepackage{amsmath}
\usepackage{amsfonts}
\usepackage{xcolor}
\title{ВШБ Бизнес-информатика: ТВиМС 2025. \\ Лист задач для самостоятельного решения \#1. \\ Классическая и комбинаторная вероятность.}
\date{}
\author{}

\begin{document}
\maketitle
Все задачи из данного списка надо научиться решать с помощью формулы классической вероятности:
$$
    P(A) = \frac{|A|}{|\Omega|}
$$
где $|A|$ -- мощность интересуемого события a.k.a. число благоприятных исходов, $|\Omega|$ -- мощность пространства элементарных исходов a.k.a. общее число элементарных исходов, все элементарные исходы равновозможны.


\begin{enumerate}

    
    % source: http://www.statslab.cam.ac.uk/~rrw1/prob/exprob1.pdf
    \item Несколько кроликов живут под деревом. Четыре кролика случайным образом выбираются из их числа. Вероятность того, что оба белых кролика попадут в выбранную группу, в два раза больше вероятности того, что ни один из белых кроликов не будет выбран. Сколько кроликов живут под деревом?

    % Новые задачи из assign_1.txt
    \item Для нового смс-сервиса случайно выбирается четырехзначный телефонный номер. Найти вероятность событий:
    \begin{enumerate}
        \item В номере не будет цифры 2.
        \item В номере будет хотя бы одна цифра 4.
        \item Известно, что номер состоит только из цифр, меньших 5. Найти вероятность того, что в нем есть хотя бы одна 1.
        \item Четных цифр в номере будет больше чем нечетных.
        \item Первые две цифры будут четными (остальные две -- любые)
        \item Все цифры будут разными.
        \item На первом месте не будет «2», если известно, что все цифры разные.
    \end{enumerate}

    \item
    \begin{enumerate}
        \item Два студента случайным образом садятся в поезд, состоящий из шести вагонов. Найти вероятность того, что они окажутся в одном вагоне. Найти вероятность того, что они окажутся в разных вагонах.
        \item Три студента садятся в поезд, состоящий из шести вагонов. Найти вероятность того, что они все поедут в разных вагонах. Найти вероятность того, что они поедут в одном вагоне. Найти вероятность того, что хотя бы двое окажутся в одном вагоне.
    \end{enumerate}

    \item 
    \begin{enumerate}
        \item Пять студентов A, B, C, D и E случайным образом встают в очередь. Найти вероятность того, что A и B окажутся рядом.
        \item Пять студентов A, B, C, D и E случайным образом рассаживаются за круглым столом. Найти вероятность того, что A и B будут сидеть рядом.
    \end{enumerate}

    \item Подброшены два кубика. Найти вероятность того, что произведение выпавших очков делится на три.
    \textit{Указание: либо просто перебор, либо через противоположное событие.}

    \item Сколько раз надо бросить кубик, чтобы с вероятностью не менее $0.9$ хотя бы один раз выпала четверка? \textit{(т.к. есть словосочетание "хотя бы один раз" – попробуйте найти через противоположное событие.)}

    \item Экзамен состоит из 9 задач. Студент A решил 5 задач правильно и 4 неправильно. Студент B списал у студента A 3 задачи. Найти вероятность того, что:
    \begin{enumerate}
        \item все три списанные задачи были решены правильно.
        \item среди списанных задач правильно решена была только одна.
        \item среди списанных задач правильно решена была как минимум одна.
    \end{enumerate}

    \item В группе из 26 человек разыгрывают пять билетов на концерт на день Вышки. Аня и Борис договорились, что друг без друга на концерт не пойдут (а вместе -- пойдут). Найти вероятность того, что они одновременно выиграют или, наоборот, не выиграют билеты в этом розыгрыше.

    \item На консультацию записалось 4, 5 и 6 человек из первой, второй и третьей групп. (считаем поведение всех студентов независимым и случайным)
    \begin{enumerate}
        \item В результате на консультацию пришли 6 человек. Найти вероятность того, что из каждой группы пришло по 2 человека.
        \item На следующую консультацию записались те же студенты, но на консультацию пришло уже 7 человек. Найти вероятность того, что из каждой группы пришло не менее двух человек.
    \end{enumerate}

    \item Подброшено три кубика.
    \begin{enumerate}
        \item Найти вероятность того, что выпадет три разных числа.
        \item Найти вероятность того, что выпадет три четных числа.
        \item Найти вероятность того, что выпадет три разных четных числа.
        \item Найти вероятность того, что все три раза выпали разные числа, если известно, что все выпавшие числа четные.
        \item Известно, что все три раза выпадали разные числа. Найти вероятность того, что все выпавшие числа четные.
    \end{enumerate}

    \item В лекционной аудитории 20 рядов по 14 мест. В пустой аудитории случайным образом рассаживаются четыре человека. Найти вероятность того, что у них всех будут разные номера рядов и разные номера мест.

    \item 
    \begin{enumerate}
        \item Чему равна вероятность выиграть главный приз в лотерею «7 из 49»? В упрощенном варианте -- во время розыгрыша лотереи будут выбраны семь разных случайных чисел от 1 до 49, а вам их надо заранее угадать.
        \item Каждую неделю проходит один розыгрыш лотереи. Выпускник Вышки, прогуливавший теорвер, придумал план, надежный как швейцарские часы: он будет участвовать в лотерее много раз подряд, и когда-нибудь обязательно выиграет главный приз. Определите, сколько лет ему надо участвовать в лотерее, чтобы вероятность хотя бы одного выигрыша за это время стала больше 0.1?
    \end{enumerate}

    \item Вы решили узнать, есть ли среди ваших однокурсников кто-нибудь, у кого день рождения (без учета года) совпадает с вашим.
    Понятно, что если вы узнаете дату др только одного однокурсника, то вероятность того, что вы сразу нашли нужного, довольно мала. 
    Если вы узнаете дату др двоих однокурсников, то вероятность того, что среди них есть нужный, чуть выше. 
    Если бы на курсе училось 100-200-300 человек, то вероятность того, что среди них есть хотя бы один с нужной датой ДР, должна быть довольно большой.

    \begin{enumerate}
        \item У вас 215 однокурсников. Найти вероятность того, что среди них есть хотя бы один с нужной датой ДР.
        \item Сколько человек должно учиться вместе с вами на курсе, чтобы вероятность того, что среди них есть хотя бы один нужный вам, была больше 0.5?
    \end{enumerate}

    \item Нас интересует вероятность того, что в некоторой группе есть хотя бы два человека, у которых дни рождения приходятся на один день (то есть без учета года). Понятно, что если группа маленькая, то и вероятность такого совпадения маленькая, а если группа большая, то начиная с некоторого момента вероятность такого события равна 1.

    Сколько человек должно быть в группе, чтобы вероятность того, что в этой группе есть хотя бы два человека с совпадающими др, оказалась более 0.5?
    \textit{Указание: при решении этой задачи могут возникнуть вычислительные сложности, попробуйте их как-нибудь преодолеть.}
   
\end{enumerate}
\end{document}
