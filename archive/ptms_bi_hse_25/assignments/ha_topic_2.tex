\documentclass{article}
% \usepackage[T2A]{fontenc}
\usepackage[utf8]{inputenc}
\usepackage[english, russian]{babel}
\usepackage[top=1cm,bottom=1cm,left=2cm,right=2cm]{geometry}
\usepackage{graphicx}
\usepackage{amsmath}
\usepackage{amsfonts}
\usepackage{xcolor}
\title{ВШБ Бизнес-информатика: ТВиМС 2025. \\ Лист задач для самостоятельного решения \#2. \\ Геометрическая вероятность.}
\date{}
\author{}

\begin{document}
\maketitle

\begin{enumerate}
    \item Студент пишет два эссе – по философии и по праву.
    На философию он потратит от 80 до 120 минут, на право (независимо от философии) он потратит от 30 до 80 минут.
    Найти вероятность того, что на философию он потратит более чем в два раза больше времени, чем на право.

    \item На отрезке $[0,1]$ случайно и независимо выбраны два числа. Найти:
    \begin{enumerate}
        \item Вероятность того, что их сумма будет больше $1.5$.
        \item Вероятность того, что разница между ними будет меньше $0.2$.
        \item Вероятность того, что их произведение будет больше $0.25$.
    \end{enumerate}

    \item На отрезке $[0,2]$ случайным образом взяты два числа.
    Найти вероятность того, что они будут отличаться друг от друга не менее чем на $1$.

    \item Студенты А и Б договариваются о встрече в определенном месте между 13-00 и 14-00.
    Т.к. никто из них не знает, во сколько именно он окажется на месте встречи (и у них нет телефонов), то договор такой:
    каждый из них ждет другого максимум 20 минут, и, если второй за это время не появился – уходит.
    Найти вероятность того, что встреча произойдет.

    \item Решить предыдущую задачу, если А готов ждать 20 минут, а Б только 10.
    
    \item Смотря фильм, студент ставит свою кружку на скатерть в клетку.
    Сторона клетки равна 8 см, а радиус кружки – 3 см.
    Найти вероятность того, что кружка окажется внутри одной клетки – не пересечет границы клеток.

    \item Покупатель хочет купить на рынке килограмм яблок и килограмм груш.
    При этом заранее точная стоимость ему неизвестна, но для себя он решил,
    что будет выбирать и яблоки и груши стоимостью от 100 до 140 рублей за килограмм.
    \begin{enumerate}
    \item Найти вероятность события А - того, что он потратит больше 260 рублей.
    \item Найти вероятность события В – того, что за килограмм груш он заплатит более 130 рублей.
    \item Найти вероятность того, что произойдут оба этих события.
    \end{enumerate}

    Через эти три вероятности выразить и вычислить вероятность того,
    что произойдет I - хотя бы одно из этих событий, II - ровно одно из этих событий.
    \textit{(предполагаем, что цена за килограмм  – это произвольное число, распределенное равномерно на промежутке от 100 до 140 рублей)}

    \item Покупатель хочет купить на рынке яблоки, груши и персики – по одному килограмму.
    При этом заранее точная стоимость ему неизвестна, но для себя он решил, что будет выбирать фрукты стоимостью от 100 до 140 рублей за килограмм.
    Найти вероятность того, что суммарно он потратит \textbf{менее 380 рублей} на эти фрукты.
    \textit{(предполагаем, что цена за килограмм  – это произвольное число, распределенное равномерно на промежутке от 100 до 140 рублей)}
    
    \item Бесконечная плоскость разлинована  прямыми, находящимися на расстоянии $2$ друг от друга.
    Мы бросаем случайным образом иголку длины $1$ на эту плоскость.
    Найти вероятность того, что игла пересечет какую-нибудь из этих прямых
    (много параллельных линий на расстоянии $2$).
    
\end{enumerate}
\end{document}
