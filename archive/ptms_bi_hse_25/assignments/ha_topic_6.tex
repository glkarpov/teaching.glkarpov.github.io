\documentclass{article}
% \usepackage[T2A]{fontenc}
\usepackage[utf8]{inputenc}
\usepackage[english, russian]{babel}
\usepackage[top=1cm,bottom=1cm,left=2cm,right=2cm]{geometry}
\usepackage{graphicx}
\usepackage{amsmath}
\usepackage{amsfonts}
\usepackage{xcolor}
\title{ВШБ Бизнес-информатика: ТВиМС 2025. \\ Лист задач для самостоятельного решения \#6. \\ Непрерывные случайные величины. Специальные распределения: равномерное, экспоненциальное.}
\date{}
\author{}

\begin{document}
\maketitle

\textit{Медианой} случайной величины $X$ называется такая точка $m$, что $P(\{ X \leq m \}) = P(\{ X \geq m \}) = \frac{1}{2}$.

\textit{Модой} случайной величины $X$ называется такая точка, в которой функция плотности вероятности $f_X(x)$ достигает максимального значения.

\begin{enumerate}
\item Функция плотности вероятности $f_X(x)$ случайной величины $X$ имеет следующий вид:

    $$
        f_X(x) = 
        \begin{cases}
            0, \; x < 2 \\
            c, \; x \in [2,4] \\
            0, \; x > 4
        \end{cases}
    $$

    \begin{enumerate}
        \item Найдите нормировочную константу $c$,
        \item Найдите функцию распределения случайной величины $X$ и проверьте, что она корректна (поведения в пределах, неубывание),
        \item Вычислите математическое ожидание и дисперсию случайной величины $X$,
        \item Постройте графики функции распределения и функции плотности вероятности.
    \end{enumerate}


    % Lecture, task 2.5.3
    \item Функция плотности вероятности $f_X(x)$ случайной величины $X$ имеет следующий вид:

    $$
        f_X(x) = 
        \begin{cases}
            0, \; x < 0 \\
            cx^3, \; x \in [0,1] \\
            0, \; x > 1
        \end{cases}
    $$

    \begin{enumerate}
        \item Найдите нормировочную константу $c$,
        \item Найдите функцию распределения случайной величины $X$ и проверьте, что она корректна (поведения в пределах, неубывание),
        \item Вычислите математическое ожидание и дисперсию случайной величины $X$,
        \item Постройте графики функции распределения и функции плотности вероятности,
        \item Найдите вероятность $P(0 < X < 0.5 )$,
        \item Найдите моду и медиану случайной величины $X$.
    \end{enumerate}


    % Lecture, task 2.5.2
    \item Функция распределения $F_X(x)$ случайной величины $X$ имеет следующий вид:

    $$
        F_X(x) = 
        \begin{cases}
            0, \; x < 0 \\
            cx^3, \; x \in [0,1] \\
            1, \; x > 1
        \end{cases}
    $$

    \begin{enumerate}
        \item Найдите нормировочную константу $c$,
        \item Найдите функцию плотности вероятности случайной величины $X$ и постройте её график,
        \item Вычислите математическое ожидание и дисперсию случайной величины $X$,
        \item Постройте графики функции распределения и функции плотности вероятности,
        \item Найдите моду и медиану случайной величины $X$.
    \end{enumerate}

    % Lecture, task 2.5.4
    \item Функция плотности вероятности $f_X(x)$ случайной величины $X$ имеет следующий вид:

    $$
        f_X(x) = 
        \begin{cases}
            0, \; x < -1 \\
            0.5, \; x \in [-1, 0] \\
            0, \; x \in [0,1] \\
            0.5, \; x \in [1, 2] \\
            0, \; x > 2
        \end{cases}
    $$

    \begin{enumerate}
        \item Найдите функцию распределения случайной величины $X$ и проверьте, что она корректна (поведения в пределах, неубывание),
        \item Постройте графики функции распределения и функции плотности вероятности,
        \item Вычислите математическое ожидание и дисперсию случайной величины $X$,
    \end{enumerate}

    \item Случайная величина $X$ распределена с плотностью вероятности:
    $$
        f_X(x) = 
        \begin{cases}
            c \cdot (x - 1), \; x \in [1,4] \\
            0, \; x \notin [1,4]
        \end{cases}
    $$
    Найдите неизвестный параметр $c$, математическое ожидание, дисперсию, стандартное отклонение и вероятности:
    \begin{enumerate}
        \item $P(X = 3)$,  
        \item $P(X < 2)$, 
        \item $P(X > 3)$, 
        \item $P(|X - E(X)| < std(X))$, 
        \item $P(|X - E(X)| > 1.5 \cdot std(X) )$, 
    \end{enumerate}

    \item \textcolor{blue}{Расширение задачи из списка 5.} Заказы на доставку еды из ресторана образуют простейший поток с интенсивностью $6$ заказов в час.
    Найдите вероятность того, что:
    \begin{enumerate}
        \item за очередные $20$ минут поступит хотя бы $2$ заказа,
        \item за очередные $5$ минут – хотя бы $1$ заказ,
        \item время между двумя заказами окажется более $10$ минут.
    \end{enumerate}
    \textit{Указание: Задача на простейший поток, число событий – распределение Пуассона,
    время между событиями (или длительность события) – экспоненциальное распределение.}

    \item \textcolor{blue}{Расширение задачи из списка 5.} В пожарную часть поступает в среднем $8$ вызовов в сутки, вызовы образуют простейший поток.
    Найдите вероятность того, что
    \begin{enumerate}
        \item время ожидания очередного вызова превзойдет $4$ часа,
        \item время ожидания очередного вызова окажется менее $3$ часов,
        \item время ожидания очередного вызова окажется в пределах от $1$ часа до $5$ часов.
    \end{enumerate}
    
    
    \item Студент может либо дойти от метро до учебного корпуса пешком,
    время в дороге в этом случае распределено по экспоненциальному закону, причем в среднем студент тратит на дорогу $15$ минут
    (то есть матожидание времени в пути равно $15$ минутам), либо доехать на автобусе.
    Время ожидания автобуса распределено равномерно на отрезке от $0$ до $9$ минут, а время в пути равно трем минутам. Пешком он ходит в $70\%$ случаев.
    \begin{enumerate}
        \item Найдите вероятность того, что он потратит на дорогу более $10$ минут,
        \item Известно, что он потратил на дорогу более $10$ минут. Найдите вероятность того, что он ехал на автобусе.
    \end{enumerate}

    \textit{Указание: у нас две случайные величины: в первом случае $X \sim Exp(1/15)$ - это следует из условия и формулы $E[X] = \frac{1}{\lambda} = 15$, во втором случае $Y \sim Uniform[3;12]$ - это время ожидания+время поездки.}
    
    \item Время приема пациента доктором – это случайная величина, распределенная по экспоненциальному закону со средним значением $10$ минут.
    На завтра у доктора записано $20$ человек.
    Найдите вероятность того, что у половины из них время приема окажется больше среднего (больше $10$ минут).

    \item Заказы на доставку еды из ресторана образуют простейший поток с интенсивностью $6$ заказов в час.
    Найдите вероятность того, что очередной заказ поступит более чем через $k$ минут, если предыдущие $n$ минут заказы не поступали.    
    Простыми словами – найдите условную вероятность $P\left\{X > n + k \mid X > n\right\}$.

    \item Для данной случайной величины найти вероятность того, что она отклонится от своего матожидания более чем на три стандартных отклонения.
    \begin{enumerate}
        \item  $X \sim Bin(7,0.2)$
        \item  $Y \sim Pois(3)$
        \item  $U \sim Uniform[2,5]$
        \item  $T \sim Exp(3)$
    \end{enumerate}

\end{enumerate}
\end{document}
