\documentclass{article}
% \usepackage[T2A]{fontenc}
\usepackage[utf8]{inputenc}
\usepackage[english, russian]{babel}
\usepackage[top=1cm,bottom=1cm,left=2cm,right=2cm]{geometry}
\usepackage{graphicx}
\usepackage{amsmath}
\usepackage{amsfonts}
\usepackage{xcolor}
\title{ВШБ Бизнес-информатика: ТВиМС 2025. \\ Лист задач для самостоятельного решения \#3b. \\ Полная  вероятность. Теорема Байеса.}
\date{}
\author{}

% Размер промежутка между задачами для вписывания решений
\newcommand{\problemspace}{6cm}

\begin{document}
\maketitle

\begin{enumerate}
    \item Торговая сеть планирует открытие нового супермаркета.
    Если рядом будут находиться магазины конкурентов, то супермаркет окажется прибыльным с вероятностью $0.3$.
    Если конкурентов рядом не будет, то прибыльным он окажется с вероятность $0.4$.
    Вероятность найти помещение, рядом с которым не будет конкурентов, равна $0.8$.
    \begin{enumerate}
        \item Найти вероятность того, что новый супермаркет окажется прибыльным.
        \item Известно, что новый супермаркет оказался прибыльным. Найти вероятность того, что рядом нет магазинов конкурентов.
    \end{enumerate}

    \vspace{\problemspace}

    % source: Athena Scientific, Section 1.3, Problem 16
    \item Есть три монеты: у одной с обеих сторон орел, у второй с обеих сторон решка, у третьей орел с одной стороны и решка с другой.
    Мы выбираем монету наугад, подбрасываем её, и выпадает орел. Какова вероятность того, что на противоположной стороне решка?

    $C_1, \, C_2, \, C_3$ – события, соответствующие выбору монеты согласно условиям задачи.
    $A$ – событие, соответствующее выпадению орла.
    $$
    P(A) = P(A|C_1)P(C_1) + P(A|C_2)P(C_2) + P(A|C_3)P(C_3) = 1 \cdot \frac{1}{3} + 0 \cdot \frac{1}{3} + 0.5 \cdot \frac{1}{3} = \frac{1}{3} + \frac{1}{6} = \frac{1}{2}
    $$
    Согласно условию, нам нужно узнать вероятность того, что была выбрана третья монета.
    $$
    P(C_3|A) = \frac{P(A|C_3)P(C_3)}{P(A)} = \frac{0.5 \cdot \frac{1}{3}}{\frac{1}{2}} = \frac{1}{3}
    $$
    
    \item В урне $A$ находятся 2 белых шара и 1 черный шар, а в урне $B$ находятся 1 белый шар и 5 черных.
    Один случайный шар из урны $A$ помещают в урну $B$. Затем из урны $B$ вынимается шар, который оказывается белым.
    Какова вероятность того, что переложенный шар был белым?

    $W$ – событие, соответствующее выпадению белого шара. $TW$ – событие, соответствующее тому, что переложенный шар был белым. $TB$ – событие, соответствующее тому, что переложенный шар был черным.
    $$
    P(W) = P(W|TW)P(TW) + P(W|TB)P(TB) = \frac{2}{7} \cdot \frac{2}{3} + \frac{1}{7} \cdot \frac{1}{3} = \frac{4}{21} + \frac{1}{21} = \frac{5}{21}
    $$
    $$
    P(TW \,| \,W) = \frac{P(W|TW)P(TW)}{P(W)} = \frac{\frac{4}{21}}{\frac{5}{21}} = \frac{4}{5}
    $$
\newpage

    \item При приеме на работу $10\%$ кандидатов сообщают ложную информацию о себе.
    Для выявления таких случаев работодатель проводит дополнительное психологическое тестирование.
    Однако тест не является абсолютно точным – если соискатель лгал, тест в $5\%$ случаев этого не определит.
    С другой стороны, у теста бывают и «ложные срабатывания» - в $20\%$ случаев человека, сообщившего только достоверную информацию, он примет за лжеца. 
    На очередном тестируемом тест сработал, то есть указал, что соискатель лжет. Найти вероятность того, что соискатель действительно лгал.

    \vspace{\problemspace}

    \item У студента ВШБ на Шаболовке $27\%$ занятий проходят в пятом корпусе, $24\%$ в третьем, $20\%$ в четвертом, а остальные занятия проходят в К9 и К10.
    Если занятие проходит в пятом корпусе, то с вероятностью $0.3$ студент займет место в первом ряду.
    Если занятие проходит в третьем корпусе, то вероятность занять место в первом ряду для него равна $0.2$, в четвертом $0.1$,
    в К9 и К10 вероятность равна $0.01$.
    Во время пары на Шаболовке студент отвечает на входящий звонок: "Извини, не могу разговаривать, сижу в первом ряду".
    Найти вероятность того, занятие проходит в четвертом корпусе.

    \vspace{\problemspace}

    \item Компания А собирается выводить на рынок новую игровую приставку. Ее конкуренты – компании В и С, работающие независимо друг от друга, каждая из которых может опередить А с вероятностью $0.6$.
    Если у приставки А на момент выхода в продажу не будет конкурентов, то она окажется успешной с вероятностью $0.9$,
    если к этому моменту свою приставку успеет выпустить только один конкурент, то вероятность успеха равна $0.7$,
    если же на рынке будут оба конкурента – вероятность успеха А равна $0.4$.
    \begin{enumerate}
        \item Найти вероятность того, что продукт окажется успешным.
        \item Известно, что продукт оказался успешным. Найти вероятность того, что это произошло при отсутствии конкурентов.
        \item Известно, что продукт не оказался успешным. Найти вероятность того, что только один из конкурентов  успел обогнать А.
    \end{enumerate}

\newpage

    \item Если в день перед экзаменом Вася идет только в кино, то сдает экзамен с вероятностью $0.6$.
    Если в день перед экзаменом Вася идет только в бар, то сдает экзамен с вероятностью $0.5$.
    Если в день перед экзаменом Вася идет и в кино, и в бар – то сдает экзамен с вероятностью $0.1$.
    Если в день перед экзаменом Вася сидит дома, то сдает экзамен с вероятностью $0.9$.
    В кино и в бар Вася ходит независимо друг от друга и от других событий, с вероятностями $0.2$ и $0.6$ соответственно.
    \begin{enumerate}
        \item Найти вероятность того, что Вася сдаст экзамен.
        \item Найти вероятность того, что Вася вчера был и в баре и в кино, если известно, что он сдал экзамен.
        \item Найти вероятность того, что Вася вчера был в баре, если известно, что он сдал экзамен.
        \item Вася сдал экзамен. Что более вероятно – то, что он вчера был в баре, или то, что он вчера был в кино?
    \end{enumerate}

    \newpage

    \item Преподаватель теорвера проводит опрос о том, насколько студенты заинтересованы в изучении предмета.
    Однако студенты не уверены в полной анонимности, поэтому преподаватель понимает, что:
    если студент написал, что его интересует предмет – с вероятностью $0.2$ он ответил неправду,
    если же студент написал, что его не интересует предмет, то это честный ответ.
    По результатам опроса оказалось, что $30\%$ студентов написали, что предмет их не интересует.
    Преподаватель знает, что некоторый студент точно не  интересуется предметом. Найти вероятность того, что этот студент честно ответил на вопрос.

    \vspace{\problemspace}

    % \item Компании необходимо получить лицензию на строительство завода. Лицензионная комиссия состоит из двух инспекторов, которые
    % будут независимо оценивать проект завода. Чтобы получить лицензию, оба инспектора должны
    % дать положительную оценку.
    % Компания знает, что каждый из инспекторов даст положительную оценку с вероятностью $70\%$,
    % независимо от другого инспектора.
    % Однако, если компания даст взятку инспектору, и он ее примет,
    % вероятность получить положительную оценку увеличится до $90\%$ (но все еще есть $10\%$ вероятность, что
    % инспектор возьмет взятку и даст отрицательную оценку). Каждый инспектор примет взятку
    % с вероятностью $50\%$. Если инспектор не примет взятку, он даст положительную оценку с вероятностью $70\%$.
    % \begin{enumerate}
    %     \item  Какова вероятность того, что компания получит лицензию, если она не будет давать взяток?
    %     \item  Какова вероятность того, что компания получит лицензию, если она попытается дать взятку обоим инспекторам?
    %     \item Предположим, что один из инспекторов принял взятку, а другой нет, но оказалось, что
    % компания не получила лицензию. Учитывая эту информацию, найдите условную вероятность того, что
    % подкупленный инспектор дал отрицательную оценку.
    % \end{enumerate}


    \item Школьник Вася должен написать и сдать работу по географии.
    С вероятностью $0.2$ он этого делать не будет. Если же он напишет работу, то с вероятностью $0.4$ работу съест его собака.
    Если собака не сделает этого, то с вероятностью $0.6$ Вася забудет взять работу в школу.
    Если Вася возьмет работу в школу, то с вероятностью $0.1$ потеряет ее по дороге. Если он донесет работу до школы, то сдаст ее.
    \begin{enumerate}
        \item Найти вероятность того, что Вася сдаст работу.
        \item Известно, что Вася не сдал работу. Найти вероятность того, что ее съела собака.
    \end{enumerate}

    \vspace{\problemspace}

    \item Студент сдает экзамен с вероятностью $0.8$.
    Если этого не происходит, то он отправляется на пересдачу, и вероятность сдать становится равной $0.6$.
    Если он не сдает и со второго раза, то отправляется на комиссию, и там сдает с вероятностью $0.7$,
    и если снова не сдал – получает ИУП.
    Известно, что студент сдал экзамен, то есть дело не дошло до ИУПа. Найти вероятность того, что экзамен был сдан со второй попытки.
\end{enumerate}

\end{document}
