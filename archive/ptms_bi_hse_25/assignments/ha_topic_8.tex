\documentclass{article}
% \usepackage[T2A]{fontenc}
\usepackage[utf8]{inputenc}
\usepackage[english, russian]{babel}
\usepackage[top=1cm,bottom=1cm,left=2cm,right=2cm]{geometry}
\usepackage{graphicx}
\usepackage{amsmath}
\usepackage{amsfonts}
\usepackage{xcolor}
\title{ВШБ Бизнес-информатика: ТВиМС 2025. \\ Лист задач для самостоятельного решения \#8. \\ Многомерные случайные величины. Ковариация и корреляция.}
\date{}
\author{}

\begin{document}
\maketitle

\begin{enumerate}



    \item Дана совместная функция вероятностей случайного вектора $(X,Y)$:
     \begin{center}
         \begin{tabular}{c | c | c | c |} 
         
          & $Y=0$ & $Y=1$ & $Y=3$\\ [0.5ex] 
          \hline
         $X = -2$ & $0.01$ & $0.4$ & $0.03$ \\ 
         $X = -1$ & $0.01$ & $c$ & $0.2$ \\ 
         $X = 0$ & $0.02$ & $0.01$ & $0.02$ \\ 
         \hline
        \end{tabular}
    \end{center}
    
    \begin{enumerate}
        \item Проверьте, являются ли переменные независимыми, 
        \item Найдите математическое ожидание и дисперсию случайной величины $X$, ковариацию и коэффициент корреляции для $X$ и $Y$,
        \item Найдите условную функцию вероятностей для $Y$ при условии $X = -1$,
        \item Найдите вероятность $P \{ 2X + Y < 0 \}$
        \item Рассмотрите новую случайную величину $T = Y^2 - X$, найдите её функцию вероятностей.
    \end{enumerate}

    \item Дана совместная функция вероятностей случайного вектора $(X,Y)$:
    
        \begin{center}
             \begin{tabular}{c | c c} 
            & \(Y=-1\) & \(Y=1\) \\
            \hline
        \(X=0\) & $\frac{1}{2}$ & \(0\) \\
        \(X=1\) & $\frac{1}{3}$ & \(\frac{1}{6}\) \\
        
            \end{tabular}
        \end{center}
    
    Найдите ковариацию и корреляцию между $X$ и $Y$.

\item Дана совместная функция вероятностей случайного вектора $(X,Y)$ следующего вида:


    \begin{center}
         \begin{tabular}{c | c c c} 
        & \(Y=-2\) & \(Y=-1\) & \(Y=3\) \\
        \hline
        \(X=-2\) & \(0.15\) & \(0.15\) & \(c\) \\
        \(X=1\) & \(0.05\) & \(0.2\) & \(0.15\) \\
    
        \end{tabular}
    \end{center}


\begin{itemize}
    \item Найдите вероятность $P(Y > -1 )$,
    \item Найдите вероятность $P(Y > X )$,
    \item Найдите вероятность $P(\{ X = -2 \} \cap \{ Y < 0 \})$,
    \item Найдите ковариацию $\text{Cov}(X,Y)$,
    \item Найдите корреляцию $\text{Corr}(X,Y)$.
\end{itemize}

\item Совместное распределение случайных величин $X$ и $Y$ задано таблицей

    \begin{center}
    \begin{tabular}{c | c c c}
    \hline & $X=-2$ & $X=0$ & $X=2$ \\
    \hline
    $Y=1$ & 0.2 & 0.3 & 0.1 \\
    $Y=2$ & 0.1 & 0.2 & $a$ \\
    \hline
    \end{tabular}
    \end{center}
    \begin{enumerate}
        \item Найдите неизвестную вероятность $a$.
        \item Проверьте, являются ли $X$ и $Y$ независимыми,
        \item Найдите вероятности $P(X>-1), P(X>Y)$
        \item Найдите условную функцию вероятностей для $X$ при условии $Y = 2$ и для $Y$ при условии $X = 0$,
        \item Найдите корреляцию $\text{Corr}(X, Y)$
    \end{enumerate}

    % src: Grimmet Welsh, Chapter 3, Problem 9 (Oxford 1980)
    \item Случайные величины $U$ и $V$ принимают значения $\pm 1$. Их совместное распределение задано следующим образом:

    $$
        P\{U = -1 \} = P\{ U = +1 \} = \frac{1}{2},
    $$


    \begin{align*}
         P\{V = + 1 \; | \; U = +1 \} & = P\{V = -1 \; | \; U = -1 \} = \frac{1}{3}, \\
         P\{V = -1 \; | \; U = +1 \} & = P\{V = +1 \; | \; U = -1 \} = \frac{2}{3}, \\
    \end{align*}

    \begin{enumerate}
        \item Найдите вероятность того, что уравнение $x^2 + U x + V = 0$ имеет хотя бы один действительный корень.
        \item Найдите вероятность того, что уравнение $x^2 + (U + V) x + (U + V) = 0$ имеет хотя бы один действительный корень.
    \end{enumerate}

    \item В группе $5$ мальчиков и $8$ девочек. Из этой группы мы случайным образом выбираем троих учеников.
        Пусть $X$ это количество мальчиков в выборке, а $Y$ – количество девочек.
        \begin{enumerate}
            \item Построить таблицу совместного распределения
            \item Найти математическое ожидание и дисперсию $X$ и $Y$
            \item Найти $\text{Cov}(X, Y)$.
        \end{enumerate}

\item Пусть известно распределение случайной величины $U$:
\begin{center}
    \begin{tabular}{c | c c c c} 
       $U$ & $u_1=0$ & $u_2 = \pi/3$ & $\ldots$ & $u_{6} = 5 \pi/3$ \\
       \hline
       $P_U$ &  $\frac{1}{6}$  &   $\ldots$  & $\ldots$ & $\frac{1}{6}$ \\
       \hline     
    \end{tabular}
    \end{center}
    
    \begin{enumerate}
            \item Введём переменные $X = \cos{U}$ и $Y = \sin{U}$. Постройте таблицу их совместного распределения.
            \item Являются ли они зависимыми? Рассмотрите, например, $P(X = \frac{1}{2})$ и $P(X = \frac{1}{2} \; | \; Y = \frac{\sqrt{3}}{2})$.
            \item Найдите математические ожидания $E[X]$ и $E[Y]$.
            \item Найдите ковариацию и корреляцию между $X$ и $Y$.
    \end{enumerate}


    \item Подброшены два кубика. Пусть $X$ и $Y$ это количество очков, выпавших на первом и втором кубиках соответственно.
        Рассмотрим две случайные величины: $U = X + Y$ и  $V = X - Y$.
        \begin{enumerate}
            \item Скоррелированы ли $U$ и $V$?
            \item Зависимы ли $U$ и  $V$?
        \end{enumerate}

    \item Для случайных величин $X$ и $Y$ заданы следующие значения:
    $$
    E[X]=1, \; E[Y]=4, \; E[X Y]=8, \; Var[X]=Var[Y]=9.
    $$
    
    Для случайных величин $U = X + Y$ и $V = X - Y$ вычислите:
    \begin{enumerate}
        \item $E[U], Var[U], E[V], Var[V], \text{Cov}(U, V)$
        \item Можно ли утверждать, что случайные величины $U$ и $V$ независимы?
    \end{enumerate}

    \item
    \begin{enumerate}
        \item Случайная величина $X$ принимает значения $-2$, $-1$, $0$, $1$ и $2$ с равными вероятностями $0.2$. Найти коэффициент корреляции между $X$ и $Y = X^2$.
        \item Случайная величина $X$ принимает значения $-1$, $0$, $1$, $2$ и $3$ с равными вероятностями $0.2$. Найти коэффициент корреляции между $X$ и $Y = X^2$.
        \item Прокомментируйте результаты.
    \end{enumerate}
    
\end{enumerate}
\end{document}
