\documentclass{article}
% \usepackage[T2A]{fontenc}
\usepackage[utf8]{inputenc}
\usepackage[english, russian]{babel}
\usepackage[top=1cm,bottom=1cm,left=2cm,right=2cm]{geometry}
\usepackage{graphicx}
\usepackage{amsmath}
\usepackage{amsfonts}
\usepackage{xcolor}
\title{ВШБ Бизнес-информатика: ТВиМС 2025. \\ Лист задач для самостоятельного решения \#9. \\ Центральная предельная теорема. Интегральная теорема Муавра-Лапласа.}
\date{}
\author{}

\begin{document}
\maketitle

\begin{enumerate}
    \item Студент играет в карты на деньги. Он может выиграть $15$р с вероятностью $0.8$, проиграть $100$р с вероятностью $0.1$, не выиграть и не проиграть (получить $0$) – с вероятностью $0.1$.
    Найти вероятность того, что по результатам $40$ игр он окажется в проигрыше, если результаты всех игр не зависят друг от друга.

    \item Рост студента $191$ см. Он складывает детальки конструктора Лего в башенку.
    Ему известно, что высота каждой детали – это случайная величина, распределенная по равномерному закону на промежутке от $9$ мм до $11$ мм независимо от остальных деталей.
    Найти вероятность того, что высота башенки, состоящей из $190$ деталей, превысит его рост.

    \item Число замечаний, полученных студентом за одну пару, распределено по закону Пуассона с параметром $\lambda = 2$.
    В месяце было $20$ учебных дней, каждый день – по четыре пары.
    Найти вероятность того, что за этот месяц студенту было сделано не менее $140$ замечаний, если все замечания, которые он получает, не зависят друг от друга.

    \item Число аварий на оживленном перекрестке в течение суток распределено по закону Пуассона со средним значением $3$ аварии  в сутки. В городе $40$ таких перекрестков.
    \begin{enumerate}
        \item Найти вероятность того, что за сутки произойдет более $130$ аварий на этих перекрестках.
        \item Найти вероятность того, что на $6$ перекрестках произойдет по $4$ аварии за сутки.
    \end{enumerate}
    \textit{Считаем, что соответствующие случайные величины независимы.}

    \item Время за неделю, посвященное студентом выполнению дз, распределено по экспоненциальному закону, причем известно, что в среднем в неделю студент на дз тратит $30$ минут.
    Найти вероятность того, что за учебный год ($40$ недель) общее время, потраченное студентом на дз, превысит $24$ часа.

    \textit{Считаем, что соответствующие случайные величины независимы.}
    
    \item Школьник заполняет тест, состоящий из $160$ вопросов, к каждому из которых есть по $4$ ответа.
    Так как он не готовился, то расставляет ответы случайным образом и независимо от других ответов.
    Оценка выставляется пропорционально количеству правильных ответов. Найти вероятность того, что он получит зачетную оценку (от $3.5$).

    \item На городской студенческой ярмарке вакансий $3$ маркетинговые компании набирают себе стажеров. Всего сразу на все $3$ собеседования в эти компании записалось $130$ студентов.
    Для каждого из них вероятность провалиться на очередном  собеседовании равна $0.3$ и не зависит от других студентов и остальных собеседований.
    Найти вероятность того, что в результате прохождения всеми студентами всех собеседований, число студентов, не прошедших хотя бы одно из них, окажется в пределах от $70$ до $90$.

    \item Небольшой город посещают $100$ туристов в день, каждый из которых обедает в одном из двух городских ресторанов – независимо от других туристов и с равной вероятностью для каждого из ресторанов.
    Один из владельцев хочет, чтобы с вероятностью $\approx 0.99$ все пришедшие в его ресторан туристы могли одновременно пообедать.
    Сколько мест должно быть в его ресторане?

    \newpage
    \item Время доставки продуктов из магазина  это случайная величина, распределенная равномерно на отрезке от $30$ минут до трех с половиной часов.
    За один день было выполнено $300$ заказов.
    \begin{enumerate}
        \item Найдите вероятность того, что общее время, потраченное на выполнение этих заказов, окажется более $620$ часов.
        \item Найдите вероятность того, что более трети из этих заказов были доставлены быстрее, чем за $100$ минут.
    \end{enumerate}

    \textit{Считаем, что соответствующие случайные величины независимы.}

    \item В магазине продаются новогодние подарочные наборы, стоимость набора это случайная величина,
    принимающая значения от $1000$ до $4000$ рублей и имеющая плотность распределения
    
    \begin{equation*}
        f_X(x)=
        \begin{cases}
        c(x-1000), & x \in [1000, 4000] \\
        0, & x \notin [1000, 4000]
        \end{cases}
    \end{equation*}
    Всего за предновогодний сезон было продано $5000$ таких наборов. 
    \begin{enumerate}
        \item Найдите вероятность того, что общая стоимость проданных наборов оказалась более $14,950,000$ рублей.
        \item Найдите вероятность того, что было продано более $2813$ наборов стоимостью более $3000$ каждый.
    \end{enumerate}
    \textit{Считаем, что соответствующие случайные величины независимы.}
\end{enumerate}

\end{document}
