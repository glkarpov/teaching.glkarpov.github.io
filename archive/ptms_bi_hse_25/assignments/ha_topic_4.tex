\documentclass{article}
% \usepackage[T2A]{fontenc}
\usepackage[utf8]{inputenc}
\usepackage[english, russian]{babel}
\usepackage[top=1cm,bottom=1cm,left=2cm,right=2cm]{geometry}
\usepackage{graphicx}
\usepackage{amsmath}
\usepackage{amsfonts}
\usepackage{xcolor}
\title{ВШБ Бизнес-информатика: ТВиМС 2025. \\ Лист задач для самостоятельного решения \#4. \\ Дискретные случайные величины и их характеристики.}
\date{}
\author{}

% Размер промежутка между задачами для вписывания решений
\newcommand{\problemspace}{6cm}

\begin{document}
\maketitle

\begin{enumerate}
    \item Есть игра, в результате которой мы можем проиграть 5 рублей с вероятностью $0.1$,
сыграть в ноль (то есть ничего не получить, но и ничего не потерять) с вероятностью $0.2$
и выиграть 1 рубль с вероятностью $0.7$.
    \begin{enumerate}

        \item Постройте таблицу функции вероятности (ряд распределения) для выигрыша в одной игре, найдите математическое ожидание выигрыша.
        Является ли игра для нас плюсовой? Т.е. если мы будем долго в нее играть – мы в результате будем больше выигрывать или проигрывать?
        Найдите дисперсию и стандартное отклонение выигрыша в одной партии (если мы проиграли, то будем говорить, что наш выигрыш равен $-5$).

        \item Запишите функцию распределения $F_X(x)$, нарисуйте график, используя функцию распределения найдите вероятность того, что наш выигрыш окажется в промежутке $(-0.5; 1.2]$.

        \item Найдите вероятность того, что выигрыш отклонится от своего математического ожидания менее чем на стандартное отклонение.
    \end{enumerate}

    \vspace{\problemspace}

    % source: Ross, Problems, 4.19
    \item Если функция распределения случайной величины $X$ задана следующим образом:
    $$
    F(b)= \begin{cases}0 & b<0 \\ \frac{1}{2} & 0 \leq b<1 \\ \frac{3}{5} & 1 \leq b<2 \\ \frac{4}{5} & 2 \leq b<3 \\ \frac{9}{10} & 3 \leq b<3.5 \\ 1 & b \geq 3.5\end{cases}
    $$

    Постройте таблицу функции вероятности $X$ (ряд распределения). Посчитайте математическое ожидание и дисперсию.

    \textbf{Решение:}
    
    Для дискретной случайной величины вероятность в точке $x$ равна скачку функции распределения в этой точке: $P_X(X = x) = F(x) - \lim_{b \to x^-} F(b)$.
    
    Найдём скачки функции распределения:
    \begin{itemize}
        \item $P_X(X = 0) = F(0) - \lim_{b \to 0^-} F(b) = 0.5$
        \item $P_X(X = 1) = F(1) - F(0) = 0.1$
        \item $P_X(X = 2) = F(2) - F(1) = 0.2$
        \item $P_X(X = 3) = F(3) - F(2) = 0.1$
        \item $P_X(X = 3.5) = F(3.5) - F(3) = 0.1$
    \end{itemize}
    
    Ряд распределения:
    \begin{center}
        \begin{tabular}{c | c c c c c} 
        \hline
        $x$ & $0$ & $1$ & $2$ & $3$ & $3.5$ \\ [0.5ex] 
        \hline
        $p_X(x)$ & $0.5$ & $0.1$ & $0.2$ & $0.1$ & $0.1$ \\ 
        \hline
    \end{tabular}
    \end{center}
    
    Математическое ожидание:
    $$E[X] = 0 + 0.1 + 0.4 + 0.3 + 0.35 = 1.15$$

    Дисперсия:
    \begin{gather*}
        E[X^2] = 0 + 0.1 + 0.8 + 0.9 + 1.225 = 3.025\\
        Var[X] = E[X^2] - (E[X])^2 = 3.025 - (1.15)^2 = 1.7025
    \end{gather*}

    \vspace{\problemspace}
    \newpage
    
    \item Заполните ряд распределения до конца, если известно, что $E[X] = 0$.
        \begin{center}
            \begin{tabular}{c | c c c c} 
            \hline
            $x$ & -1 & 0 & $a$ & $4$ \\ [0.5ex] 
            \hline
            $p_X(x)$ & $0.6$ & $p_1$ & $0.1$ & $0.1$ \\ 
            \hline
        \end{tabular}
    \end{center}

    \vspace{\problemspace}

    \item Найдите $a$, $p_1$ и $p_2$, если известно, что $E[X] = 0$ и $Var[X] = 5.4$.
        \begin{center}
            \begin{tabular}{c | c c c c c} 
            \hline
            $x$ & -2 & -1 & 0 & $a$ & $4$ \\ [0.5ex] 
            \hline
            $p_X(x)$ & $0.4$ & $0.2$ & $p_1$ & $0.1$ & $p_2$ \\ 
            \hline
        \end{tabular}
    \end{center}

    \vspace{\problemspace}
    \newpage

    % source: ICEF, B.2.40
    \item Пусть $X$ — случайная величина с $E[X^2] = 3.6$, $P_X(X = 2) = 0.6$ и $P_X(X = 3) = 0.1$.
    Случайная величина $X$ принимает только одно другое значение между $0$ и $3$. Найдите дисперсию $X$.

    Составим уравнение для дисперсии:
    \begin{gather*}
        E[X^2] = 4 \cdot 0.6 + 9 \cdot 0.1 + x^2 \cdot 0.3 = 2.4 + 0.9 + 0.3x^2 = 3.3 + 0.3x^2\\
        E[X^2] = 3.6 \Rightarrow 3.3 + 0.3x^2 = 3.6 \Rightarrow 0.3x^2 = 0.3 \Rightarrow x^2 = 1 \Rightarrow x = 1
    \end{gather*}

    Ряд распределения:
    \begin{center}
        \begin{tabular}{c | c c c} 
        \hline
        $x$ &  1 & 2 & 3 \\ [0.5ex] 
        \hline
        $p_X(x)$ & $0.3$ & $0.6$ & $0.1$ \\ 
        \hline
    \end{tabular}
    \end{center}
    Математическое ожидание: $E[X] = 0.3 + 1.2 + 0.3 = 1.8$

    Дисперсия: $Var[X] = E[X^2] - (E[X])^2 = 3.6 - (1.8)^2 = 0.36$


    \vspace{\problemspace}

    \item В урне находятся $2$ черных и $5$ белых шаров. Вы случайным образом выбираете $3$ из них.
    $X$ — это количество черных шаров в вашей выборке. Найдите математическое ожидание и дисперсию $X$.

    \begin{itemize}
        \item $X=0$, означает 0 черных шаров в выборке. Вероятность этого события: $P(X=0) = \frac{C_5^3}{C_7^3} = \frac{10}{35} = \frac{2}{7}$
        \item $X=1$, означает 1 черный шар в выборке. Вероятность этого события: $P(X=1) = \frac{C_2^1 \cdot C_5^2}{C_7^3}  = \frac{4}{7}$
        \item $X=2$, означает 2 черных шара в выборке. Вероятность этого события: $P(X=2) = \frac{C_2^2 \cdot C_5^1}{C_7^3}  = \frac{1}{7}$
    \end{itemize}

    Можем построить таблицу функции вероятности (ряд распределения):
    \begin{center}
        \begin{tabular}{c | c c c} 
        \hline
        $x$ & 0 & 1 & 2 \\ [0.5ex] 
        \hline
        $p_X(x)$ & $\frac{2}{7}$ & $\frac{4}{7}$ & $\frac{1}{7}$ \\ 
        \hline
    \end{tabular}
    \end{center}

    Математическое ожидание: $E[X] = 0 + \frac{4}{7} + \frac{2}{7} = \frac{6}{7}$

    Дисперсия:
    \begin{gather*}
        E[X^2] = 0 + \frac{4}{7} + \frac{4}{7} = \frac{8}{7} = \frac{56}{49}\\
        Var[X] = E[X^2] - (E[X])^2 = \frac{56}{49} - \left(\frac{6}{7}\right)^2 = \frac{56}{49} - \frac{36}{49} = \frac{20}{49}
    \end{gather*}

    \newpage
    
    \item В клубе $10$ хороших стрелков и $3$ плохих. На очередное соревнование случайным образом отбирается команда из трех человек.
    \begin{enumerate}
        \item Пусть случайная величина $X$ – число плохих стрелков, попавших в команду. Для всех возможных значений $X$ найдите значения функции вероятности $P_X(x)$, также посчитайте $E[X]$ и $Var[X]$.
        \item Пусть случайная величина $Y$ – число хороших стрелков, попавших в команду. Постройте ряд распределения случайной величины $Y$, посчитайте $E[Y]$ и $Var[Y]$.
        \item Сравните результаты.
    \end{enumerate}

    \vspace{\problemspace}

    \item Три рассеянных математика, уходя из гостей, произвольным образом забирают по зонту (всего их  было три – каждый из математиков пришел со своим зонтом).
    Пусть $X$ – число математиков, забравших именно свой зонт.
    Постройте ряд распределения, найдите математическое ожидание и стандартное отклонение случайной величины $X$.

    \vspace{\problemspace}
    \newpage

    \item Цены на акции компаний A, B и C растут независимо друг от друга с вероятностями $0.3$, $0.4$ и $0.8$ соответственно.
    Пусть $X$ – число тех компаний среди этих трех, чьи акции выросли.
    \begin{enumerate}
        \item Постройте ряд распределения, найдите математическое ожидание и стандартное отклонение.
        \item Найдите вероятность того, что $X$ отклонится от своего математического ожидания более чем на одно стандартное отклонение.
    \end{enumerate}

    \vspace{\problemspace}

    \item При обсуждении любимой кофейни около Вышки студенты выяснили следующее:
    \begin{enumerate}
    \item Андрей по дороге в Вышку покупает кофе с вероятностью $1/3$, независимо от этого по дороге домой он покупает кофе с вероятностью $1/4$,
    Пусть случайная величина $X$ это то, сколько раз за день Андрей купил кофе.
    
    \item Борис по дороге вперед покупает кофе с вероятностью $1/3$, по дороге домой он покупает кофе с вероятностью $1/4$.
    Так же известно, что с вероятностью $1/5$ он купит кофе как по дороге вперед, так и по дороге домой.
    Пусть случайная величина $Y$ это то, сколько раз за день Борис купил кофе.
    
    \item Валерия  по дороге вперед покупает кофе с вероятностью $1/3$ (это всегда большая порция Американо за $240$ рублей),
    независимо от этого по дороге домой она покупает кофе с вероятностью $1/4$ (маленькая порция Капучино за $90$ рублей). 
    Пусть случайная величина $W$ это то, сколько денег за день Валерия потратила на кофе.
    
    \end{enumerate}

    Найти матожидание и стандартное отклонение упомянутых случайных величин.
    (Будем считать, что в других ситуациях люди в кофейню не заходят, нигде кроме как в этой кофейне кофе не покупают)

    \vspace{\problemspace}
    \newpage
    
    % source: Ross, theoretical, 4.6
    \item Пусть вероятности и значения случайной величины $X$ заданы следующим образом:
    $$
        P_X(X = 1) = p = 1 - P_X(X = -1)
    $$

    Найдите такую константу $c$, $c \neq 1$, чтобы случайная величина $Y = f(X) = c^X$ имела математическое ожидание $1$.

    \begin{gather*}
        E[Y] = c^1 \cdot p + c^{-1} \cdot (1 - p) = 1\\
        c^1 p + c^{-1} (1 - p) = 1\\
        c p + \frac{(1 - p)}{c} = 1\\
        \frac{pc^2 -c + 1 - p}{c} = 0\\
        pc^2 -c + 1 - p = 0
    \end{gather*}

    В итоге получим $c=1$ или $c = \frac{1-p}{p}$.

    \item  Задача важная на будущее :)
    
    Пусть $X$ - дискретная случайная величина с $E[X] = a$ и $Var[X] = b^2 \neq 0$. Мы строим новую случайную величину,
    которая является функцией от $X$:

    $$
        Y = g(X) =  \frac{X - a}{b}
    $$

    Найдите $E[Y]$ и $Var[Y]$.

    \textbf{Подсказка:} можно попробовать разные способы: или использовать готовые свойства математического ожидания и дисперсии,
    или напрямую поработать с их формулами.

    \begin{gather*}
        E[Y] = E \left[ \frac{X - a}{b} \right] = E \left[ \frac{X}{b} + \frac{-a}{b}\right] =  \frac{1}{b}E[X] - \frac{a}{b} = \frac{a}{b} - \frac{a}{b} = 0\\
        Var[Y] = Var \left[ \frac{X - a}{b} \right] = Var \left[ \frac{X}{b} + \frac{-a}{b}\right] = \frac{1}{b^2}Var[X] = \frac{b^2}{b^2} = 1
    \end{gather*}
\end{enumerate}

\end{document}
