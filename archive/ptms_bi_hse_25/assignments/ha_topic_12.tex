\documentclass{article}
% \usepackage[T2A]{fontenc}
\usepackage[utf8]{inputenc}
\usepackage[english, russian]{babel}
\usepackage[top=1cm,bottom=1cm,left=2cm,right=2cm]{geometry}
\usepackage{graphicx}
\usepackage{amsmath}
\usepackage{amsfonts}
\usepackage{xcolor}
\title{ВШБ Бизнес-информатика: ТВиМС 2025. \\ Лист задач для самостоятельного решения \#12. \\ Проверка статистических гипотез.}
\date{}
\author{}

\begin{document}
\maketitle

\section*{Основные формулы}

\subsection*{Распределения статистик}
\begin{itemize}
    \item Выборочное среднее, дисперсия известна:
    $$\bar{X} \sim \mathcal{N}\left(\mu, \frac{\sigma^2}{n}\right)$$
    
    \item Выборочное среднее, дисперсия неизвестна:
    $$\frac{\bar{X} - \mu}{\frac{S}{\sqrt{n}}} \sim t_{n-1}$$
    
    \item Выборочная доля:
    $$\hat{p} \sim \mathcal{N}\left(p, \frac{p(1-p)}{n}\right)$$
    
    \item Разность долей:
    $$D = \left( \hat{p}_1 - \hat{p}_2 \right) \sim \mathcal{N}\left(p_1 - p_2, \frac{p_1(1-p_1)}{n} + \frac{p_2(1-p_2)}{m}\right)$$
    
    \item Разность средних, дисперсии известны:
    $$D = \left( \bar{X} - \bar{Y} \right) \sim \mathcal{N}\left(\mu_X - \mu_Y, \frac{\sigma_X^2}{n} + \frac{\sigma_Y^2}{m}\right)$$

    \item Разность средних, дисперсии неизвестны, но предполагаются равными:
    $$\frac{\bar{X} - \bar{Y} - (\mu_X - \mu_Y)}{S_p \sqrt{\frac{1}{n} + \frac{1}{m}}} \sim t_{n+m-2}$$
    где $S_p^2 = \frac{(n-1)S_X^2 + (m-1)S_Y^2}{n+m-2}$ — объединённая выборочная дисперсия.
    
    \item Разность средних, дисперсии неравны (тест Уэлча):
    \begin{gather*}
    \frac{\bar{X} - \bar{Y} - (\mu_X - \mu_Y)}{\sqrt{\frac{S_X^2}{n} + \frac{S_Y^2}{m}}} \sim t_k \, , \text{ где степень свободы равна } \\
    k \approx \frac{(V_X + V_Y)^2}{\frac{V_X^2}{n-1} + \frac{V_Y^2}{m-1}}, \quad V_X = \frac{S_X^2}{n}, \quad V_Y = \frac{S_Y^2}{m}
    \end{gather*}
\end{itemize}

\clearpage

\subsection*{Формулы для score (статистик)}
\begin{itemize}
    \item Выборочное среднее, дисперсия известна:
    $$z_{\text{score}} = \frac{\bar{x} - \mu_0}{\frac{\sigma}{\sqrt{n}}}$$
    
    \item Выборочное среднее, дисперсия неизвестна:
    $$t_{\text{score}} = \frac{\bar{x} - \mu_0}{\frac{s}{\sqrt{n}}}$$
    
    \item Выборочная доля:
    $$z_{\text{score}} = \frac{\tilde{p} - p_0}{\sqrt{\frac{p_0(1-p_0)}{n}}}$$
    
    \item Разность долей:
    $$z_{\text{score}} = \frac{\tilde{p}_1 - \tilde{p}_2}{\sqrt{p_c(1-p_c)\left(\frac{1}{n} + \frac{1}{m}\right)}},$$ где $p_c = \frac{\tilde{p}_1 n + \tilde{p}_2 m}{n + m}$ — объединённая доля.
    
    \item Разность средних, дисперсии известны:
    $$z_{\text{score}} = \frac{\bar{x} - \bar{y}}{\sqrt{\frac{\sigma_X^2}{n} + \frac{\sigma_Y^2}{m}}}$$
    
    \item Разность средних, дисперсии неизвестны, но предполагаются равными:
    $$t_{\text{score}} = \frac{\bar{x} - \bar{y}}{s_p \sqrt{\frac{1}{n} + \frac{1}{m}}}$$
    
    \item Разность средних, дисперсии неизвестны:
    $$t_{\text{score}} = \frac{\bar{x} - \bar{y}}{\sqrt{\frac{s_X^2}{n} + \frac{s_Y^2}{m}}}$$
\end{itemize}

\clearpage

\begin{enumerate}
    \item Пусть $X$ — случайная величина, показывающая реальное количество кофе в банке "$100$ г кофе", $E[X] = \mu$. При выборке размера $16$ вы хотите проверить нулевую гипотезу $H_0: \mu = 100$ против альтернативы $H_1: \mu > 100$ на уровне значимости $5\%$.
    Пусть $X \sim \mathcal{N}(\mu, \sigma^2)$ и дисперсия известна: $\sigma^2 = 1$. Найдите критическую (отклоняющую) область для этого теста в оригинальной шкале, т.е. при каком выборочном среднем содержания кофе мы начнем отклонять нулевую гипотезу.
    У вас нет фактических экспериментальных данных для выполнения теста, вам нужно только определить границу критической области. 

    % C 3.73
    \item Исследователь проводил односторонний тест, но вместо использования (как должно было быть) верхней $5\%$ критической точки стандартного нормального распределения,
    он использовал верхнюю $5\%$ точку распределения Стьюдента с $6$ степенями свободы.
    Каков истинный уровень значимости этого теста?

    \item Случайная выборка из десяти студентов показала следующие значения времени (в часах), потраченного на изучение в неделю перед финальными экзаменами:

    $$
    28 \quad 57 \quad 42 \quad 35 \quad 61 \quad 39 \quad 55 \quad 46 \quad 49 \quad 38.
    $$


   Предположим, что распределение генеральной совокупности нормальное. 
    \begin{enumerate}
        \item Найдите выборочное среднее и выборочное стандартное отклонение.
        \item Проверьте гипотезу о том, что среднее генеральной совокупности равно $40$, против альтернативы, что оно больше. 
    \end{enumerate}

    \textit{Можно пользоваться следующим разбиением на подзадачи, чтобы лучше разобраться в теме:}
    \begin{enumerate}
        \item Опишите распределение статистики $\bar{X}$ в этой задаче. Известна ли вам дисперсия генеральной совокупности здесь?
        \item Сформулируйте нулевую и альтернативную гипотезы.
        \item Опишите все случайные величины, необходимые для процедуры проверки, их свойства и распределения при нулевой гипотезе, когда мы уверены, что $H_0$ полностью верна.
        \item Найдите границу критической области в оригинальной шкале и выполните проверку, используя $\alpha = 1 \%, \; 5 \%, \; 10 \%$.
        \item Выполните проверку с помощью правильного \textit{score}, \textit{т.е.} путём преобразования $\bar{x}$ в шкалу выбранного распределения.
        \item Покажите, что результаты проверки в двух подходах идентичны.
    \end{enumerate}

\item Если вы живёте в Калифорнии, решение о покупке страховки от землетрясений является критически важным.
Статья в научном журнале от июня 1992 года исследовала множество факторов, которые жители Калифорнии учитывают при покупке страховки от землетрясений.
Опрос показал, что только $133$ из $337$ случайно выбранных домохозяйств в округе Лос-Анджелес были защищены страховкой от землетрясений.
    \begin{enumerate}
        \item Каковы подходящие нулевая и альтернативная гипотезы для проверки утверждения, что менее $40\%$ жителей округа Лос-Анджелес были защищены страховкой от землетрясений?
        \item Предоставляют ли данные достаточные доказательства в поддержку нулевой гипотезы? (Используйте $\alpha = 0.10$)
    \end{enumerate}


\item Американская ассоциация больниц сообщает в Hospital Statistics, что средняя стоимость для общих общественных больниц на одного пациента в день в больницах США составляла $\$951$ в $1998$ году.
В том же году случайная выборка из $30$ дневных затрат в больницах Нью-Йорка дала среднее значение $\$1185$.
Предполагая стандартное отклонение генеральной совокупности $\$333$ для больниц Нью-Йорка, предоставляют ли данные достаточные доказательства для заключения, что в $1998$ году средняя стоимость в больницах Нью-Йорка превышала национальное среднее $\$951$?
Выполните требуемую проверку гипотез на уровне значимости $5\%$.

    \newpage

% C 3.76    
\item Во время ночной смены в пятницу из производственной линии случайным образом было отобрано $n=28$ мятных конфет и взвешено.
Они имели средний вес $\bar{x}=21.45$ граммов. Известно, что стандартное отклонение веса конфеты равно $\sigma = 0.31$ грамма.
\begin{enumerate}
    \item Проверьте нулевую гипотезу $\mu=20$ против альтернативы $\mu>20$ на уровне значимости $5\%$.
    \item Какое в данном случае получилось $P$-значение?
    \item Предположим, что вдруг \textit{на самом деле} $\mu = 22$ (то есть верна альтернативная гипотеза).
    Также пусть $K$ - граница критической области в оригинальной шкале из предыдущего пункта.
    Напишите, как будут выглядеть необходымые распределения статистик в таком случае (то есть распределения при $\mu = 22$).
    Найдите вероятность:

    $$
    P_{H_1}(\bar{X} < K) = \beta
    $$

    Это вероятность того, что мы не отклоним нулевую гипотезу, когда она на самом деле неверна.
    Это и есть вероятность ошибки II рода.
    Говоря более подробно, это вероятность такого случая, когда при верной альтернативной гипотезе мы волей случая получаем неубедительные данные,
    и, как следствие, не отклоняем нулевую гипотезу.
\end{enumerate}

% Problem C3.107.
\item  Были проведены два опроса в Москве и Твери. Из выборки $200$ человек в Москве $125$ были против курения в ресторанах.
В Твери $52$ из выборки $100$ были против курения в ресторанах.
Пусть $p_1$ и $p_2$ — доли генеральных совокупностей людей, которые против курения в Твери и Москве соответственно.
\begin{enumerate}
    \item Постройте $95\%$ доверительный интервал для разности долей $p_1 - p_2$.
    \item На уровне значимости $5\%$ проверьте нулевую гипотезу $H_0: p_1 = p_2$ против $ H_1: p_2 > p_1$.
    \item На уровне значимости $2.5\%$ проверьте нулевую гипотезу $H_0: p_2 = 0.55$ против $H_1: p_2 > 0.55$. 
\end{enumerate}

\item Супермаркет провёл исследование, чтобы выяснить, одинаковы ли средние недельные продажи шоколадных батончиков при использовании обычного расположения на полке и при использовании витрины в конце прохода (дисплейная выкладка).
Сводка данных представлена в таблице:
    
    \begin{center}
         \begin{tabular}{c | c | c | c} 
         \hline
          & Размер выборки & Выборочное среднее & Выборочная дисперсия \\ 
         \hline
         Обычное расположение на полке & $11$ & $5.3$ & $2.4$ \\ 
         \hline
         Витрина в конце прохода & $10$ & $7.2$ & $2.8$ \\
        \end{tabular}
    \end{center}

    Предполагая, что недельные продажи распределены нормально, аналитический отдел хочет определить,
    действительно ли средние недельные продажи при использовании витрины в конце прохода выше, чем при обычном расположении товаров на полке.
    
    \textit{Можно пользоваться следующим разбиением на подзадачи, чтобы лучше разобраться в теме:}
    \begin{enumerate}
        \item Пусть первая выборка — это $\{ X_1, \ldots, X_{11} \}$, а вторая — $\{Y_{1}, \ldots, Y_{10} \}$. Опишите распределения статистик $\bar{X}$, $\bar{Y}$ и $D = \bar{X} - \bar{Y}$ в этой задаче.
    \item Сформулируйте нулевую и альтернативную гипотезы.
    \item Опишите все случайные величины, необходимые для процесса тестирования, их свойства и распределения при нулевой гипотезе, когда мы уверены, что $H_0$ полностью верна.
    \item Выполните проверку с помощью правильного \textit{score} теста, \textit{т.е.} путём преобразования $\bar{x} - \bar{y}$ в шкалу выбранного распределения. Используйте уровни значимости $\alpha = 1\%, \; 5\%, \; 10\%$.
    \item \textit{В задачах на разность параметров это может быть скорее в качестве доп. пункта для проверки себя. Быстрее решать конечно же с помощью score.}
    
    Найдите границу критической области в оригинальной шкале и выполните проверку, используя предыдущие уровни значимости. Покажите, что результаты проверки в двух подходах идентичны.
    \end{enumerate}

    \newpage

    \item Данные в следующей таблице показывают количество ежедневных нарушений парковки в двух районах города. Идентификация дней неизвестна, и записи не обязательно были сделаны в одни и те же дни.
    Есть ли доказательства того, что районы имеют разные средние количества нарушений?
    Укажите необходимые предположения и выполните проверку гипотез на уровнях значимости $\alpha = 1\%, \; 5\%, \; 10\%$.
    \begin{center}
    \begin{tabular}{ |c|c| } 
    \hline
    Район A & Район B \\
     \hline
    38 & 32 \\
    38 & 38 \\ 
    29 & 22 \\
    45 & 30 \\
    42 & 34 \\
    33 & 28 \\
    27 & 32 \\
    32 & 34 \\
    32 & 24 \\
    34 & нет данных \\
     \hline
    \end{tabular}
    \end{center}

% C 3.46
\item \textit{Насколько помогают ремни безопасности?} Чтобы ответить на этот вопрос, было проведено исследование автомобилей, которые были оборудованы ремнями безопасности (поясные и плечевые ремни) и впоследствии попали в аварии. Случайная выборка из $10,000$ пассажиров показала следующие показатели травматизма (восстановлено из U.S. \textit{Department of Transportation}, 1981):

    \begin{center}
    \begin{tabular}{ |c|c|c|c| } 
    & \multicolumn{2}{|c|}{Ремень безопасности надет} & \\
    \hline
    Тяжёлая или смертельная травма & Да & Нет & Всего \\
     \hline
    Да & 3 & 119 & 122 \\
    \hline
    Нет & 829 & 9049 & 9878 \\ 
    \hline
    Всего & 832 & 9168 & 10000 \\
     \hline
    \end{tabular}
    \end{center}

\begin{enumerate}
    \item Как бы вы проверили положительный эффект использования ремней безопасности с помощью теста на разность долей? Сформулируйте $H_0$ словами и формализованно. 
    \item Выполните проверку гипотез при различных уровнях значимости $\alpha = 0.1, 0.05, 0.01$. Какое получилось $P$-значение в вашем выбранном тесте? Прокомментируйте результаты.
    \item Постройте соответствующий доверительный интервал. Исследуйте поведение границ при различных уровнях доверия: $90\%$, $95\%$, $99\%$. Какие выводы можно сделать?
\end{enumerate}
\end{enumerate}

\end{document}
