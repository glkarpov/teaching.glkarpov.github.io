\documentclass{article}
% \usepackage[T2A]{fontenc}
\usepackage[utf8]{inputenc}
\usepackage[english, russian]{babel}
\usepackage[top=1cm,bottom=1cm,left=2cm,right=2cm]{geometry}
\usepackage{graphicx}
\usepackage{amsmath}
\usepackage{amsfonts}
\usepackage{xcolor}
\title{ВШБ Бизнес-информатика: ТВиМС 2025. \\ Лист задач для самостоятельного решения \#3a. \\ Условная вероятность. Независимость событий. \\ (Полная вероятность. Теорема Байеса)}
\date{}
\author{}

% Размер промежутка между задачами для вписывания решений
\newcommand{\problemspace}{6cm}

\begin{document}
\maketitle

\begin{enumerate}
    \item Компания А обанкротится с вероятностью $0.4$, компания В обанкротится с вероятность $0.2$ независимо от первой.
    Найти вероятности следующих событий:
    \begin{enumerate}
        \item обанкротится ровно одна компания.
        \item обанкротится хотя бы одна компания.
        \item обанкротятся обе компании.
        \item обанкротилась А, если известно, что обанкротилась ровна одна компания.
        \item обанкротилась А, если известно, что обанкротилась как минимум одна компания.
    \end{enumerate}
    \textit{Укажите все места, где использовалась независимость, все места, где использовалась несовместность.}
    
    \vspace{4cm}
    
    \item Брат и сестра путешествуют на поезде. У них обоих нет билетов, и контролёр поймал их.
    Он уполномочен применить особое наказание за это правонарушение.
    У него есть коробка с девятью внешне одинаковыми шоколадными конфетами, три из которых содержат снотворное моментального действия, такое, что человек проспит месяц.
    Контролёр заставляет каждого из нарушителей по очереди выбрать и немедленно съесть одну конфету.

    \begin{enumerate}
        \item Если брат выбирает перед сестрой, какова вероятность того, что он уснет?
         \item Если брат выбирает первым и не впадает в кому, какова вероятность того, что сестра останется бодрствующей?
         \item Если брат выбирает первым и впадает в кому, какова вероятность того, что сестра останется бодрствующей?
         \item В интересах ли сестры убедить брата выбирать первым? \textit{I.e.} есть ли разница в вероятности засыпания в зависимости от того, кто выбирает первым?
    \end{enumerate}
    
    $A$ – событие, соответствующее впасть в кому, выбирая первым. $B$ – событие, соответствующее впасть в кому, выбирая вторым.
    \begin{enumerate}
        \item $P(A) = \frac{3}{9} = \frac{1}{3}$, следовательно $P(\bar{A}) = \frac{2}{3}$
        \item $P(\bar{B} | \bar{A}) = \frac{5}{8}$
        \item $P(\bar{B} | A) = \frac{6}{8}$
        \item $$P(\bar{B}) = P(\bar{B} | \bar{A})P(\bar{A}) + P(\bar{B} | A)P(A) = \frac{5}{8} \cdot \frac{2}{3} + \frac{6}{8} \cdot \frac{1}{3} = \frac{10}{24} + \frac{6}{24} = \frac{2}{3}$$
        Получаем, что $P(\bar{B}) = P(\bar{A})$, что означает, что нет разницы в стратегиях выбора.
    \end{enumerate}
        
    \item Подбросили две игральные кости. Введем следующие события: $A$ — на первой кости выпала тройка, $B$ — сумма очков является четным числом,
    $C$ — на второй кости выпало больше очков, чем на первой.
    \begin{enumerate}
    \item Найдите вероятность каждого из событий $A$, $B$ и $C$. 
    \item Найдите условную вероятность $P(A\,|\,C)$. 
    \item Проверьте, есть ли среди событий $A$, $C$ и $B \cap C$ пары независимых событий.
    \end{enumerate}

\vspace{\problemspace}

    \item Три студента – А, В и С независимо друг от друга опаздывают на занятия с соответствующими вероятностями $0.6$, $0.3$, и $0.8$.
    \begin{enumerate}
        \item Найти вероятность того, что никто из них не опоздает.
        \item Найти вероятность того, что опоздают ровно двое из них. 
        \item Найти вероятность того, что опоздает хотя бы один из них.
        \item Найти вероятность того, что А опоздал, если известно, что опоздал как минимум один из них.
        \item Известно, что А опоздал, найти вероятность того, что опоздали как минимум двое из них.
    \end{enumerate}
    \textit{Укажите все места, где использовалась независимость, все места, где использовалась несовместность.}
    
    \newpage
    
    \item Студент сдает сессию. Рассмотрим два события: $A$ – студент готовится к экзамену по теорверу, $B$ – студент сдает экзамен по теорверу на отлично.
    Известны вероятности следующих событий: $P(A) = 0.5$, $P(B)=0.2$, $P(A \cap B)=0.18$.

    Найдите вероятности, дайте словесные формулировки событий: $P(A \cup B)$, $P(A \, | \, B)$, $P(A|\bar{B})$, $P(\bar{A}|B)$, $P(\bar{B} \, | \, A)$, $P(B \, | \, \bar{A})$.

    \textit{Например, по последней вероятности словесная формулировка: найти вероятность того, что студент сдаст экзамен, если известно, что он не готовился. }
    
    \vspace{\problemspace}

    \item Событие $A$ заключается в наступлении кризиса, вероятность этого $0.4$,
    событие $B$ заключается в том, что компания обанкротится, вероятность этого $0.5$.
    Вероятность того, что произойдут оба события одновременно, равна $0.3$.
    
    Найти вероятности, дать словесные формулировки: 
    $P(A|B)$, $P(\bar{A}|\bar{B})$, $P(B|\bar{A})$,    $P(A|(\bar{A}B \, \cup \, A\bar{B}))$    

    Указание: в последнем пункте условием является все событие $(\bar{A}B \, \cup \, A\bar{B})$ – если записывать словами, то можно сформулировать так:
    "известно, что произошло ровно одно из этих двух событий".

\newpage

    % source: Athena Scientific, Chapter 1, #31
    \item Wi-Fi роутер передаёт мемы через зашумлённый канал связи.
    Телефон принимает символы \textbf{неправильно} с вероятностью $\varepsilon_0$ и $\varepsilon_1$ соответственно.
    Ошибки в различных передачах символов \textbf{независимы}.

    \begin{center}
    \includegraphics[width=7cm]{../files/channel.pdf}
    \end{center}

    \begin{enumerate}
        \item  Роутер передает заранее ему известную (не случайную) строку символов. Какова вероятность того, что:
        \begin{enumerate}
            \item переданная строка символов $1011$ будет принята правильно?
            $$P(1011) = P(1)P(0)P(1)P(1) = (1 - \varepsilon_1)(1 - \varepsilon_0)(1 - \varepsilon_1)(1 - \varepsilon_1) = (1 - \varepsilon_1)^3 (1 - \varepsilon_0)$$
            \item переданная строка символов $0001$ будет принята правильно?
            $$P(0001) = P(0)P(0)P(0)P(1) = (1 - \varepsilon_0)^3 (1 - \varepsilon_1)$$
        \end{enumerate}
        \item В попытке улучшить надёжность, каждый символ передаётся три раза, и
принятая строка декодируется по правилу большинства. Другими словами, например, $0$ передаётся как $000$, и декодируется на приёмнике как 0 \textbf{тогда и только тогда}, когда принятая трёхсимвольная строка содержит не менее двух 0.
Например, $101$ декодируется как $1$, а $001$ декодируется как $0$. Какова вероятность того, что:
        \begin{enumerate}
            \item переданный символ $1$ будет правильно декодирован?

            Передаем цепочку $111$, а вот получить уже можем разное. Символ декодируется правильно, если в принятой цепочке не менее двух единиц.
            $$
            P(1) = P\{101, 111, 110, 011\} = (1 - \varepsilon_1) \varepsilon_1 (1 - \varepsilon_1) + (1 - \varepsilon_1)^3 + (1 - \varepsilon_1)^2 \varepsilon_1 +  \varepsilon_1 (1 - \varepsilon_1)^2 = 3(1 - \varepsilon_1)^2 \varepsilon_1 + (1 - \varepsilon_1)^3
            $$
            \item переданный символ $0$ будет правильно декодирован?

            Аналогично, передаем цепочку $000$. Символ декодируется правильно, если в принятой цепочке не менее двух нулей.
            $$
            P(0) = P\{000, 010, 001, 100\} = (1 - \varepsilon_0)^3 + 3(1 - \varepsilon_0)^2 \varepsilon_0
            $$

            \item Для каких значений $\varepsilon_0$ в новой схеме будет наблюдаться улучшение в вероятности правильного приёма символа $0$ по сравнению с базовым вариантом?
            \begin{gather*}
                1 - \varepsilon_0 < (1 - \varepsilon_0)^3 + 3(1 - \varepsilon_0)^2 \varepsilon_0 \\ 
                1 - \varepsilon_0 < 1 - 3 \varepsilon_0 + 3 \varepsilon_0^2 - \varepsilon_0^3 + 3 \varepsilon_0 - 6 \varepsilon_0^2 + 3 \varepsilon_0^3 \\
                2\varepsilon_0^3 - 3 \varepsilon_0^2 + \varepsilon_0 > 0 \\
                \varepsilon_0(\varepsilon_0 - 1)(2\varepsilon_0 - 1) > 0 \\
                0 < \varepsilon_0 < \frac{1}{2}
            \end{gather*}
        \end{enumerate}
    \end{enumerate}

    \vspace{\problemspace}

    \item Для сдачи зачета студенту предлагают последовательно решить три задачи. Зачет он получит в том случае, если у него будут две \textbf{подряд} решенные задачи.
Сами задачи бывают "Сложные" и "Простые" (отличаются вероятностями их решить, для первого типа $p_1$, для второго $p_2$, $p_1 < p_2$).
Преподаватель чередует задачи по сложности (то есть существует только два варианта выдачи: СПС и ПСП).
Что выгоднее для студента: чтобы среди трех выданных задач было две сложные или две простые задачи? 

\end{enumerate}

\end{document}
