\documentclass[landscape]{article}
\usepackage[utf8]{inputenc}
\usepackage[russian]{babel}
\usepackage[margin=0.5cm, paperwidth=29.7cm, paperheight=21cm]{geometry}
\usepackage{tikz}
\usepackage{amsmath}
\usepackage{amssymb}

% Убираем все отступы и границы
\setlength{\parindent}{0pt}
\setlength{\parskip}{0pt}
\setlength{\topskip}{0pt}
\setlength{\headsep}{0pt}
\setlength{\footskip}{0pt}

% Увеличиваем размер шрифта
\fontsize{14pt}{17pt}\selectfont

\begin{document}
\pagestyle{empty}
\noindent
\vspace{0.8cm}

% Разделение на 2 горизонтальные области
\begin{tikzpicture}[remember picture, overlay]
    % Вертикальная линия посередине
    \draw[very thin, gray] ([xshift=14.85cm]current page.north west) -- ([xshift=14.85cm]current page.south west);
\end{tikzpicture}

% Верхняя горизонтальная область: Варианты 1 и 2
\noindent
\begin{minipage}[t][18.5cm][t]{14.35cm}
\textbf{Вариант 1: ($-\infty$, Гу]}

Случайная величина $X$ имеет функцию плотности, заданную формулой
$$
f_X(x)= c x^2(x+1)
$$
определенную на области $0<x<1$.
\begin{enumerate}
    \item Найдите нормировочную константу $c$, при котором это является функцией плотности.
    \item Постройте функцию распределения случайной величины $X$.
    \item Вычислите вероятности $P(-4 < X < 0.5)$, $P(0.5 < X < 2 \mid X>0.25)$.
    \item Вычислите $E[X]$ и $Var[X]$.
\end{enumerate}

\vspace{0.5cm}

\textbf{Вариант 3: (Ли, Се]}

Случайная величина $X$ имеет функцию плотности, заданную формулой
$$
f_X(x) = k(3x+1)(x+1)
$$
определенную на области $0<x<1$.
\begin{enumerate}
    \item Найдите нормировочную константу $k$, при котором это является функцией плотности.
    \item Постройте функцию распределения случайной величины $X$.
    \item Вычислите вероятности $P(-10<X<0.5)$, $P(X > 0.5 \mid X > -100)$.
    \item Вычислите $E[X]$ и $Var[X]$.
\end{enumerate}
\end{minipage}
\hfill
\begin{minipage}[t][18.5cm][t]{14.35cm}
\textbf{Вариант 2: (Гу, Ли]}

Случайная величина $X$ имеет функцию плотности, заданную формулой
$$
f_X(x) = c x(x+1)
$$
определенную на области $0<x<1$.
\begin{enumerate}
    \item Найдите нормировочную константу $c$, при котором это является функцией плотности.
    \item Постройте функцию распределения случайной величины $X$.
    \item Вычислите вероятности $P(X>0.25)$, $P(X < 0.25 \mid X < 0.5)$.
    \item Вычислите $E[X]$ и $Var[X]$.
\end{enumerate}

\vspace{0.5cm}

\textbf{Вариант 4: (Се, $+\infty$]}

Случайная величина $X$ имеет функцию плотности, заданную формулой
$$
f_X(x) = kx^2(1-x)
$$
определенную на области $0<x<1$.
\begin{enumerate}
    \item Найдите нормировочную константу $k$, при котором это является функцией плотности.
    \item Постройте функцию распределения случайной величины $X$.
    \item Вычислите вероятности $P(0.25<X<0.75)$, $P(-2<X<0.5)$.
    \item Вычислите $E[X]$ и $Var[X]$.
\end{enumerate}
\end{minipage}

\newpage

% Новая страница в том же стиле
\noindent
\vspace{0.8cm}

% Разделение на 2 горизонтальные области
\begin{tikzpicture}[remember picture, overlay]
    % Вертикальная линия посередине
    \draw[very thin, gray] ([xshift=14.85cm]current page.north west) -- ([xshift=14.85cm]current page.south west);
\end{tikzpicture}

% Две вертикальные колонки: Варианты 1 и 3 слева, Варианты 2 и 4 справа
\noindent
\begin{minipage}[t][18.5cm][t]{14.35cm}

% src: UoL, st104b, 2015, zone A, question 3.
Случайная величина $X$ имеет функцию плотности, заданную формулой:

$$
    f_X(x) = \beta x + \gamma x^2,
$$

определенную на области $0 < x < 1$, и равную $0$ вне этой области. Известно значение математического ожидания $E[X]$.

\begin{enumerate}
    \item Найдите корректные значения параметров $\beta$ и $\gamma$, при которых это является функцией плотности.
    \item Вычислите $E[\frac{1}{X}]$.
    \item Постройте функцию распределения случайной величины $X$.
    \item Вычислите вероятности $p_1$, и $p_2$.
\end{enumerate}

\begin{itemize}
    \item 1 вариант: имя на \textbf{A}, $E[X] = \frac{25}{36}$, $p_1 = P \{1/4 < X < 3/4  \}$,
    
    $p_2 = P \{3/4 < X < 4/3 | X > 1/2 \}$
    \item 2 вариант: имя на \textbf{Б-З}, $E[X] = \frac{17}{24}$, $p_1 = P \{1/2 < X < 3/2  \}$,
    
    $p_2 = P \{1/4 < X < 3/4 | X < 1/2 \}$
\end{itemize}

\end{minipage}
\hfill
\begin{minipage}[t][18.5cm][t]{14.35cm}


    % src: UoL, st104b, 2005, zone B, section B.
    Случайная величина $X$ имеет функцию плотности, заданную формулой:
    
    $$ f_X(x) = a + bx^2, $$
    
    определенную на области $0 < x < 2$, и равную $0$ вне этой области.
    Известно значение математического ожидания $E[X]$.
    \begin{enumerate}
        \item Найдите корректные значения параметров $a$ и $b$, 
        \item Постройте функцию распределения случайной величины $X$.
        \item Вычислите $Var[X]$.
        \item Вычислите вероятности $p_1$, и $p_2$.
    \end{enumerate}
    
    \begin{itemize}
        \item 3 вариант: имя на \textbf{И-М}, $E[X] = \frac{3}{2}$, $p_1 = P \{ X < 1/2  \}$,
        
        $p_2 = P \{1 < X < 5/2 | X > 1/2 \}$
        \item 4 вариант: имя на \textbf{Н-Я}, $E[X] = \frac{7}{6}$, $p_1 = P \{ X > 1\}$,
        
        $p_2 = P \{1/2 < X < 2 | X < 1 \}$
    \end{itemize}

\end{minipage}

\end{document}

