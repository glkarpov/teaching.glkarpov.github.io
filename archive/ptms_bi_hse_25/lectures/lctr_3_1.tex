\documentclass[landscape]{article}
\usepackage[utf8]{inputenc}
\usepackage[russian]{babel}
\usepackage[margin=0.5cm, paperwidth=29.7cm, paperheight=21cm]{geometry}
\usepackage{tikz}
\usepackage{amsmath}
\usepackage{amssymb}
\usepackage{relsize}

% Убираем все отступы и границы
\setlength{\parindent}{0pt}
\setlength{\parskip}{0pt}
\setlength{\topskip}{0pt}
\setlength{\headsep}{0pt}
\setlength{\footskip}{0pt}

% Увеличиваем размер шрифта
\fontsize{14pt}{17pt}\selectfont

% Увеличиваем размер математических формул
\DeclareMathSizes{14}{20}{16}{14}

\begin{document}
\pagestyle{empty}
\noindent
\vspace{0.8cm}

% Разделение на 2 горизонтальные области
\begin{tikzpicture}[remember picture, overlay]
    % Вертикальная линия посередине
    \draw[very thin, gray] ([xshift=14.85cm]current page.north west) -- ([xshift=14.85cm]current page.south west);
\end{tikzpicture}

% Две вертикальные колонки: Варианты 1 и 3 слева, Варианты 2 и 4 справа
\noindent
\begin{minipage}[t][18.5cm][t]{14.35cm}

    \textbf{1 вариант: Отчество на А}

    \begingroup\Large
    Предположим, у нас есть две независимые случайные выборки: $\{X_1 , \ldots , X_4\}$, каждая $X_i \sim \mathcal{N}(1, 4)$, и $\{Y_1 , \ldots , Y_5\}$, каждая $Y_i \sim \mathcal{N}(2, 10)$.

    Пусть $$W = \frac{\bar{X} + \bar{Y}}{2} + X_1$$

    Найдите вероятность $P(W < 2)$.
    \endgroup

    \vspace{3.5cm}

    \textbf{2 вариант: Отчество от Б до Е:}

    \begingroup\Large
    Предположим, у нас есть две независимые случайные выборки: $\{X_1 , \ldots , X_{12}\}$, каждая $X_i \sim \mathcal{N}(2, \sigma^2)$, и $\{Y_1 , \ldots , Y_3\}$, каждая $Y_i \sim \mathcal{N}(3, 2 \sigma^2)$.

    Найдите вероятность $P(W < -1 + \frac{\sigma}{2})$, где $W = \bar{X} - \bar{Y}$.
    \endgroup

\end{minipage}
\hfill
\begin{minipage}[t][18.5cm][t]{14.35cm}
    
    \textbf{3 вариант: Отчество от Ж до Н:}

    \begingroup\Large
    Предположим, у нас есть две независимые случайные выборки: $\{X_1 , \ldots , X_4\}$, каждая $X_i \sim \mathcal{N}(1, 4)$, и $\{Y_1 , \ldots , Y_5\}$, каждая $Y_i \sim \mathcal{N}(2, 10)$.

    Найдите точку $k$, такую что $P(\bar{X} + \bar{Y} > k) = 0.8$.
    \endgroup

    \vspace{3.5cm}

    \textbf{4 вариант: Отчество от О до Я:}

    \begingroup\Large
    Предположим, у нас есть две независимые случайные выборки: $\{X_1 , \ldots , X_{12}\}$, каждая $X_i \sim \mathcal{N}(2, \sigma^2)$, и $\{Y_1 , \ldots , Y_3\}$, каждая $Y_i \sim \mathcal{N}(3, 2 \sigma^2)$.

    Найдите вероятность $P(W < 1 + \sigma)$, где $W = \bar{Y} - \bar{X}$.
    \endgroup
\end{minipage}

\end{document}
