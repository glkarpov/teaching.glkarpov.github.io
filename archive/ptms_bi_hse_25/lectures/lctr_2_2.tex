\documentclass[landscape]{article}
\usepackage[utf8]{inputenc}
\usepackage[russian]{babel}
\usepackage[margin=0.5cm, paperwidth=29.7cm, paperheight=21cm]{geometry}
\usepackage{tikz}
\usepackage{amsmath}
\usepackage{amssymb}

% Убираем все отступы и границы
\setlength{\parindent}{0pt}
\setlength{\parskip}{0pt}
\setlength{\topskip}{0pt}
\setlength{\headsep}{0pt}
\setlength{\footskip}{0pt}

% Увеличиваем размер шрифта
\fontsize{14pt}{17pt}\selectfont

\begin{document}
\pagestyle{empty}
\noindent
\vspace{0.8cm}

% Разделение на 2 горизонтальные области
\begin{tikzpicture}[remember picture, overlay]
    % Вертикальная линия посередине
    \draw[very thin, gray] ([xshift=14.85cm]current page.north west) -- ([xshift=14.85cm]current page.south west);
\end{tikzpicture}

% Верхняя горизонтальная область: Варианты 1 и 2
\noindent
\begin{minipage}[t][18.5cm][t]{14.35cm}
\textbf{Вариант 1: ($-\infty$, Гу]}

Случайная величина $X$ имеет функцию плотности, заданную формулой
$$
f_X(x)= c x^2(x+1)
$$
определенную на области $0<x<1$.
\begin{enumerate}
    \item Найдите нормировочную константу $c$, при котором это является функцией плотности.
    \item Постройте функцию распределения случайной величины $X$.
    \item Вычислите вероятности $P(-4 < X < 0.5)$, $P(0.5 < X < 2 \mid X>0.25)$.
    \item Вычислите $E[X]$ и $Var[X]$.
\end{enumerate}

\vspace{0.5cm}

\textbf{Вариант 3: (Ли, Се]}

Случайная величина $X$ имеет функцию плотности, заданную формулой
$$
f_X(x) = k(3x+1)(x+1)
$$
определенную на области $0<x<1$.
\begin{enumerate}
    \item Найдите нормировочную константу $k$, при котором это является функцией плотности.
    \item Постройте функцию распределения случайной величины $X$.
    \item Вычислите вероятности $P(-10<X<0.5)$, $P(X > 0.5 \mid X > -100)$.
    \item Вычислите $E[X]$ и $Var[X]$.
\end{enumerate}
\end{minipage}
\hfill
\begin{minipage}[t][18.5cm][t]{14.35cm}
\textbf{Вариант 2: (Гу, Ли]}

Случайная величина $X$ имеет функцию плотности, заданную формулой
$$
f_X(x) = c x(x+1)
$$
определенную на области $0<x<1$.
\begin{enumerate}
    \item Найдите нормировочную константу $c$, при котором это является функцией плотности.
    \item Постройте функцию распределения случайной величины $X$.
    \item Вычислите вероятности $P(X>0.25)$, $P(X < 0.25 \mid X < 0.5)$.
    \item Вычислите $E[X]$ и $Var[X]$.
\end{enumerate}

\vspace{0.5cm}

\textbf{Вариант 4: (Се, $+\infty$]}

Случайная величина $X$ имеет функцию плотности, заданную формулой
$$
f_X(x) = kx^2(1-x)
$$
определенную на области $0<x<1$.
\begin{enumerate}
    \item Найдите нормировочную константу $k$, при котором это является функцией плотности.
    \item Постройте функцию распределения случайной величины $X$.
    \item Вычислите вероятности $P(0.25<X<0.75)$, $P(-2<X<0.5)$.
    \item Вычислите $E[X]$ и $Var[X]$.
\end{enumerate}
\end{minipage}



\newpage

% Новая страница в том же стиле
\noindent
\vspace{0.8cm}

% Разделение на 2 горизонтальные области
\begin{tikzpicture}[remember picture, overlay]
    % Вертикальная линия посередине
    \draw[very thin, gray] ([xshift=14.85cm]current page.north west) -- ([xshift=14.85cm]current page.south west);
\end{tikzpicture}

% Две вертикальные колонки: Варианты 1 и 3 слева, Варианты 2 и 4 справа
\noindent
\begin{minipage}[t][18.5cm][t]{14.35cm}

    1 вариант: Отчество на \textbf{A}:

    Закон распределения пары случайных величин $X$ и $Y$ и задан таблицей

    \begin{tabular}{lccc}
    \hline & $X=-1$ & $X=0$ & $X=2$ \\
    \hline$Y=1$ & 0.2 & 0.1 & 0.2 \\
    $Y=2$ & 0.1 & 0.2 & 0.2 \\
    \hline
    \end{tabular}
    
    Найдите $\mathbb{E}(X), \mathbb{E}(Y), \operatorname{Var}(X), \operatorname{Cov}(X, Y), \operatorname{Cov}(2 X+3,1-3 Y)$

    \vspace{1.5cm}

    2 вариант: Отчество на \textbf{Б-И}:
    Совместное распределение доходов акций двух компаний $Y$ и $X$ задано в виде таблицы

    \begin{tabular}{cccc}
    \hline & $X=-1$ & $X=0$ & $X=1$ \\
    \hline$Y=-1$ & 0.1 & 0.2 & 0.2 \\
    $Y=1$ & 0.2 & 0.1 & 0.2 \\
    \hline
    \end{tabular}
    
    Найдите:
    a) Частные распределения случайных величин $X$ и $Y$

    б) $\operatorname{Cov}(X, Y)$

    в) Можно ли утверждать, что случайные величины $X$ и $Y$ зависимы?

    г) У инвестора портфель, в котором доля акций $X$ составляет $\alpha$, а доля акций $Y-(1-\alpha)$. Каковы должны быть доли, чтобы риск портфеля (дисперсия дохода) был бы минимальным?

    д) Условное распределение случайной величины $X$ при условии $Y=-1$.

\end{minipage}
\hfill
\begin{minipage}[t][18.5cm][t]{14.35cm}

    3 вариант: Отчество на \textbf{К-С}:
Задана таблица совместного распределения случайных величин $X$ и $Y$.

\begin{tabular}{cccc}
\hline & $Y=-1$ & $Y=0$ & $Y=1$ \\
\hline$X=0$ & 0.2 & 0.1 & 0.3 \\
$X=1$ & 0.2 & 0.1 & 0.1 \\
\hline
\end{tabular}


Найдите $\mathbb{E}(X), \mathbb{E}\left(X^2\right), \mathbb{E}(Y), \mathbb{E}\left(Y^2\right)$;

Найдите $\operatorname{Var}(X), \operatorname{Var}(Y)$;

Найдите $\operatorname{Cov}(X, Y), \operatorname{Corr}(X, Y)$

\vspace{1.5cm}

4 вариант: Отчество на \textbf{Т-Я}:

Совместный закон распределения случайных величин $X$ и $Y$ задан таблицей:

\begin{tabular}{cccc}
\hline & $Y=-1$ & $Y=0$ & $Y=2$ \\
\hline$X=0$ & 0.2 & $c$ & 0.2 \\
$X=1$ & 0.1 & 0.1 & 0.1 \\
\hline
\end{tabular}

Найдите $c, \mathbb{P}(Y>-X), \mathbb{E}\left(X \cdot Y^2\right)$.

Найдите $\operatorname{Var}(X), \operatorname{Var}(Y)$;

Найдите $Cov(X, Y), Corr(X, Y)$
\end{minipage}

\end{document}
