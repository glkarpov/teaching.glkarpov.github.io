\documentclass[landscape]{article}
\usepackage[utf8]{inputenc}
\usepackage[russian]{babel}
\usepackage[margin=0.5cm, paperwidth=29.7cm, paperheight=21cm]{geometry}
\usepackage{tikz}
\usepackage{amsmath}
\usepackage{amssymb}

% Убираем все отступы и границы
\setlength{\parindent}{0pt}
\setlength{\parskip}{0pt}
\setlength{\topskip}{0pt}
\setlength{\headsep}{0pt}
\setlength{\footskip}{0pt}

\begin{document}
\pagestyle{empty}
\noindent
\vspace{0.3cm}

% Разделение на 4 равные области
\begin{tikzpicture}[remember picture, overlay]
    % Вертикальная линия посередине
    \draw[very thin, gray] (current page.north) ++(0,-10.3cm) -- (current page.south) ++(0,10.3cm);
    % Горизонтальная линия посередине
    \draw[very thin, gray] (current page.west) ++(14.85cm,0) -- (current page.east) ++(-14.85cm,0);
\end{tikzpicture}

% Верхняя часть: два квадранта рядом
\noindent
\begin{minipage}[t][9.5cm][t]{14.35cm}
\textbf{Вариант 1: ($-\infty$, Гу]}

Случайная величина $X$ имеет функцию плотности, заданную формулой
$$
f_X(x)= c x^2(x+1)
$$
определенную на области $0<x<1$.
\begin{enumerate}
    \item Найдите нормировочную константу $c$, при котором это является функцией плотности.
    \item Постройте функцию распределения случайной величины $X$.
    \item Вычислите вероятности $P(-4 < X < 0.5)$, $P(0.5 < X < 2 \mid X>0.25)$.
    \item Вычислите $E[X]$ и $Var[X]$.
\end{enumerate}
\end{minipage}
\hfill
\begin{minipage}[t][9.5cm][t]{14.35cm}
\textbf{Вариант 2: (Гу, Ко]}

Случайная величина $X$ имеет функцию плотности, заданную формулой
$$
f_X(x) = c x(x+1)
$$
определенную на области $0<x<1$.
\begin{enumerate}
    \item Найдите нормировочную константу $c$, при котором это является функцией плотности.
    \item Постройте функцию распределения случайной величины $X$.
    \item Вычислите вероятности $P(X>0.25)$, $P(X < 0.25 \mid X < 0.5)$.
    \item Вычислите $E[X]$ и $Var[X]$.
\end{enumerate}

\end{minipage}

\vspace{0.2cm}

% Нижняя часть: два квадранта рядом
\noindent
\begin{minipage}[t][9.5cm][t]{14.35cm}
\textbf{Вариант 3: (Ко, По]}

Случайная величина $X$ имеет функцию плотности, заданную формулой
$$
f_X(x) = k(3x+1)(x+1)
$$
определенную на области $0<x<1$.
\begin{enumerate}
    \item Найдите нормировочную константу $k$, при котором это является функцией плотности.
    \item Постройте функцию распределения случайной величины $X$.
    \item Вычислите вероятности $P(X>0.25)$, $P(X < 0.25 \mid X < 0.5)$.
    \item Вычислите $E[X]$ и $Var[X]$.
\end{enumerate}

\end{minipage}
\hfill
\begin{minipage}[t][9.5cm][t]{14.35cm}
\textbf{Вариант 4: (По, $+\infty$]}

Случайная величина $X$ имеет функцию плотности, заданную формулой
$$
f_X(x) = kx^2(1-x)
$$
определенную на области $0<x<1$.
\begin{enumerate}
    \item Найдите нормировочную константу $k$, при котором это является функцией плотности.
    \item Постройте функцию распределения случайной величины $X$.
    \item Вычислите вероятности $P(0.25<X<0.75)$, $P(-2<X<0.5)$.
    \item Вычислите $E[X]$ и $Var[X]$.
\end{enumerate}


\end{minipage}

\end{document}

