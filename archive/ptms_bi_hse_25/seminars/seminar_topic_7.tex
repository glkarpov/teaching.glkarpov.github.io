\documentclass{article}
\usepackage[utf8]{inputenc}
\usepackage[english,russian]{babel}
\usepackage[top=1cm,bottom=1cm,left=2cm,right=2cm]{geometry}
\usepackage{graphicx}
\usepackage{amsmath}
\usepackage{amsfonts}
\usepackage{xcolor}
\title{ВШБ БИ: ТВиМС 2025. \\ Лист seminar-only задач \#7. \\ Нормальное распределение.}
\date{}
\author{}

\begin{document}
\maketitle

\textit{Две первые простые, следующие - сложнее, на линейные комбинации нормальных случайных величин.}

\begin{enumerate}
    % src: Ross, 5.11
    \item Годовое количество осадков в Огайо приближенно распределено по нормальному закону
    со средним значением $40.2$ дюйма и стандартным отклонением $8.4$ дюйма.
    Чему равна вероятность того, что (а) количество осадков в следующем году превысит $44$ дюйма?
    (б) количество осадков ровно в $3$ из следующих $7$ лет превысит $44$ дюйма?

    Предположим, что если $A_i$ — событие, что количество осадков превышает $44$ дюйма в году $i$ (начиная с текущего),
    то события $A_i$, $i \geq 1$, независимы.

    \item Количество новых клиентов за месяц – случайная величина, распределенная по нормальному закону со средним значением $40$ человек.
    Известно, что с вероятностью $0.4$ число  новых клиентов будет в пределах от $32$ до $48$. 
    Исходя из предоставленной информации, найдите стандартное отклонение.
    
    \item Студент собирается отпраздновать свой день рождения. Для этого он хочет купить 6 шоколадок и 4 пирожка.
    Ему известно, что стоимость шоколадки – случайная величина, распределенная по нормальному закону, причем средняя стоимость шоколадки – $70$ р, а стандартное отклонение стоимости – $10$ р.
    Про пирожки известно, что их стоимость тоже распределена по нормальному закону со средним значением $30$ р и стандартным отклонением $7$ р.
    Еще ему надо купить пакет – его стоимость фиксирована и равна $12$ р.
    Найдите вероятность того, что он потратит больше $500$ р, если известно, что все указанные случайные величины независимы.
    \begin{enumerate}
        \item Все шоколадки разные, все пирожки тоже.
        \item Все шоколадки и все пирожки одинаковые.
    \end{enumerate}

    \item Известно, что величина чека – это случайная величина, распределенная по нормальному закону,
    причем для мужчины средний чек равен $1020$ рублей  со стандартным отклонением $40$ рублей,
    а для женщины – средний чек $1500$ рублей со стандартным отклонением $500$ рублей.
    За час независимо друг от друга совершили покупки 7 мужчин и 4 женщины. 
    \begin{enumerate}
        \item Найдите вероятность того, что общая сумма покупок превысит $13,300$ рублей.
        \item Найдите вероятность того, что семеро мужчин потратили суммарно больше денег, чем четыре женщины.
        \item Один из покупателей забыл взять чек.
        Найдите вероятность того, что это был мужчина, если известно, что этот чек на сумму, меньшую $1,000$ рублей.
    \end{enumerate}

\end{enumerate}

\end{document}
