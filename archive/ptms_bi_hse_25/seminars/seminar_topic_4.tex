\documentclass{article}
\usepackage[utf8]{inputenc}
\usepackage[english,russian]{babel}
\usepackage[top=1cm,bottom=1cm,left=2cm,right=2cm]{geometry}
\usepackage{graphicx}
\usepackage{amsmath}
\usepackage{amsfonts}
\title{ВШБ БИ: ТВиМС 2025. \\ Лист seminar-only задач \#4. \\ Дискретные случайные величины и их характеристики.}
\date{}
\author{}

\begin{document}
\maketitle

\begin{enumerate}
    % source: Ross, Problems, 4.13
    \item Продавец назначил две встречи для продажи 
    энциклопедий. Его первая встреча приведет к 
    продаже с вероятностью $0.3$, а вторая независимо 
    приведет к продаже с вероятностью $0.6$. 
    Любая совершенная продажа с равной вероятностью 
    может быть либо делюкс-моделью стоимостью $\$1000$, 
    либо стандартной моделью стоимостью $\$500$. 

    Пусть $X$ — случайная величина, обозначающая общую стоимость всех продаж в долларах.
    Определите пространство значений случайной величины и её функцию вероятности. Постройте ряд распределения для $X$

    Ответ: Существует 9 возможных исходов. Суммируя все возможные способы получить различные значения $X$, находим
    $$
    \begin{aligned}
    P\{X=0\} & =.28 \\
    P\{X=500\} & =.21+.06=.27 \\
    P\{X=1000\} & =.21+.045+.06=.315 \\
    P\{X=1500\} & =.045+.045=.09 \\
    P\{X=2000\} & =.045
    \end{aligned}
    $$
    
    \begin{tabular}{|l|l|l|l|}
    \hline Продажа клиенту 1 & Продажа клиенту 2 & X & Вероятность \\
    \hline 0 & 0 & 0 & $(1-.3)(1-.6)=.28$ \\
    \hline 0 & 500 & 500 & $(1-.3)(.6)(.5)=.21$ \\
    \hline 0 & 1000 & 1000 & $(1-.3)(.6)(.5)=.21$ \\
    \hline 500 & 0 & 500 & $(.3)(.5)(1-.6)=.06$ \\
    \hline 500 & 500 & 1000 & $(.3)(.5)(.6)(.5)=.045$ \\
    \hline 500 & 1000 & 1500 & $(.3)(.5)(.6)(.5)=.045$ \\
    \hline 1000 & 0 & 1000 & $(.3)(.5)(1-.6)=.06$ \\
    \hline 1000 & 500 & 1500 & $(.3)(.5)(.6)(.5)=.045$ \\
    \hline 1000 & 1000 & 2000 & $(.3)(.5)(.6)(.5)=.045$ \\
    \hline
    \end{tabular}

    % source: Ross, Example 1d and Problem 4.10
    \item Три шара случайным образом выбираются из урны, содержащей 3 белых, 3 красных и 5 черных шаров. Предположим, что мы выигрываем $\$1$ за каждый выбранный белый шар и проигрываем $\$1$ за каждый выбранный красный шар.
    Пусть $X$  - случайная величина, обозначающая наш общий выигрыш. 
    
    \begin{enumerate}
    \item Постройте ряд распределения для $X$.
    
    \item Какая вероятность того, что мы выиграем?
    
    \item Найдите условную вероятность того, что мы выиграем $i$ долларов при условии, что мы что-то выиграли.
    \end{enumerate}

    Ответ: $X$ — это случайная величина, принимающая возможные значения $0, \pm 1, \pm 2, \pm 3$ с соответствующими вероятностями
    $$
    \begin{aligned}
    & P\{X=0\}=\frac{\binom{5}{3}+\binom{3}{1}\binom{3}{1}\binom{5}{1}}{\binom{11}{3}}=\frac{55}{165} \\
    & P\{X=1\}=P\{X=-1\}=\frac{\binom{3}{1}\binom{5}{2}+\binom{3}{2}\binom{3}{1}}{\binom{11}{3}}=\frac{39}{165} \\
    & P\{X=2\}=P\{X=-2\}=\frac{\binom{3}{2}\binom{5}{1}}{\binom{11}{3}}=\frac{15}{165} \\
    & P\{X=3\}=P\{X=-3\}=\frac{\binom{3}{3}}{\binom{11}{3}}=\frac{1}{165}
    \end{aligned}
    $$
    
    Эти вероятности получаются, например, при отмечании того, что для $X$ равного 0, либо все 3 выбранных шара должны быть черными, либо должен быть выбран 1 шар каждого цвета.
    Аналогично, событие $\{X=1\}$ происходит либо если выбраны 1 белый и 2 черных шара, либо если выбраны 2 белых и 1 красный шар.
    Для проверки отметим, что
    $$
    \sum_{i=0}^3 P\{X=i\}+\sum_{i=1}^3 P\{X=-i\}=\frac{55+39+15+1+39+15+1}{165}=1
    $$
    
    Вероятность того, что мы выиграем деньги, равна
    $$
    \sum_{i=1}^3 P\{X=i\}=\frac{55}{165}=\frac{1}{3}
    $$

    Далее мы хотим вычислить условную вероятность того, что мы выиграем $i$ долларов при условии, что мы что-то выиграли. Пусть $E$ — событие того, что мы что-то выигрываем, и мы хотим вычислить $P\{X=i \mid E\}$:
$$
P\{X=i \mid E\}=\frac{P\{E \mid X=i\} P\{X=i\}}{P\{E\}} .
$$

Таким образом, имеем $P\{X=i \mid E\}=0$ при $i=0,-1,-2,-3$, и
$$
\begin{aligned}
P\{X=1 \mid E\} & =\frac{P\{X=1\}}{P\{E\}}=\frac{\frac{39}{165}}{\frac{1}{3}}=0.709 \\
P\{X=2 \mid E\} & =\frac{P\{X=2\}}{P\{E\}}=\frac{\frac{15}{165}}{\frac{1}{3}}=0.2727 \\
P\{X=3 \mid E\} & =\frac{P\{X=3\}}{P\{E\}}=\frac{\frac{1}{165}}{\frac{1}{3}}=0.01818
\end{aligned}
$$

    % source: Ross, Problems, 4.14
    \item Пять различных чисел случайным образом 
    распределены между игроками с номерами от 1 до 5. 
    Когда два игрока сравнивают свои числа, тот, 
    у кого число больше, объявляется победителем. 
    Сначала игроки 1 и 2 сравнивают свои числа; 
    затем победитель сравнивает свое число с числом 
    игрока 3, и так далее. Пусть $X$ обозначает 
    количество раз, когда игрок 1 является 
    победителем. Найдите $P\{X = i\}$, $i = 0, 1, 2, 3, 4$.

    Ответ: Для начала отметим, что существует $5!$ равновероятных возможных перестановок чисел $1-5$, которые могут быть распределены между пятью игроками. Игрок 1 выиграет 4 раза, если у него самое большое из пяти чисел. Таким образом, первый игрок должен получить число 5, за которым следует любое из $4!$ возможных упорядочиваний остальных чисел. Это дает вероятность
    $$
    P\{X=4\}=\frac{1 \cdot 4!}{5!}=\frac{1}{5} .
    $$
    
    Далее, игрок 1 выиграет 3 раза, если его число превышает числа игроков 2, 3 и 4, но меньше числа игрока 5. Другими словами, игрок 1 должен иметь второе по величине число, а игрок 5 — наибольшее. Это означает, что игроку 5 должно быть дано число 5, игроку 1 — число 4, а остальные 3 числа могут быть в любом порядке среди оставшихся игроков. Это дает вероятность
    $$
    P\{X=3\}=\frac{1 \cdot 1 \cdot 3!}{5!}=\frac{1}{20} .
    $$
    
    Чтобы игрок 1 выиграл дважды, он должен иметь число больше чисел игроков 2 и 3, но меньше числа игрока 4; то есть из первых четырех игроков у игрока 4 должно быть наибольшее число, а у игрока 1 — второе по величине. Мы выбираем четыре числа для назначения первым четырем игрокам $\binom{5}{4}$ способами. Это оставляет одно число для игрока 5. Затем мы выбираем наибольшее число из этой группы четырех для игрока 4 (одним способом), а затем выбираем второе по величине число (одним способом) для первого игрока. Это дает два оставшихся числа, которые можно упорядочить двумя способами. Это дает вероятность
    $$
    P\{X=2\}=\frac{\binom{5}{4} \cdot 1 \cdot 1 \cdot 2!\cdot 1}{5!}=\frac{1}{12}
    $$
    
    Игрок 1 выигрывает ровно один раз, если его число больше числа игрока 2 и меньше числа игрока 3. Следуя логике случая, когда мы выигрываем дважды, мы выбираем три числа для игроков 1-3 $\binom{5}{3}$ способами. Это дает два числа для игроков 4 и 5, которые можно упорядочить 2 способами. Из исходного множества трех чисел мы назначаем наибольшее из этого множества игроку 3, а следующее по величине — игроку 1. Последнее число достается игроку 2. Вместе это дает вероятность
    $$
    P\{X=1\}=\frac{\binom{5}{3} \cdot 1 \cdot 1 \cdot 1 \cdot 2!}{5!}=\frac{1}{6}
    $$
    
    Наконец, игрок 1 никогда не выигрывает, если его число меньше числа игрока 2. Та же логика, что и выше, дает для этой вероятности следующее
    $$
    P\{X=0\}=\frac{\binom{5}{2} \cdot 1 \cdot 1 \cdot 3!}{5!}=\frac{1}{2}
    $$

    % source: Ross, Problems, 4.20
    \item Книга об азартных играх рекомендует следующую "выигрышную стратегию" для игры в рулетку: Поставьте $\$1$ на красное.
    Если выпадает красное (что происходит с вероятностью $\frac{18}{38}$), то получаете прибыль в $\$1$ и прекратите игру.
    Если красное не выпадает и вы проигрываете эту ставку (что происходит с вероятностью $\frac{20}{38}$), сделайте дополнительные ставки по $\$1$ на красное в каждом из следующих двух вращений колеса рулетки, а затем прекратите игру.

    Пусть $X$ - случайная величина, обозначающая ваш выигрыш, когда вы прекращаете игру.
    \begin{enumerate}
        \item Найдите $P\{X>0\}$.
        \item Найдите $E[X]$.
        \item Убеждены ли вы, что эта стратегия действительно является "выигрышной"?
    \end{enumerate}


    Ответ: Можем составить следующую таблицу всевозможных исходов игры:
        \begin{table}[h]
        \centering
        \begin{tabular}{|c|c|c|c|c|}
        \hline Первая игра & Вторая игра & Третья игра & $X$ & Вероятность \\
        \hline выигрыш & н/д & н/д & +1 & $p$ \\
        \hline проигрыш & выигрыш & выигрыш & +1 & $p^2 q$ \\
        \hline проигрыш & выигрыш & проигрыш & -1 & $p q^2$ \\
        \hline проигрыш & проигрыш & выигрыш & -1 & $p q^2$ \\
        \hline проигрыш & проигрыш & проигрыш & -3 & $q^3$ \\
        \hline
        \end{tabular}
        \end{table}


        Если мы выигрываем в первой игре (с вероятностью $p=\frac{18}{38}$ ), мы прекращаем играть и выигрываем $+1$.
        Если мы проигрываем в первой игре, мы будем играть еще два раза. В этих двух играх мы можем выиграть дважды, выиграть один раз и проиграть один раз в разной последовательности, или проиграть дважды.
        Игнорируя пока начальный проигрыш, эти четыре исхода происходят с вероятностями:
        $$
        p^2, \quad p q, \quad q p, \quad q^2 .
        $$
        
        Выигрыш для каждого из этих исходов задается как:
        $$
        +2, \quad 0, \quad 0, \quad -2 .
        $$
        
        Поскольку дополнительные две игры играются только в случае проигрыша первой игры, мы должны учесть это при подсчете общего выигрыша и вероятностей.
        Таким образом, общие вероятности (и суммы выигрыша $X$) приведены в таблице выше. Используя их, мы можем ответить на поставленные вопросы.
        
        \begin{enumerate}
            \item $P\{X>0\}=p+p^2 q=\frac{9}{19}+\frac{10}{19}\left(\frac{9}{19}\right)^2=0.5917$.
            \item Находим $E[X]=1 p+1 p^2 q-1 p q^2-1 p q^2-3 q^3=-0.108$.
            \item Есть два варианта, где мы выигрываем, но три варианта, где мы проигрываем. Математическое ожидание отрицательное, хоть и вероятность выигрыша больше половины.
        \end{enumerate}
\end{enumerate}

\end{document}
