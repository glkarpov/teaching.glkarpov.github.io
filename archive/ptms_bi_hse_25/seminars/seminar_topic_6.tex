\documentclass{article}
\usepackage[utf8]{inputenc}
\usepackage[english,russian]{babel}
\usepackage[top=1cm,bottom=1cm,left=2cm,right=2cm]{geometry}
\usepackage{graphicx}
\usepackage{amsmath}
\usepackage{amsfonts}
\usepackage{xcolor}
\title{ВШБ БИ: ТВиМС 2025. \\ Лист seminar-only задач \#6. \\ Непрерывные случайные величины. Специальные распределения: равномерное, экспоненциальное.}
\date{}
\author{}

\begin{document}
\maketitle

\begin{enumerate}
    \item Оценка за задачу, решенную у доски, это случайная величина с плотностью
    \begin{equation*}
        f_X(x) = 
        \begin{cases}
            0, & x < 1 \\
            c, & x \in[1 ; 5] \\
            0, & x > 5
        \end{cases}
    \end{equation*}
    Найдите нормировочную константу, математическое ожидание, стандартное отклонение, функцию распределения, $P(a<X \leq b)$ при $a, b \in[1 ; 5]$. (через определения, то есть через интегралы)
    
    \textcolor{red}{Решение:}
    $\int_{-\infty}^{+\infty} f_X(x) dx = 1$ - условие «нормировки» плотности
    $$
    \begin{gathered}
    \int_{-\infty}^{+\infty} f(x) d x=\int_{-\infty}^1 0 d x+\int_1^5 c d x+\int_5^{+\infty} 0 d x=\left.c x\right|_1 ^5=(5-1) \cdot c=4 c \Rightarrow c=1 / 4 \\
    E(X)=\int_{-\infty}^{+\infty} x f(x) d x=\int_1^5 x \cdot \frac{1}{4} d x=\left.\frac{1}{4} \cdot \frac{x^2}{2}\right|_1 ^5=\frac{5^2-1^2}{8}=3 \\
    Var(X)=\int_{-\infty}^{+\infty} x^2 f(x) d x-E(X)^2=\int_1^5 x^2 \cdot \frac{1}{4} d x-9=\left.\frac{1}{4} \cdot \frac{x^3}{3}\right|_1 ^5-9=\frac{5^3-1^3}{8}-9=\frac{4}{3} \\
    F(x)=P(X \leq x)=\int_{-\infty}^{x} f(t) d t=\int_{-\infty}^1 0 d t+\int_1^x \frac{1}{4} d t=\frac{x-1}{4} \text { при } x \in[1 ; 5], \text { до этого } 0, \text { после этого } 1, \text { то есть: } \\
    F(x)=\left\{\begin{array}{c}
    0, x<1 \\
    \frac{x-1}{4}, 1 \leq x \leq 5 \\
    1, x>5
    \end{array}\right. \\
    P(a<X \leq b)=F(b)-F(a)=\frac{b-1}{4}-\frac{a-1}{4}=\frac{b-a}{4} \text { при } a, b \in[1 ; 5]
    \end{gathered}
    $$
    
    \item Оценка за каждую из задач на контрольной работе это случайная величина с плотностью
    \begin{equation*}
        f_X(x) = 
        \begin{cases}
            0, & x < 0 \\
            c x, & x \in[0 ; 2] \\
            0, & x > 2
        \end{cases}
    \end{equation*}
        \begin{enumerate}
            \item Найдите вероятность того, что такая случайная величина примет значение от $1$ до $1.5$,
            \item Найдите вероятность того, что такая случайная величина примет значение меньше $1.7$,
            \item Найдите вероятность того, что такая случайная величина отклонится от своего математического ожидания более, чем на одно стандартное отклонение.
        \end{enumerate}

    \textcolor{red}{Решение:}
    Сначала найдем константу $\int_{-\infty}^{+\infty} f(x) d x=1$ - условие «нормировки» плотности
    $$
        \int_{-\infty}^{+\infty} f(x) d x=\int_{-\infty}^0 0 d x+\int_0^2 c x d x+\int_2^{+\infty} 0 d x=\left.c \frac{x^2}{2}\right|_0 ^2=c \frac{\left(2^2-0^2\right)}{2}=2 c \Rightarrow c=1 / 2
    $$
    a) $P(1<X \leq 1.5)=\int_1^{1.5} \frac{x}{2} d x=\left.\frac{x^2}{4}\right|_1 ^{1.5}=\frac{5}{16}=0.3125$

    б) $P(X \leq 1.7)$ - работаем по определению, через интеграл от плотности вероятности:
    $$
        P(X \leq 1.7)=P(0<X \leq 1.7)=\int_0^{1.7} \frac{x}{2} d x=\left.\frac{x^2}{4}\right|_0 ^{1.7}=\frac{289}{400}=0.7725
    $$

    в) $P(|X-E(X)|< \text{std}(X))$ - сначала найдем E(X) и Var(X) :
    
    \begin{gather*}
    E(X)=\int_{-\infty}^{+\infty} x f(x) d x=\int_0^2 x \cdot \frac{1}{2} x d x=\frac{4}{3} \\
    Var(X)=\int_{-\infty}^{+\infty} x^2 f(x) d x-E(X)^2=\int_0^2 x^2 \cdot \frac{1}{2} x d x-\left(\frac{4}{3}\right)^2=2-\frac{16}{9}=\frac{2}{9}=0.222 \Rightarrow \sigma=\frac{\sqrt{2}}{3}=0.471 \\
    P(|X-E(X)|< \text{std}(X))=P\left(\frac{4}{3}-\frac{\sqrt{2}}{3}<X<\frac{4}{3}-\frac{\sqrt{2}}{3}\right)=P(0.862<X<1.805)= \\
    = \int_{0.862}^{1.805} \frac{x}{2} d x=\left.\frac{x^2}{4}\right|_{0.862} ^{1.805}=0.629
    \end{gather*}

    \item Время (в часах), которое тратится на решение каждой из задач дз, это случайная величина с плотностью 
    \begin{equation*}
        f(x) = 
        \begin{cases}
            c e^{-5 x}, & x \geq 0 \\
            0, & x<0
        \end{cases}
    \end{equation*}
    \begin{enumerate}
        \item Найдите вероятность того, что время решения окажется в пределах от $7$ до $11$ минут
        \item Найдите математическое ожидание и дисперсию такой случайной величины (через определения, то есть через интегралы)
    \end{enumerate}

    \textcolor{red}{Решение:}

    Сначала найдем константу $\int_{-\infty}^{+\infty} f(x) d x=1$ - условие «нормировки» плотности
    $$
    \int_{-\infty}^{+\infty} f(x) d x=\int_0^{+\infty} c e^{-5 x} d x=\left.c \cdot \frac{e^{-5 x}}{-5}\right|_0 ^{+\infty}=0+\frac{c}{5}=1 \Rightarrow c=5
    $$
    a) $P(7<X \leq 11)$ - так неверно, так как все измеряется в часах
    $$
    P\left(\frac{7}{60}<X \leq \frac{11}{60}\right) \int_{\frac{7}{60}}^{\frac{11}{60}} f(x) d x=\int_{7 / 60}^{11 / 60} 5 e^{-5 x} d x=-\left.e^{-5 x}\right|_{7 / 60} ^{11 / 60}=-e^{-5 \cdot \frac{11}{60}}-\left(-e^{-5 \cdot \frac{7}{60}}\right)=e^{-5 \cdot \frac{7}{60}}-e^{-5 \cdot \frac{11}{60}}=0.158
    $$
    b) $E(X)=\int_{-\infty}^{+\infty} x f(x) d x=\int_0^{+\infty} x \cdot 5 e^{-5 x} d x=-\int_0^{+\infty} x d e^{-5 x}=-\left(\left.e^{-5 x} \cdot x\right|_0 ^{+\infty}-\int_0^{+\infty} e^{-5 x} d x\right)=-\left(0-\left.\frac{e^{-5 x}}{-5}\right|_0 ^{+\infty}\right)= -\left.\frac{e^{-5 x}}{5}\right|_0 ^{+\infty}=\frac{1}{5}$
    
    
    $Var(X) = E(X^2)  - (E(X))^2$, найдем отдельно $E(X^2) = \int_{-\infty}^{+\infty} x^2 f(x) dx$
    
    \begin{gather*}
    \int_0^{+\infty} x^2 5 e^{-5 x} d x=-\int_0^{+\infty} x^2 d e^{-5 x}=-\left(\left.e^{-5 x} \cdot x^2\right|_0 ^{+\infty}-\int_0^{+\infty} 2 x e^{-5 x} d x\right)=2 \int_0^{+\infty} x \cdot e^{-5 x} d x \\
    2 \int_0^{+\infty} x \cdot e^{-5 x} d x=2 \cdot \frac{1}{5} \int_0^{+\infty} x \cdot 5 e^{-5 x} d x=2 \cdot \frac{1}{5} \cdot \frac{1}{5}=\frac{2}{25}
    \end{gather*}
    $$
    E(X^2) = \frac{2}{25} \Rightarrow Var(X) = \frac{2}{25} - \left(\frac{1}{5}\right)^2 = \frac{1}{25}
    $$

    \newpage
    \item (Простая! Быстро пока не нахожу другого. Захотите - можно ее для разминки, по крайней мере, её нет в дз.)

    Андрей считает, что общее число километров, которое автомобиль может проехать до того, как его нужно будет совсем списать, является экспоненциальной случайной величиной с параметром $\frac{1}{20}$.

    У Максима есть подержанный автомобиль, который, по его словам, и по показаниям одометра, проехал только $10,000$ км. (Но кто верит одометру, верно? :))
    Если Андрей покупает этот автомобиль, какова вероятность того, что он проедет на нём хотя бы $20,000$ дополнительных километров?

    Повторите вычисления в предположении, что пробег автомобиля распределён не экспоненциально, а равномерно на интервале $(0, 40)$ (в тысячах километров).

    \textcolor{red}{Решение:}
    Поскольку экспоненциальная случайная величина не имеет памяти, тот факт, что автомобиль проехал $10,000$ км, не имеет значения. Вероятность, которую мы ищем, равна
    $$
    P\{T>20000\}=1-P\{T<20000\}=1-\left(1-e^{-\frac{1}{20}(20)}\right)=e^{-1}
    $$

    Если распределение пробега не экспоненциальное, а равномерное на $(0,40)$, то искомая вероятность задаётся формулой
    $$
    P\left\{T_{\text {тыс }}>30 \mid T_{\text {тыс }}>10\right\}=\frac{P\{T>30\}}{P\{T>10\}}=\frac{(1 / 4)}{(3 / 4)}=\frac{1}{3} .
    $$

    \newpage
    \item На отрезке длины $L$ абсолютно случайным образом выбирается координата точки согласно равномерному распределению.
    Найдите вероятность того, что отношение длины меньшего отрезка к длине большего отрезка меньше $\frac{1}{4}$.

    \textcolor{red}{Решение:}
    Интерпретация задачи состоит в том, что точка "$X$" выбирается из равномерного распределения на интервале $[0, L]$. Тогда задача требует найти
    $$
    P\left\{\frac{\min (X, L-X)}{\max (X, L-X)}<\frac{1}{4}\right\}
    $$

    Эту вероятность можно вычислить, интегрируя по подходящей области:
    $$
    \int_E f_X(x) d x
    $$
    где $f_X(x)$ --- плотность равномерного распределения для нашей задачи, т.е. $\frac{1}{L}$, а множество "$E$" --- это некое множество значений $X$, удовлетворяющее неравенству выше, т.е.
    $$
    \min (x, L-x) \leq \frac{1}{4} \max (x, L-x)
    $$

    Нужно догадаться, что области значений $X$, по которым мы должны вычислять интеграл выше, ограничены двумя концами отрезка.
    Интеграл выше становится равным
    $$
    \int_0^{l_1} f_X(x) d x+\int_{l_2}^L f_X(x) d x
    $$

    Для $l_1$ мы должны решить
    $$
    x=\frac{1}{4}(L-x)
    $$
    решением которого является $x=\frac{L}{5}$. Для $l_2$ мы должны решить
    $$
    L-x=\frac{1}{4} x
    $$
    решением которого является $x=\frac{4}{5} L$. С этими двумя пределами считаем вероятность:
    $$
    \int_0^{\frac{L}{5}} \frac{1}{L} d x+\int_{\frac{4}{5} L}^L \frac{1}{L} d x=\frac{1}{5}+\frac{1}{5}=\frac{2}{5}
    $$

\newpage
    \item 
    \begin{enumerate}
        \item Пожарная станция должна быть расположена вдоль дороги длины $A$, $A < \infty$.
        Если пожары происходят в точках, равномерно выбранных на $(0, A)$, где должна быть расположена станция, чтобы минимизировать ожидаемое расстояние до пожара?
        То есть, выберите $a$ так, чтобы минимизировать $E[|X-a|]$, когда $X$ равномерно распределена на $(0, A)$.

        \item Теперь предположим, что дорога имеет бесконечную длину --- простирается от точки $0$ до $\infty$.
        Если расстояние пожара от точки $0$ распределено экспоненциально с параметром $\lambda$, где теперь должна быть расположена пожарная станция?
        То есть, мы хотим минимизировать $E[|X-a|]$, где $X$ теперь распределена экспоненциально с параметром $\lambda$.
    \end{enumerate}

    \textcolor{red}{Решение:}

    Часть (a): Пусть $X$ (место пожара) равномерно распределена на $(0, A)$, и мы хотим выбрать $a$ (положение пожарной станции) так, чтобы
$l(a) = E[|X-a|]$ было минимальным.
Мы вычислим это, разбив интеграл, участвующий в определении математического ожидания, на области, где $x-a$ отрицательно и положительно. Получаем, что
\begin{gather*}
E[|X-a|] = \int_0^A|x-a| \frac{1}{A} d x \\
= -\frac{1}{A} \int_0^a(x-a) d x + \frac{1}{A} \int_a^A(x-a) d x \\
= -\left.\frac{1}{A} \frac{(x-a)^2}{2}\right|_0 ^a + \left.\frac{1}{A} \frac{(x-a)^2}{2}\right|_a ^A \\
= -\frac{1}{A}\left(0-\frac{a^2}{2}\right) + \frac{1}{A}\left(\frac{(A-a)^2}{2}-0\right) \\
= \frac{a^2}{2 A} + \frac{(A-a)^2}{2 A}
\end{gather*}

Чтобы найти $a$, минимизирующее это выражение, мы вычисляем $l^{\prime}(a)$ и приравниваем к нулю:
$$
l^{\prime}(a)=\frac{a}{A}+\frac{2(A-a)(-1)}{2 A}=0 .
$$

Отсюда получаем решение $a^{\ast}$, задаваемое формулой $a^{\ast}=\frac{A}{2}$. Вторая производная нашей функции $l$ показывает, что $l^{\prime \prime}(a)=\frac{2}{A}>0$, что означает, что точка $a^{\ast}=A / 2$ действительно является минимумом.

Часть (b): Формулировка задачи та же, что и в части (a), но поскольку распределение положения пожаров теперь экспоненциальное, мы хотим минимизировать
$$
l(a) = E[|X-a|]=\int_0^{\infty}|x-a| \lambda e^{-\lambda x} d x
$$

Мы вычислим это, разбив интеграл, участвующий в определении математического ожидания, на области, где $x-a$ отрицательно и положительно. Получаем, что

\begin{gather*}
E[|X-a|] = \int_0^{\infty}|x-a| \lambda e^{-\lambda x} d x \\
= -\int_0^a(x-a) \lambda e^{-\lambda x} d x + \int_a^{\infty}(x-a) \lambda e^{-\lambda x} d x \\
= -\lambda\left(\left.\frac{(x-a)}{-\lambda} e^{-\lambda x}\right|_0 ^a + \frac{1}{\lambda} \int_0^a e^{-\lambda x} d x\right) \\
+ \lambda\left(\left.\frac{(x-a)}{-\lambda} e^{-\lambda x}\right|_a ^{\infty} + \frac{1}{\lambda} \int_a^{\infty} e^{-\lambda x} d x\right) \\
= -\lambda\left(\frac{-a}{\lambda} - \left.\frac{1}{\lambda^2} e^{-\lambda x}\right|_0 ^a\right) + \lambda\left(0 - \left.\frac{1}{\lambda^2} e^{-\lambda x}\right|_a ^{\infty}\right) \\
= a + \frac{1}{\lambda}\left(e^{-\lambda a} - 1\right) - \frac{1}{\lambda}\left(-e^{-\lambda a}\right) \\
= a + \frac{-1 + 2 e^{-\lambda a}}{\lambda}
\end{gather*}


Чтобы найти $a$, минимизирующее это выражение, мы вычисляем $l^{\prime}(a)$ и приравниваем к нулю. Получаем, что
$$
l^{\prime}(a)=1-2 e^{-\lambda a}=0
$$

Отсюда получаем решение $a^{\ast}$, задаваемое формулой $a^{\ast}=\frac{\ln (2)}{\lambda}$.
Вторая производная нашей функции $l$ показывает, что $l^{\prime \prime}(a)=2 \lambda e^{-\lambda a}>0$, что означает,
что точка $a^{\ast}=\frac{\ln (2)}{\lambda}$ действительно является минимумом.

\end{enumerate}
\end{document}
