\documentclass{article}
\usepackage[utf8]{inputenc}
\usepackage[english,russian]{babel}
\usepackage[top=1cm,bottom=1cm,left=2cm,right=2cm]{geometry}
\usepackage{graphicx}
\usepackage{amsmath}
\usepackage{amsfonts}
\title{ВШБ БИ: ТВиМС 2025. \\ Лист seminar-only задач \#1. \\ Классическая и комбинаторная вероятность.}
\date{}
\author{}

\begin{document}
\maketitle

\begin{enumerate}

    \item На день рождения к Васе пришли две Маши, два Саши, Петя и Коля. Все вместе с Васей сели за круглый стол.
    Какова вероятность, что Вася окажется между двумя тезками?

    \textit{Решение.}
    Всего $7!$ рассадок.У Васи 7 мест. Слева может сесть один из 4 тезок, справа ровно один, парный к левому. Остальные могут сесть $4!$ способами.
$7 \cdot 4 \cdot 1 \cdot 4! / 7! = 4 / (6 \cdot 5) = 2/15$.

\item Алиса бросает четыре монеты, Борис бросает кубик. Найти вероятность того, что у Алисы выпадет не меньше орлов, чем результат броска Бориса.

    \textit{Решение.}
    \begin{itemize}
        \item 4 орла (один вариант) И 4 или 3 или 2 или 1 на кубике – 4 благоприятных исхода
        \item 3 орла (4 варианта) И 3 или 2 или 1 на кубике – 12 благоприятных исходов
        \item 2 орла (6 вариантов) И 2 или 1 на кубике – 12 благоприятных исходов
        \item 1 орел (4 варианта) И 1 на кубике – 4 благоприятных исхода
    \end{itemize}
    Всего благоприятных исходов $32$, общее число исходов $2 \cdot 2 \cdot 2 \cdot 2 \cdot 6$. Ответ: $\frac{32}{2 \cdot 2 \cdot 2 \cdot 2 \cdot 6}$.

    \item В лифт 12-этажного дома на первом этаже вошли 11 человек. Каждый из них выходит независимо от других и с равной вероятностью на любом из этажей, начиная со второго. Найдите вероятность того, что
    \begin{enumerate}
        \item поднимаясь вверх, на каждом этаже со второго по 12-й будет выходит ровно один человек; 
        \item все пассажиры выйдут не выше 9-го этажа, если никто из них не вышел со второго по шестой этажи.
    \end{enumerate}
    
    \textit{Решение.}
    \begin{enumerate}
        \item общее число исходов $11^{11}$ так как каждому из пассажиров можно приписать любой номер с 2 до 12.
        первому из них можно приписать любой из 11 возможных этажей, второму любой из 10 и тд. Ответ $\frac{11!}{11^{11}}$.
        \item получается, что в нашем распоряжении этажи 7, 8, 9, 10, 11, 12, то есть общее число исходов $11^6$.
        Если все вышли не выше 9-го, то в нашем распоряжении 7, 8, 9, то есть число благоприятных исходов $3^6$. ОТВЕТ: $\frac{3^6}{11^6}$
    \end{enumerate}

    \item Восемь  студентов (два Егора, два Михаила, две Оли, Аня и Ира) стоят в очереди за пирожками.
    \begin{enumerate}
        \item Найти вероятность того, что юноши и девушки чередуются.
        \item Найти вероятность того, что Аня стоит между двумя тезками.
    \end{enumerate}
    
    \textit{Решение.}
    \begin{enumerate}
        \item $2 \cdot \frac{4! \cdot 4!}{8!}$ на 4 четных ставим мальчиков, на нечетные девочек, потом умножаем это на два так как они могут встать наоборот – четные Д, нечетные М.
        \item  Всего $8!$ вариантов. Объединим Егоров и Аню в одного человека Е1-А-Е2, остальные М1, М2, О1, О2 и И остаются как есть.
        Из получившихся шести людей можно собрать $6!$ очередей. Так как Егоров можно поменять местами, то получаем 
        $2 \cdot 6!$ вариантов. Вместо Егоров можно брать Михаилов или Оль, то есть всего благоприятных исходов $3 \cdot 2 \cdot 6!$
        ответ $\frac{3 \cdot 2 \cdot 6!}{8!} = \frac{6}{56}$
        
        Другое решение: Аню можно поставить на любое из 6 мест не по краям, слева от нее можно поставить любого из 6 тех, у кого есть тезки, справа - только парного тому кто левее Ани.
        Остальных пятерых произвольно, то есть ответ $6 \cdot 6 \cdot \frac{5!}{8!} = \frac{6}{56}$
    \end{enumerate}    
    \item Ленивый преподаватель не хочет проверять работы 5 студентов, и отдает эти работы для проверки самим студентам, распределяя их случайным образом. Найти вероятность того, что никто из студентов не получит для проверки свою работу.
    
    \textit{Решение.}
    Всего вариантов $5!$. Благоприятные: либо все пятеро поменялись все по какому-то циклу – А получил одну из чужих работ, для этого 4 варианта, тот, чью работу получил А, может взять любую оставшуюся кроме А (иначе будет цикл) – еще 3 варианта, и так далее. Всего $4!=24$ вариантов.
Либо трое поменялись в цикле и оставшиеся двое тоже поменялись. Это $C_5^3 \cdot 2! \cdot C_2^2 \cdot 1!=20$. Итого благоприятных $44$.

\end{enumerate}
\end{document}
