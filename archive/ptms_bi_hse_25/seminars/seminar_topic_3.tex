\documentclass{article}
\usepackage[utf8]{inputenc}
\usepackage[english,russian]{babel}
\usepackage[top=1cm,bottom=1cm,left=2cm,right=2cm]{geometry}
\usepackage{graphicx}
\usepackage{amsmath}
\usepackage{amsfonts}
\title{ВШБ БИ: ТВиМС 2025. \\ Лист seminar-only задач \#3. \\ Условная вероятность. Независимость событий. Полная  вероятность. Теорема Байеса.}
\date{}
\author{}

\begin{document}
\maketitle

\begin{enumerate}
    \item Погода завтра может быть ясной с вероятностью $0.3$ и пасмурной с вероятностью $0.7$.
    Вне зависимости от того, какая будет погода, Маша даёт верный прогноз с вероятностью $0.8$.
    Вовочка, не разбираясь в погоде, делает свой прогноз по принципу: с вероятностью $0.9$ копирует Машин прогноз, и с вероятностью $0.1$ меняет его на противоположный.
    \begin{enumerate}
        \item Какова вероятность того, что Маша спрогнозирует ясный день?
        \item Какова вероятность того, что Машин и Вовочкин прогнозы совпадут?
        \item Какова вероятность того, что день будет ясный, если Маша спрогнозировала ясный?
        \item Какова вероятность того, что день будет ясный, если Вовочка спрогнозировал ясный?
    \end{enumerate}
    
    Ответы:
    \begin{enumerate}
        \item $\mathbb{P}(A)=0.8 \cdot 0.3+0.7 \cdot 0.2=0.38$
        \item $\mathbb{P}(B)=0.9$
        \item $\mathbb{P}(C \mid A)=\frac{0.3 \cdot 0.8}{0.38}=0.632$
        \item $\mathbb{P}(C \mid D)=\frac{0.3 \cdot(0.9 \cdot 0.8+0.1 \cdot 0.2)}{0.9 \cdot 0.38+0.1 \cdot(1-0.38)}=0.55$
    \end{enumerate}

    \item Юра и Петя подбросили монетку по четыре раза.
    \begin{enumerate}
        \item Найти вероятность того, что у них выпало одинаковое количество орлов.
        \item Известно, что у них выпало одинаковое количество орлов.
    Найти вероятность того, что у них выпало по два орла.
    \end{enumerate}

    \item Юра и Петя подбросили кубик по четыре раза.
    \begin{enumerate}
        \item Найти вероятность того, что у них выпало одинаковое количество шестерок.
        \item Известно, что у них выпало одинаковое количество шестерок.
    Найти вероятность того, что у них выпало по две шестерки.
    \end{enumerate}
    
    % source: ICEF Introductory Probability and Statistics Examination, 18.10.2021
    \item
    \begin{enumerate}
    \item Сначала бросается игральная кость. Затем подбрасывается $n$ монет, где $n$ — число очков на кости.
    Найдите вероятность того, что все монеты выпадут орлом вверх.
    \item В контексте предыдущего вопроса предположим, что все монеты выпали орлом вверх. Какова
    условная вероятность того, что на кости выпало $3$ очка?
    \end{enumerate}

    \item
    В ряду 8 мест, сидят А, Б, В, Г, Д. Известно, что между А и Б никто не сидит. Найти вероятность того, что они сидят на соседних стульях

    \item Известно, что $P(A) = 0.3$, $P(B) = 0.4$, $P(C) = 0.5$. События $A$ и $B$ несовместны, события $A$ и $C$ независимы и $P(B | C) = 0.1$.
Найдите $P(A \cup B \cup C)$.

    \item Имеется три монетки. Две «правильных» и одна — с орлами по обеим сторонам.
    Петя выбирает одну монетку наугад и подкидывает её два раза.
    Оба раза выпадает орёл. Какова вероятность того, что монетка «неправильная»?

    \item Имеются три урны и обычный шестигранный кубик.
    В первой урне лежат $8$ черных и $2$ белых шариков.
    Во второй урне находятся $6$ белых и $14$ черных шариков.
    В третью урну положили $24$ белых и $6$ черных шариков.
    Если на кубике выпадает четное число, то Лаврентий наугад достает шарик из первой урны.
    Если выпадает нечетное число меньше $5$, то Лаврентий наугад берет шарик из второй урны.
    В противном случае Лаврентий достает наугад шарик из третьей урны. Найдите вероятность того, что:
    \begin{enumerate}
        \item Лаврентий достанет белый шарик.
        \item Была выбрана третья урна, если Лаврентий достал белый шарик.
        \item На кубике выпало менее 3 -х очков, если Лаврентий достал белый шарик. (доп.)
    \end{enumerate}

    Решение:
a) Обозначим через $U_i$ событие, при котором Лаврентий берет шарик из $i$-й урны, где $i \in \{1,2,3\}$. Рассчитаем вероятности данных событий:
$$
\begin{gathered}
P\left(U_1\right)=P(\text { на кубике выпало } 2,4 \text { или } 6 \text { очков })=\frac{3}{6}=\frac{1}{2} \\
P\left(U_2\right)=P(\text { на кубике выпало } 1 \text { или } 3 \text { очка })=\frac{2}{6}=\frac{1}{3} \\
P\left(U_3\right)=P(\text { на кубике выпало } 5 \text { очков })=\frac{1}{6}
\end{gathered}
$$

Обратим внимание, что события $U_1, U_2$ и $U_3$ составляют полную группу попарно несовместных событий. При этом вероятности вида $P\left(W \mid U_i\right)$ считаются достаточно просто. Например, вероятность события $W \mid U_2$ это вероятность достать белый шарик из второй урны и, поскольку во второй урне из $6+14=20$ шариков 6 являются белыми, то вероятность этого события составит $\frac{6}{20}$. Следовательно, для нахождения вероятности события $W$, в соответствии с которым Лаврентий достает белый шарик, удобно воспользоваться формулой полной вероятности:
$$
\begin{gathered}
P(W)=P\left(W \mid U_1\right) P\left(U_1\right)+P\left(W \mid U_2\right) P\left(U_2\right)+P\left(W \mid U_3\right) P\left(U_3\right)= \\
\frac{2}{2+8} \times \frac{1}{2}+\frac{6}{6+14} \times \frac{1}{3}+\frac{24}{24+6} \times \frac{1}{6}=\frac{1}{3}
\end{gathered}
$$
б) Воспользуемся формулой условной вероятности:
$$
P\left(U_3 \mid W\right)=\frac{P\left(W \mid U_3\right) P\left(U_3\right)}{P(W)}=\frac{\frac{24}{24+6} \times \frac{1}{6}}{\frac{1}{3}}=\frac{2}{5}
$$
в) Обозначим через $G$ событие, при котором на кубике выпало менее 3 -х очков. Найдем условные вероятности урн:
$$
\begin{gathered}
P\left(U_1 \mid G\right)=P(\text { на кубике выпало } 2,4 \text { или } 6 \text { очков } \mid \text { выпало менее } 3 \text {-х очков })= \\
\quad=P(\text { выпало } 2 \text { очка } \mid \text { выпало } 1 \text { или } 2 \text { очка })=\frac{1}{2} \\
P\left(U_2 \mid G\right)=P(\text { на кубике выпало } 1 \text { или } 3 \text { очка } \mid \text { выпало менее } 3 \text {-х очков })= \\
\quad=P(\text { выпало } 1 \text { очко } \mid \text { выпало } 1 \text { или } 2 \text { очка })=\frac{1}{2} \\
P\left(U_3 \mid G\right)=P(\text { на кубике выпало } 5 \text { очков } \mid \text { выпало менее } 3 \text {-х очков })= \\
\quad=P(\emptyset \mid \text { выпало } 1 \text { или } 2 \text { очка })=0
\end{gathered}
$$

Применяя формулу полной вероятности рассчитаем условную вероятность достать белый шарик. При этом учтем, что при условии наступления события $G$ событие $U_3$ является невозможным.

\begin{gather*}
P(W \mid G) =P\left(W \mid U_1 \cap G\right) P\left(U_1 \mid G\right)+P\left(W \mid U_2 \cap G\right) P\left(U_2 \mid G\right)= \\
=P\left(W \mid U_1\right) P\left(U_1 \mid G\right)+P\left(W \mid U_2\right) P\left(U_2 \mid G\right)= \\
\frac{2}{2+8} \times \frac{1}{2}+\frac{6}{6+14} \times \frac{1}{2}=\frac{1}{4}
\end{gather*}


С помощью формулы условной вероятности получаем ответ:
$$
P(G \mid W)=\frac{P(W \mid G) P(G)}{P(W)}=\frac{\frac{1}{4} \times \frac{2}{6}}{\frac{1}{3}}=\frac{1}{4}
$$

\item При переливании крови надо учитывать группы крови донора и больного.
Человеку, имеющему четвертую (AB) группу крови, можно перелить кровь любой группы.
Человеку со второй (A) или третьей (B) группой можно перелить кровь той же группы или первой.
Человеку с первой (0) группой крови только кровь первой группы.
Среди населения $33,7 \%$ имеют первую, $37,5 \%$ - вторую, $20,9 \%$ - третью и $7,9 \%$ - четвертую группы крови.

\begin{enumerate}
    \item Найдите вероятность того, что случайно взятому больному можно перелить кровь случайно взятого донора.
    \item Найдите вероятность того, что переливание можно осуществить, если есть два донора.
\end{enumerate}

Ответы:
\begin{enumerate}
    \item $\mathbb{P}\left(A_1\right)=0,079+0,209(0,209+0,337)+0,375(0,375+0,337)+0,337 \cdot 0,337 \approx 0,574$
    \item $\mathbb{P}\left(A_2\right) \approx 0,778$
\end{enumerate}

\item Предположим, что социологическим опросам доверяют $70 \%$ жителей.
Те, кто доверяют, опросам всегда отвечают искренне; те, кто не доверяют, отвечают наугад.
Социолог Петя в анкету очередного опроса включил вопрос "Доверяете ли Вы социологическим опросам?".
\begin{enumerate}
    \item Какова вероятность, что случайно выбранный респондент ответит «Да»?
    \item Какова вероятность того, что он действительно доверяет, если известно, что он ответил «Да»?
\end{enumerate}

Решение:
\begin{enumerate}
    \item $0.7+0.3 \cdot 0.5=0.85$
    \item $\frac{0.7}{0.85}=\frac{14}{17} \approx 0.82$
\end{enumerate}

\item Снайпер попадает в «яблочко» с вероятностью $0.8$,
если в предыдущий раз он попал в «яблочко» и с вероятностью $0.7$,
если в предыдущий раз он не попал в «яблочко» или если это был первый выстрел.
Снайпер стрелял по мишени $3$ раза.
\begin{enumerate}
    \item Какова вероятность попадания в «яблочко» при втором выстреле?
    \item Какова вероятность попадания в «яблочко» при втором выстреле, если при первом снайпер попал, а при третьем - промазал?
\end{enumerate}

Решение:
\begin{enumerate}
    \item $0.7 \cdot 0.8+0.3 \cdot 0.7=0.77$
    \item $\frac{0.7 \cdot 0.8 \cdot 0.2}{0.7 \cdot 0.8 \cdot 0.2+0.7 \cdot 0.2 \cdot 0.3}=\frac{8}{11}$
\end{enumerate}

\item В анкету для чиновников включён скользкий вопрос: «Берёте ли Вы взятки?».
Чтобы стимулировать чиновников отвечать правдиво, используется следующий прием.
Перед ответом на вопрос чиновник втайне от анкетирующего подкидывает специальную монетку, на гранях которой написано «правда», «ложь».
Если монетка выпадает «правдой», то предлагается отвечать на вопрос правдиво, если монетка выпадает на «ложь», то предлагается солгать.
Таким образом, ответ «да» не обязательно означает, что чиновник берёт взятки.
Допустим, что треть чиновников берёт взятки, а монетка неправильная и выпадает «правдой» с вероятностью $0.2$.
    \begin{enumerate}
        \item Какова вероятность того, что чиновник ответит «да»? 
        \item Какова вероятность того, что чиновник берёт взятки, если он ответил «да»? Если ответил «нет»?
    \end{enumerate}
\end{enumerate}
\end{document}
