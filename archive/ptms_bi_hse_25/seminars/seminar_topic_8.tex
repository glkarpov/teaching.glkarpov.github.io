\documentclass{article}
\usepackage[utf8]{inputenc}
\usepackage[english,russian]{babel}
\usepackage[top=1cm,bottom=1cm,left=2cm,right=2cm]{geometry}
\usepackage{graphicx}
\usepackage{amsmath}
\usepackage{amsfonts}
\usepackage{xcolor}
\title{ВШБ БИ: ТВиМС 2025. \\ Лист seminar-only задач \#8. \\ Многомерные случайные величины. Ковариация и корреляция.}
\date{}
\author{}

\begin{document}
\maketitle

\begin{enumerate}

    % src: Example 3.22, Grimmet, Welsh
    \item Предположим, что $X$ имеет распределение, заданное следующим образом:

    $$
        P\{X = -1 \} = P\{X = 0 \} = P\{ X = 1 \} = \frac{1}{3},
    $$

    а $Y$ задана следующим образом:

    $$ Y = 
    \begin{cases}
    0, \quad \text{ если } X = 0 \\
    1, \quad \text{ иначе}
    \end{cases}
    $$

    Проверьте, являются ли $X$ и $Y$ независимыми. Найдите ковариацию между $X$ и $Y$.

    \textbf{Решение:}
    Основная проблема может быть в интерпретации условия. 

    На самом деле там зашита условная вероятность: $Y=0$ \textbf{только если} $X=0$,
    то есть $P(Y=0 \mid X = -1) = 0$ и $P(Y=0 \mid X=1) = 0$, а еще $Y$ точно не может быть равна $1$, если $X=0$, то есть $P(Y=1 \mid X=0) = 0$.

    Тогда:
    $$
    P(X=-1, Y=0) = P(Y=0 \mid X=-1)P(X=-1) = 0 \times \frac{1}{3} = 0,
    $$

    $$
    P(X=1, Y=0) = P(Y=0 \mid X=1)P(X=1) = 0 \times \frac{1}{3} = 0,
    $$

    $$
    P(X=0, Y=1) = P(Y=1 \mid X=0)P(X=0) = 0 \times \frac{1}{3} = 0,
    $$

    А дальше восстанавливаем по таблице, зная маргинальные вероятности $X$.

    \begin{center}
        \begin{tabular}{c | c c c c} 
        & $Y=0$ & $Y=1$ \\ [0.5ex] 
        \hline
        $X = -1 $ & $0$ & $1/3$ \\ 
        $X = 0$ & $1/3$ & $0$  \\ 
        $X = 1$ & $0$ & $1/3$ \\ 
        \hline
       \end{tabular}
   \end{center}

   В процессе решения дальше получаем тот самый пример, когда ковариация равна нулю, но величины зависимы.
   
    \item Известно, что $E(X)=-1, E(Y)=1, Var(X)=9, Var(Y)=4, \text{Corr}(X, Y)=1$. Найдите
    \begin{enumerate}
        \item $E(Y - 2X - 3), Var(Y - 2X - 3)$,
        \item $\text{Corr}(Y - 2X - 3, X)$,
        \item Можно ли выразить $Y$ через $X$ ? Если да, то запишите уравнение связи.
    \end{enumerate}


    \item Вася сидит на контрольной работе между Дашей и Машей и отвечает на $10$ тестовых вопросов.
    На каждый вопрос есть два варианта ответа, «да» или «нет».
    Первые три ответа Васе удалось списать у Маши, следующие три - у Даши, а оставшиеся четыре пришлось проставить наугад.
    Маша ошибается с вероятностью $0.1$, а Даша - с вероятностью $0.7$.
    \begin{enumerate}
        \item Найдите вероятность того, что Вася ответил на все $10$ вопросов правильно.
        \item Вычислите корреляцию между числом правильных ответов Васи и Даши, Васи и Маши.
    \end{enumerate}
    Подсказка: иногда задача упрощается, если представить случайную величину в виде суммы.

    \newpage
    \item Из коробки, содержащей три синих стержня, два красных стержня и три зелёных стержня, случайным образом выбираются два стержня для шариковой ручки.
    Определим следующие случайные величины:
    $X=$ количество выбранных синих стержней,
    $Y=$ количество выбранных красных стержней.
    \begin{enumerate}
        \item Покажите, что $P(X=1, Y=1) = \frac{3}{14}$.
        \item Составьте таблицу совместного распределения вероятностей $X$ и $Y$.
        \item Найдите ковариацию между $X$ и $Y$.
        \item Являются ли $X$ и $Y$ независимыми случайными величинами? Обоснуйте свой ответ.
    \end{enumerate}

    \textbf{Решение:}
    \begin{enumerate}
        \item При обозначениях $B$ синий и $R$ красный:
        $$
            P(X=1, Y=1)=P(B R)+P(R B)=\frac{3}{8} \times \frac{2}{7}+\frac{2}{8} \times \frac{3}{7}=\frac{3}{14} .
        $$

    \item Имеем:
    
    \begin{tabular}{cc|c|c|c} 
        & & \multicolumn{3}{|c}{$X=x$} \\
        & & 0 & 1 & 2 \\
        \hline & 0 & $\frac{3}{28}$ & $\frac{9}{28}$ & $\frac{3}{28}$ \\
        $Y=y$ & 1 & $\frac{3}{14}$ & $\frac{3}{14}$ & 0 \\
        & 2 & $\frac{1}{28}$ & 0 & 0
    \end{tabular}

    \item Маргинальное распределение $X$:

    \begin{tabular}{c|c|c|c}
    $X=x$ & 0 & 1 & 2 \\
    \hline$p_X(x)$ & $\frac{10}{28}$ & $\frac{15}{28}$ & $\frac{3}{28}$
    \end{tabular}
    
    Отсюда:
    $$
    \mathrm{E}(X)=0 \times \frac{10}{28}+1 \times \frac{15}{28}+2 \times \frac{3}{28}=\frac{3}{4} .
    $$
    
    Маргинальное распределение $Y$:
    
    \begin{tabular}{l|c|c|c}
    $Y=y$ & 0 & 1 & 2 \\
    \hline$p_Y(y)$ & $\frac{15}{28}$ & $\frac{12}{28}$ & $\frac{1}{28}$
    \end{tabular}
    
    Отсюда:
    $$
    \mathrm{E}(Y)=0 \times \frac{15}{28}+1 \times \frac{12}{28}+2 \times \frac{1}{28}=\frac{1}{2} .
    $$
    
    
    Распределение $XY$:
    
    \begin{tabular}{l|c|c}
    $X Y=x y$ & 0 & 1 \\
    \hline$p_{X Y}(x y)$ & $\frac{22}{28}$ & $\frac{6}{28}$
    \end{tabular}
    
    Отсюда:
    $$
    \mathrm{E}(X Y)=0 \times \frac{22}{28}+1 \times \frac{6}{28}=\frac{3}{14}
    $$
    и:
    $$
    \text{Cov}(X, Y)=\mathrm{E}(X Y)-\mathrm{E}(X) \mathrm{E}(Y)=\frac{3}{14}-\frac{3}{4} \times \frac{1}{2}=-\frac{9}{56} .
    $$
    \item Поскольку $\text{Cov}(X, Y) \neq 0$, необходимое условие независимости не выполняется. Случайные величины не являются независимыми.
    \end{enumerate}

    \newpage
    \item Случайные величины $X_1$ и $X_2$ независимы и имеют одинаковую функцию вероятности $p_X(x)$ (т.е. вместо $X$ подставляем $X_1$ или $X_2$), заданную следующим образом:

    \begin{tabular}{c|c|c|c|c}
    $X=x$ & 0 & 1 & 2 & 3 \\
    \hline$p_X(x)$ & 0.2 & 0.4 & 0.3 & 0.1
    \end{tabular}
    
    Предположим теперь, что случайные величины $W$ и $Y$ определены как:
    $$
    W=\max \left(X_1, X_2\right), \quad Y=\min \left(X_1, X_2\right).
    $$
    
    Найдите:
    \begin{enumerate}
        \item Составьте таблицу совместного распределения $W$ и $Y$.
        \item Найдите маргинальное распределение $W$,
        \item Найдите условную функцию вероятности $Y$ при условии $W=2$,
        \item Найдите ковариацию между $W$ и $Y$.
    \end{enumerate}

    \textbf{Решение:}
    \begin{enumerate}
        \item Совместное распределение $W$ и $Y$:

\begin{tabular}{cc|cccc} 
& & \multicolumn{4}{|c}{$W=w$} \\
& & 0 & 1 & 2 & 3 \\
\hline & 0 & $(0.2)^2$ & $2(0.2)(0.4)$ & $2(0.2)(0.3)$ & $2(0.2)(0.1)$ \\
$Y=y$ & 1 & 0 & $(0.4)(0.4)$ & $2(0.4)(0.3)$ & $2(0.4)(0.1)$ \\
& 2 & 0 & 0 & $(0.3)(0.3)$ & $2(0.3)(0.1)$ \\
& 3 & 0 & 0 & 0 & $(0.1)(0.1)$ \\
\hline & & $(0.2)^2$ & $(0.8)(0.4)$ & $(1.5)(0.3)$ & $(1.9)(0.1)$
\end{tabular}


Что в итоге переходит в:

\begin{tabular}{cc|cccc} 
& & \multicolumn{4}{|c}{$W=w$} \\
& & 0 & 1 & 2 & 3 \\
\hline & 0 & 0.04 & 0.16 & 0.12 & 0.04 \\
$Y=y$ & 1 & 0.00 & 0.16 & 0.24 & 0.08 \\
& 2 & 0.00 & 0.00 & 0.09 & 0.06 \\
& 3 & 0.00 & 0.00 & 0.00 & 0.01 \\
\hline & & 0.04 & 0.32 & 0.45 & 0.19
\end{tabular}

\item Отсюда маргинальное распределение $W$:

    \begin{tabular}{l|c|c|c|c}
    $W=w$ & 0 & 1 & 2 & 3 \\
    \hline$p_W(w)$ & 0.04 & 0.32 & 0.45 & 0.19
    \end{tabular}

\item Условное распределение $Y$ при условии $W=2$:

\begin{tabular}{l|c|c|c|c}
$Y=y$ & 0 & 1 & 2 & 3 \\
\hline$p_{Y \mid W=2}(y \mid W=2)$ & $\frac{4}{15}$ & $\frac{8}{15}$ & $\frac{2}{10}$ & 0 \\
\end{tabular}

\item $E(WY)=1.69$, $E(W)=1.79$ и $E(Y)=0.81$, следовательно:
$$
\text{Cov}(W, Y) = E(W Y) - E(W) E(Y)=1.69-1.79 \times 0.81=0.2401.
$$
\end{enumerate}
\end{enumerate}

\end{document}
