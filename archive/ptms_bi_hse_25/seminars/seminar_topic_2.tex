\documentclass{article}
\usepackage[utf8]{inputenc}
\usepackage[english,russian]{babel}
\usepackage[top=1cm,bottom=1cm,left=2cm,right=2cm]{geometry}
\usepackage{graphicx}
\usepackage{amsmath}
\usepackage{amsfonts}
\title{ВШБ БИ: ТВиМС 2025. \\ Лист seminar-only задач \#2. \\ Геометрическая вероятность.}
\date{}
\author{}

\begin{document}
\maketitle

\textit{Задачи 1-3 без ответов – просто нарисовать все на плоскости и посмотреть на подходящие куски.}
\begin{enumerate}
\item Поезда метрополитена идут регулярно с интервалом 2 минуты. После занятий на Покровке студент Вася приходит на платформу метро Курская–кольцевая, чтобы доехать до Киевской, и садится в поезд в сторону Комсомольской или в поезд в сторону Таганской, в зависимости от того, какой из них приходит быстрее. 
Считая, что поезда в разных направлениях ходят независимо друг от друга, найдите вероятность того, что Васе придётся ждать поезда не более 30 секунд.

\item За время курс студенту надо написать контрольную и экзамен.
Время подготовки к каждой из этих двух работ – два независимых случайных числа в промежутке от 0 до 2 часов.
\begin{enumerate}
    \item Найти вероятность того, что суммарно на подготовку к двум работам он потратит менее часа.
    \item Найти вероятность того, что время подготовки к экзамену превзойдет время подготовки к контрольной более чем на 30 минут.
\end{enumerate}

\item Иван заранее хорошо подготовился к контрольной, поэтому его оценка – это некоторое случайное число от 7 до 10.
Сергей тоже готовился к контрольной, но не так ответственно,
поэтому его оценка – это некоторое случайное число в промежутке от 4 до 8. (будем считать, что оценки непрерывны).
\begin{enumerate}
    \item Найти вероятность того, что оценка Ивана окажется выше, чем оценка Сергея.
    \item Найти вероятность того, что оценка Ивана окажется более чем в два раза выше, чем оценка Сергея.
    \item Найти вероятность того, что хотя бы один из них получит «отл» (7.5 и выше)
\end{enumerate}

    \item Аня хватается за верёвку в форме окружности в произвольной точке.
    Боря берёт мачете и с завязанными глазами разрубает верёвку в двух случайных независимых местах.
    Аня забирает себе тот кусок, за который держится. Боря забирает оставшийся кусок.
    Вся верёвка имеет единичную длину. Найдите вероятность того, что у Ани верёвка длиннее.

    Ответ: Будем считать, что Аня схватилась в нуле, а $X$ и $Y$ – место разреза, оба число от $0$ до $1$.
    Аня возьмет больший кусок, если $|X-Y|<0.5$ – это задача о встрече, ответ $0.75$.

    \item Вася гоняет на мотоцикле по единичной окружности с центром в начале координат.
    В случайный момент времени он останавливается. 
\begin{enumerate}
    \item Найти вероятность того, что абсцисса окажется больше $\frac{1}{2}$
    \item Найти вероятность того, что ордината окажется меньше $\frac{1}{2}$ по модулю
\end{enumerate}

    Решение: рисуем окружность, вася равномерно катается, с одинаковой вероятностью окажется в любой точке.
    \begin{enumerate}
    \item $\cos \alpha > \frac{1}{2}$ – это правая 1/3 от окружности
    \item $\sin \alpha < \frac{1}{2}$ – это правая 1/6 и левая 1/6 от окружности, всего 1/3
    \end{enumerate}

    
\end{enumerate}
\end{document}
