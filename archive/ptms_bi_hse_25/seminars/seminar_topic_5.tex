\documentclass{article}
\usepackage[utf8]{inputenc}
\usepackage[english,russian]{babel}
\usepackage[top=1cm,bottom=1cm,left=2cm,right=2cm]{geometry}
\usepackage{graphicx}
\usepackage{amsmath}
\usepackage{amsfonts}
\title{ВШБ БИ: ТВиМС 2025. \\ Лист seminar-only задач \#5. \\ Распределения Бернулли, биномиальное, Пуассона.}
\date{}
\author{}

\begin{document}
\maketitle

\begin{enumerate}

\item Сколько следует сыграть партий в шахматы с вероятностью победы в одной партии, равной $1 / 3$, чтобы наивероятнейшее число побед было равно 5 ?

Решение: $n p-q \leq k \leq n p+p$
Здесь $p=1 / 3$ (вероятность победы), $q=2 / 3$ (вероятность проигрыша), $n$ - неизвестное число партий. Подставляя данные значения, получаем:
$$
\begin{aligned}
& -q \leq k-n p \leq p \\
& -2 / 3 \leq 5-n / 3 \leq 1 / 3 \\
& -17 / 3 \leq-n / 3 \leq-14 / 3 \\
& -17 \leq-n \leq-14
\end{aligned}
$$

Получаем, что $n=14,15,16, 17$.

\item
Разыскивая специальную книгу, студент решил обойти пять библиотек. Для каждой библиотеки одинаково вероятно, есть в ее фондах книга или нет. И если книга есть, то одинаково вероятно, занята она другим читателем или нет.
Известно, что студенту удалось взять книгу. Найти вероятность того, что книга была доступна ровно в одной из библиотек.

Решение $\mathrm{P}(\text{Нет в библиотеке})=1 / 2+1 / 2 \cdot 1 / 2=3 / 4$

$P(\text{Есть хотя бы в одной}) =1-(3 / 4)^{5}$

$P(\text{Есть только в одной}) = C_5^{1} \cdot (1 / 4)^{1} \cdot (3 / 4)^{4}$

\item В карточной игре четверым игрокам раздается колода из $52$ карт.
Один из игроков по результатам семи раздач не получал тузы в пяти раздачах. Чему равна вероятность такого события при честной раздаче карт?

Решение: Игрок получил $52 / 4 = 13$ карт, каждая из них не туз - это гипергеометрическое распределение.
$$
P(\text{не получить туз})= \frac{C_{48}^{13}}{C_{52}^{13}}
$$
Далее по схеме Бернулли считаем вероятность 5 успехов в семи испытаниях.

\item Студент считает, что если он возьмется изучать 4 предмета, то вероятность сдачи экзамена по каждому из них равна 0.8. Если он возьмется изучать 5 предметов, то вероятность сдать каждый отдельный предмет равна 0.7 ; в случае 6 и 7 предметов эта вероятность равна 0.6 и 0.5 соответственно. Необходимо сдать экзамен по меньшей мере по четырем предметам. Сколько предметов он должен выбрать, чтобы иметь наилучшие шансы достижения этой цели?

Ответ:6. Вероятности: $0.4096,0.52822,0.54432,0.5$.
\item На предприятии работает 8 служащих. Эти служащие завтракают в одной из двух закусочных, причем выбор ими той или другой закусочной одинаково вероятен (по $1 / 2$ ).
Если владельцы обеих закусочных хотят быть уверенными более чем на 95\% в том, что у них найдется достаточное число мест, то сколько мест должно быть в каждой закусочной?
Каков будет ответ на поставленный вопрос, если у закусочных один владелец и его цель с той же надежностью не допустить очереди ни в одной из закусочных?

Ответ: В первом случае по 6 , во втором случае $6+7$

\item Спортивные общества А и В состязаются тремя командами. Вероятности выигрыша матчей команд общества А у соответствующих команд общества В можно принять равными 0.7 для 1-й (против 1-й В), 0.6 для 2-й (против 2-й В) и 0.3 для 3-й (против 3-й В).
Для победы необходимо выиграть не менее двух матчей из трех (ничьих не бывает). Чья победа вероятнее?

Ответ: A.
\item Проводится 8 независимых испытаний, в каждом из которых подсчитывается число гербов при одновременном подбрасывании трех монет. Найти вероятность того, что все возможные исходы одного испытания произойдут по два раза.

Ответ: $25515 / 524288 \approx 0.048666$.


\item Усложнение схемы Бернулли + гипергеометрическое распределение
\textit{С.В. СИМУШКИН, Л.Н. ПУШКИН ЗАДАЧИ ПО ТЕОРИИ ВЕРОЯТНОСТЕЙ}

Пятеро студентов проживали в одной комнате общежития. Поэтому, готовясь к экзамену «параллельно», каждый из них сумел выучить только 75 одних и тех же вопросов из 100.
На экзамене каждый из них независимо друг от друга получит три случайных вопроса.
Какова вероятность, что на экзамене ими будет получено две пятерки, две четверки и одна тройка, если пятерка ставится за три правильных ответа, четверка - за два правильных ответа и тройка - за один правильный ответ на три вопроса?

Решение. Для каждого студента на экзамене возможно четыре исхода О5, O4, O3, O2, очевидным образом связанные с полученной оценкой. Для того чтобы подсчитать вероятности этих исходов, воспользуемся гипергеометрическим распределением, то есть предположим, что три вопроса в билете формируются путем выбора трех шаров из урны, содержащей 75 красных шаров и 25 белых шаров:
$$
\begin{aligned}
& \mathbf{P}\left\{O_5\right\}=\mathbb{G} g(3 \mid 100,75,3)=\frac{\mathbf{C}_{75}^3 \cdot \mathbf{C}_{25}^0}{\mathbf{C}_{100}^3}=\frac{67525}{161700} \approx 0.418, \\
& \mathbf{P}\left\{O_4\right\}=\mathbb{G} g(2 \mid 100,75,3)=\frac{\mathbf{C}_{75}^2 \cdot \mathbf{C}_{25}^1}{\mathbf{C}_{100}^3}=\frac{69375}{161700} \approx 0.429, \\
& \mathbf{P}\left\{O_3\right\}=\mathbb{G} g(1 \mid 100,75,3)=\frac{\mathbf{C}_{75}^1 \cdot \mathbf{C}_{25}^2}{\mathbf{C}_{100}^3}=\frac{22500}{161700} \approx 0.139, \\
& \mathbf{P}\left\{O_2\right\}=\mathbb{G} g(0 \mid 100,75,3)=\frac{\mathbf{C}_{75}^0 \cdot \mathbf{C}_{25}^3}{\mathbf{C}_{100}^3}=\frac{2300}{161700} \approx 0.014,
\end{aligned}
$$

Результат сдачи экзамена пятью студентами можно представить в виде пятимерного вектора, каждая координата которого закреплена за конкретным студентом ( $\omega 1, \omega 2, \omega 3, \omega 4, \omega 5$ ).
Условие задачи будет удовлетворено, если произойдет, например, событие ( $O_5,O_5,O_4,O_4,O_3$ ).
Вероятность этого события, если считать независимым процесс сдачи экзамена студентами, равна $\mathrm{P}\{(O_5,O_5,O_4,O_4,O_3)\}=(0.418)^2 \cdot(0.429)^2 \cdot 0.139 \approx 0.00447$.
Совокупность благоприятных событий можно найти, перебрав все $5!=120$ вариантов перестановок исходов в рассмотренном нами событии.
Однако некоторые из этих перестановок будут неразличимы, поскольку два места в векторе занимают одинаковые исходы $O_5$ и два - исходы $O_4$. Легко понять, что всего имеется $5! / 2!\cdot 2!\cdot 1!= 120 / 4 = 30$ различных (но равновероятных) событий, отвечающих условию задачи.
Поэтому искомая вероятность равна $30 \cdot 0.00447 \approx 0.134$

\item
\begin{enumerate}
    \item Доказать, что сумма вероятностей показательного закона равна $1$.
    \item Доказать, что матожидание показательного закона равно $\lambda$ - не через предельный переход от биномиального закона, а через сумму ряда.
\end{enumerate}
$$
\begin{aligned}
& \sum p_i=\sum k \cdot \frac{e^{-\lambda} \lambda^k}{k!}=e^{-\lambda}\left(\frac{1}{0!}+\frac{\lambda}{1!}+\frac{\lambda^2}{2!}+\cdots\right)=e^{-\lambda} e^\lambda=1 \\
& \sum x_i p_i=\sum k \cdot \frac{e^{-\lambda} \lambda^k}{k!}=e^{-\lambda} \lambda\left(\frac{1}{0!}+\frac{\lambda}{1!}+\frac{\lambda^2}{2!}+\cdots\right)=e^{-\lambda} \lambda e^\lambda=\lambda
\end{aligned}
$$

\item Сотрудник техподдержки Иван обслуживает каждый день по $5$ заявок. Вероятность того, что клиент окажется недоволен результатом работы, и пожалуется на Ивана начальству, равна $0.005$ (отдельно по каждой заявке).
    \begin{enumerate}
        \item Контракт с сотрудником заключается на год, и если за год (247 рабочих дней) у него наберется более 3 дней, за которые на него жаловались, контракт не будет продлен. Найти вероятность того, что контракт с Иваном не будет продлен.
        \item За год Иван обслужил $1235 \; (247 \cdot 5)$ заявок. Найти вероятность того, что всего по результатам года на него поступило более трех жалоб.
        \item Есть ли разница в пунктах а) и б)?
    \end{enumerate}
    \textit{(Использовать приближение пуассоновским распределением, где это возможно)}

Ответ: а) 0.86 б) 0.864 в) есть.

а) Решение: Всего 247 дней (одинаковых независимых испытаний), вероятность успеха (на него пожалуются, причем неважно сколько раз за день) постоянная, нас интересует количество успехов - это схема Бернули.
Однако нам неизвестна вероятность успеха - вероятность того, что будет хотя бы одна жалоба.
Это вновь ищется по схеме Бернулли: $n=5$ одинаковых независимых испытаний, $p=0.005=$ const, $k \geq 1$ - число успехов.
Получаем
$$
P(k \geq 1)=1-P(k=0)=1-C_5^0 \cdot 0.005^0 \cdot 0.995^5=0.0248
$$

Возвращаемся к первой схеме Бернулли:
$$
n=247, p=0.0248, k>3: P(k>3)=1-(P(k=0)+P(k=1)+P(k=2)+P(k=3))
$$

но так как $n$ велико, $p$ мало, и $\lambda = n \cdot p=247 \cdot 0.0248 = 6.1256$, то можно воспользоваться формулой Пуассона $P(k)=\frac{e^{-\lambda} \lambda^k}{k!}$, получаем ответ 0.86


б) Указание:
$$
\begin{aligned}
& n=1235, p=0.005, k>3: P(k>3)=1-(P(k=0)+P(k=1)+P(k=2)+P(k=3)), \text { но т.к. } \\
& \lambda=n p=1235 \cdot 0.005=6.175, \text { то можно воспользоваться формулой Пуассона } P(k)=\frac{e^{-\lambda} \lambda^k}{k!},
\end{aligned}
$$
получаем ответ 0.864

    \item В некотором агентстве по набору текста работают 2 сотрудника. 
    Среднее количество ошибок на статью составляет $3$, когда 
    текст набирает первый сотрудник, и $4.2$, когда набирает 
    второй. Если ваша статья с равной вероятностью может быть 
    набрана любым из сотрудников, найдите приближенную вероятность 
    того, что в ней не будет ошибок.


    Let $A$ and $B$ the event that the paper is typed by typist $A$ or typist $B$ respectively. Let $E$ be the event that our article has a least one error, then
$$
P(E)=P(E \mid A) P(A)+P(E \mid B) P(B)
$$
since both typist are equally likely $P(A)=P(B)=\frac{1}{2}$ and
$$
P(E \mid A)=\sum_{i=1}^{\infty} P\{E=i \mid A\}=\sum_{i=1}^{\infty} e^{-\lambda_A} \frac{\lambda_A^i}{i!}=1-e^{-\lambda_A}=1-e^{-3}=0.9502
$$
and in the same way
$$
P(E \mid B)=1-e^{-4.2}=0.985
$$
so that $P(E)=0.5 \cdot (0.9502)+0.5 \cdot (0.985)=0.967$ so the probability of no errors is given by $1-P(E)=0.03239$
\end{enumerate}
\end{document}
