\documentclass{article}
\usepackage[utf8]{inputenc}
\usepackage[english,russian]{babel}
\usepackage[top=1cm,bottom=1cm,left=2cm,right=2cm]{geometry}
\usepackage{graphicx}
\usepackage{amsmath}
\usepackage{amsfonts}
\usepackage{xcolor}
\title{ВШБ БИ: ТВиМС 2025. \\ Лист seminar-only задач \#9. \\ Центральная предельная теорема. Интегральная теорема Муавра-Лапласа.}
\date{}
\author{}

\begin{document}
\maketitle

\begin{enumerate}
    % src: Ross, 8.7
    \item У человека есть $100$ лампочек, время жизни которых
    представляется независимыми экспоненциальными случайными величинами со средним $5$ часов. Если
    лампочки используются по одной, при этом перегоревшая лампочка
    немедленно заменяется новой, приближенно найдите
    вероятность того, что после $525$ часов всё ещё есть работающая
    лампочка.

    \textit{Решение:}

    Суммарное время жизни всех лампочек задаётся формулой
    $$
    T=\sum_{i=1}^{100} X_i,
    $$
    где $X_i$ --- экспоненциальная случайная величина со средним пять часов. Поскольку случайная величина $T$ является суммой независимых одинаково распределённых случайных величин, мы можем использовать центральную предельную теорему для получения оценок относительно $T$. Например, мы знаем, что
    $$
    \frac{\sum_{i=1}^n X_i-n \mu}{\sigma \sqrt{n}},
    $$
    приближённо имеет стандартное нормальное распределение. Таким образом, для вычисления (поскольку $\sigma^2=25$) имеем, что
    $$
    \begin{aligned}
    P\{T>525\} & =P\left\{\frac{T-100\cdot 5}{10\cdot 5}>\frac{525-500}{50}\right\} \\
    & =1-P\{Z<1 / 2\} \\
    & =1-\Phi(0.5)=1-0.6915=0.3085 .
    \end{aligned}
    $$

    % src: Ross, 8.8
    \item В предыдущей задаче предположим, что замена перегоревшей
    лампочки занимает случайное время,
    равномерно распределённое на интервале $(0, 0.5)$. Приближенно найдите вероятность того, что все
    лампочки перегорят к моменту времени $550$.

    \textit{Решение:}

    Выражение для суммарного времени работы лампочек без учёта времени замены задаётся формулой
$$
T=\sum_{i=1}^{100} X_i .
$$

Если к этому еще добавляется случайное время для замены каждой лампочки,
то наша случайная величина $T$ должна теперь это учитывать:
$$
T=\sum_{i=1}^{100} X_i+\sum_{i=1}^{99} U_i .
$$

Результат выше можно интерпретировать так, что мы можем заменить лампочки 99 раз, а последняя лампочка не заменяется.

Нам нужно вычислить $P\{T \leq 550\}$. Для этого перепишем
$$
T=\sum_{i=1}^{99}\left(X_i+U_i\right)+X_{100},
$$
и можно определить новые случайные величины $V_i$, которые будут равны сумме времени работы лампочки и времени замены:
$$
V_i=\left\{\begin{array}{cc}
X_i+U_i & i=1, \cdots, 99 \\
X_{100} & i=100
\end{array}\right.
$$

Тогда $T=\sum_{i=1}^{100} V_i$ и все $V_i$ независимы.
Ниже будем использовать $\mu_i$ и $\sigma_i$ для обозначения математического ожидания и стандартного отклонения случайных величин $V_i$ соответственно.
Вычисляя математическое ожидание $T$, получим:
$$
\begin{aligned}
E[T] & =\sum_{i=1}^{100} E\left[V_i\right]=\sum_{i=1}^{99}\left(E\left[X_i\right]+E\left[U_i\right]\right)+E\left[X_{100}\right] \\
& =100 \cdot 5+99\left(\frac{1}{4}\right)=524.75 .
\end{aligned}
$$

Аналогичным образом дисперсия этой суммы также задаётся формулой
$$
\begin{aligned}
\operatorname{Var}(T) & =\sum_{i=1}^{99}\left(\operatorname{Var}\left(X_i\right)+\operatorname{Var}\left(U_i\right)\right)+\operatorname{Var}\left(X_{100}\right) \\
& =100 \cdot 5+99 \cdot \frac{1}{4}\left(\frac{1}{12}\right)=502.0625
\end{aligned}
$$

По центральной предельной теореме получаем, что
$$
P\left\{\sum_{i=1}^{100} V_i \leq 550\right\}=P\left\{\frac{\sum_{i=1}^{100}\left(V_i-\mu_i\right)}{\sqrt{\sum_{i=1}^n \sigma_i^2}} \leq \frac{550-\sum_{i=1}^{100} \mu_i}{\sqrt{\sum_{i=1}^n \sigma_i^2}}\right\} .
$$

Вычисляя выражение в правой части неравенства:
$$
\frac{550-\sum_{i=1}^{100} \mu_i}{\sqrt{\sum_{i=1}^n \sigma_i^2}},
$$

получаем, что оно равно $\frac{550-524.75}{\sqrt{502.0625}}=1.1269$. Следовательно,
$$
P\left\{\sum_{i=1}^{100} V_i \leq 550\right\} \approx F_Z(1.1269)=0.8701.
$$
    
    
    \item Допустим, что срок службы пылесоса имеет экспоненциальное распределение.
    В среднем один пылесос бесперебойно работает $7$ лет.
    Завод предоставляет гарантию $5$ лет на свои изделия.
    Предположим также, что примерно $80 \%$ потребителей аккуратно хранят все бумаги, необходимые, чтобы воспользоваться гарантией.
    \begin{enumerate}
    \item Какой процент потребителей в среднем обращается за гарантийным ремонтом?
    \item Какова вероятность того, что из $1000$ потребителей за гарантийным ремонтом обратится более $35\%$ покупателей?
    \end{enumerate}
    Подсказка: $\exp (5 / 7) \approx  2.0427$
    
    \textit{Решение:}
    $$
    \begin{aligned}
    & P_{\text {break }}=1-\exp (-5 / 7)=0.51=\int_0^5 \frac{1}{7} e^{-\frac{t}{7}} dt \\
    & p=0.8 \cdot 0.51 \approx 0.4 \\
    & \mathbb{E}(S)=1000 p=400, \operatorname{Var}(S)=1000 p(1-p)=240 \\
    & \mathbb{P}(S>350)=\mathbb{P}(Z>-3.23) \approx 1
    \end{aligned}
    $$
    
    \newpage
    \item Театр имеет два различных входа. Около каждого из входов имеется свой гардероб.
    Эти гардеробы ничем не отличаются. На спектакль приходит $1000$ зрителей.
    Предположим, что зрители приходят по одиночке и выбирают входы равновероятно.
    Сколько мест должно быть в каждом из гардеробов для того,
    чтобы в среднем в $99$ случаях из $100$ все зрители могли раздеться в гардеробе того входа,
    через который они вошли?
    
    
    \textit{Решение:}
    Пусть $X_i$ - случайная величина, которая равна $1$, если посетитель $i$ выбрал первый вход и $0$, если второй.
    $X_i \sim \operatorname{Bernoulli}(p)$. Тогда $\bar{X}=\sum_{i=1}^{100} X_i / 1000$ - доля посетителей, вошедших через первый вход.
    По условию:
    $$
    E\left(X_i\right)=\frac{1}{2} \cdot \sigma=\sqrt{\frac{\frac{1}{2} \cdot \frac{1}{2}}{1000}}=\frac{1}{20 \sqrt{10}}
    $$
    
    Найдем такое $k$, что $\mathbb{P}(\bar{X}<k)>0.99$ :
    $$
    \begin{gathered}
    \mathbb{P}\left(\frac{\bar{X}-1 / 2}{\frac{1}{20 \sqrt{10}}}<\frac{k-1 / 2}{\frac{1}{20 \sqrt{10}}}\right)>0.99 \\
    \mathbb{P}(Z<10 \sqrt{10}(2 k-1))>0.99 \\
    10 \sqrt{10}(2 k-1)>2.33 \\
    k>0.536841
    \end{gathered}
    $$
    
    Аналогичную долю получаем и для второго гардероба.
    Наименьшее необходимое число мест в гардеробе будет равно $\lceil 1000 k\rceil=\lceil 536.841\rceil=537$

    
    \item Компания кабельного телевидения \textit{HBT} исследует возможность присоединения к своей сети пригородов города \textit{N}.
   Опросы показали, что в среднем каждые $3$ из $10$ семей жителей пригородов хотели бы стать абонентами сети.
   Стоимость работ, необходимых для организации сети в любом пригороде оценивается величиной $2080000$ у.е.
   При подключении каждого пригорода \textit{HBT} надеется получить $1000000$ у.е. в год от рекламодателей.
   Планируемая чистая прибыль от оплаты за кабельное телевидение одной семьей в год равна $120$ у.е.
   Каким должно быть минимальное количество семей в пригороде для того, чтобы с вероятностью $0.99$ расходы на организацию сети в этом пригороде окупились за год?
   
   \textit{Решение:}
   Обозначим $N$ - количество подключенных абонентов, тогда $N \sim \operatorname{Bin}(n, 0.3)$.
   При больших $n$ биномиальное распределение можно заменить на нормальное, $N \sim N(0.3 n, 0.21 n)$.
   $$
   \mathbb{P}(120 N>1080000)=\mathbb{P}(N>9000)=\mathbb{P}\left(Z>\frac{9000-0.3 n}{\sqrt{0.21 n}}\right)=0.99
   $$
   
   Из таблицы находим, что
   $$
   \frac{9000-0.3 n}{\sqrt{0.21 n}}=-2.3263479
   $$
   
   Решаем квадратное уравнение, находим корни, один - отрицательный, другой, $n \approx 30622$.
\end{enumerate}

\end{document}
