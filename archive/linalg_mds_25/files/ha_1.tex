\documentclass{article}
\usepackage[utf8]{inputenc}
\usepackage[english, russian]{babel}
\usepackage[top=1cm,bottom=1cm,left=2cm,right=2cm]{geometry}
\usepackage{graphicx}
\usepackage{amsmath}
\usepackage{amsfonts}
\title{ФКН ВШЭ, МНаД. \\ Линейная алгебра. \\ Домашнее задание 1, 2025.}
\date{}
\author{}
\graphicspath{{data/}}  

\begin{document}
\maketitle
% \begin{center}
%     Deadline: before seminar starts
% \end{center}

\begin{enumerate}

    % src: Strang 1.3.1
    \item Даны три вектора: $a_1 = (1,1,1)^{\top}$, $a_2 = (0, 1, 1)^{\top}$, $a_3 = (0,0,1)^{\top}$. Вычислите линейную комбинацию
    $$
        b = 3a_1 + 4a_2 + 5a_3.
    $$
    Затем запишите $b$ в виде произведения матрицы на вектор $Ax$, где $A = \left( a_1 \; | \; a_2 \; | \; a_3 \right)$, а $x = (3,4,5)^{\top}$. Вычислите это произведение матрицы на вектор по методу строка-на-столбец, \textit{т.е.} вычислите три скалярных произведения: строка матрицы $A$ на вектор $x$. Проверьте и сравните результаты.

    % src: LADW
    \item Вычислите произведение матрицы на вектор (любым удобным способом):
    \begin{enumerate}
        \item 
        $\begin{pmatrix}
            1 & 2 & 3\\
            4 & 5 & 6
        \end{pmatrix} \begin{pmatrix}
            1 \\
            3 \\
            2
        \end{pmatrix}$,

        \item 
        $\begin{pmatrix}
            1 & 2 \\
            0 & 1 \\
            2 & 0
        \end{pmatrix} \begin{pmatrix}
            1 \\
            3
        \end{pmatrix}$,

        \item 
        $\begin{pmatrix}
            1 & 2 & 0 & 0 \\
            0 & 1 & 2 & 0 \\
            0 & 0 & 1 & 2 \\
            2 & 0 & 0 & 1
        \end{pmatrix} \begin{pmatrix}
            1 \\
            3 \\
            2 \\
            4
        \end{pmatrix}$.
    \end{enumerate}

    % LADW Chapter 1, 5.1
    \item Даны матрицы: $A = \begin{pmatrix}
            1 & 2 \\
            3 & 1
            \end{pmatrix}$,
        $B = \begin{pmatrix}
            1 & 0 & 2\\
            3 & 1 & -2
        \end{pmatrix}$,
        $C = \begin{pmatrix}
            1 & -2 & 3 \\
            -2 & 1 & -1
        \end{pmatrix}$,
        $D = \begin{pmatrix}
            -2 \\
             2 \\
             1
        \end{pmatrix}$.

        \begin{enumerate}
            \item Отметьте все произведения, которые определены (с точки зрения размеров матриц), и укажите размерности
            результата:
            $$ AB, BA, ABC, ABD, BC, BC^{\top}, B^{\top}C, DC, D^{\top}C^{\top}. $$
            \item Вычислите:

            $$ AB, A(3B +C), B^{\top}A. $$
        \end{enumerate}

        % src: MDI, LA, 23-24, Topic1
        \item Даны матрицы:
        $$
        A=\left(\begin{array}{ll}
        2 & 1 \\
        1 & 1 \\
        0 & 3
        \end{array}\right), b=\left(\begin{array}{c}
        1 \\
        1 \\
        -1
        \end{array}\right), C=\left(\begin{array}{ccc}
        1 & 2 & 1 \\
        3 & 0 & -1 \\
        4 & 1 & 1
        \end{array}\right), D=\left(\begin{array}{ll}
        0 & 1 \\
        2 & 5 \\
        6 & 3
        \end{array}\right)
        $$
        Какие из следующих матричных выражений определены? Вычислите те, которые определены.
        
        a)$A b$, b) $C A$, c) $A+Cb$, d) $A+D$, e) $b^T D$, f) $D A^T+C$, g) $b^T b$, h) $b b^T$, i) $C b$.
        
        % src: FCS, 23-24, sem1
        \item Вычислите оптимальным способом:
        $$
        A^2+A B+B A+B^2,
        $$
        
        если даны матрицы:
        $$
        A=\left(\begin{array}{cc}
        47 & 59 \\
        -23 & 21
        \end{array}\right), B=\left(\begin{array}{cc}
        -42 & -59 \\
        22 & -20
        \end{array}\right).
        $$

        % src: FCS, 23-24, sem1
        \item Вычислите следующее произведение двумя возможными способами: $(A B) C$ и $A(B C)$.
        $$
        A B C=\left(\begin{array}{ccc}
        1 & -1 & 3 \\
        -1 & 1 & -3 \\
        2 & -2 & 6
        \end{array}\right)\left(\begin{array}{ccc}
        1 & 5 & 2 \\
        0 & 3 & -1 \\
        2 & 1 & -1
        \end{array}\right)\left(\begin{array}{c}
        1 \\
        3 \\
        -2
        \end{array}\right)
        $$

        Проверьте, что результаты одинаковы. Какой способ быстрее с точки зрения количества операций? (Можете ответить либо интуитивно, либо используя формулы сложности для произведений матрица-матрица и матрица-вектор).

        % src: % LADW Chapter 1, 5.7
        \item Приведите пример ненулевой матрицы $A \in \mathbb{R}^{2 \times 2}$, такой что $A A = \begin{pmatrix}
            0 & 0 \\
            0 & 0
        \end{pmatrix}$.

        % src: Strang 1.3.4
        \item Найдите коэффициенты $x_2$ и $x_3$ такие, что линейная комбинация $1 \cdot w_1 + x_2 \cdot w_2 + x_3 \cdot w_3$ дает нулевой вектор. Даны векторы: $w_1 = (1, 2, 3)^{\top}$, $w_2 = (4, 5, 6)^{\top}$, $w_3 = (7, 8, 9)^{\top}$. Проверьте ответ также в классическом виде произведения матрицы на вектор (три скалярных произведения строка-на-столбец).

        % src: MDI, LA, 23-24, Topic1
        \item Если $A$ — это матрица размера $n \times n$, докажите, что матрица $(A + A^{\top})$ симметрична, а матрица $(A - A^{\top})$ кососимметрична.
        
        \textit{Указание: симметричная матрица обладает свойством $X = X^{\top}$, а кососимметричная матрица обладает свойством $X = -X^{\top}$.}
        
        \item Докажите, что если в группе векторов $v_1, \ldots, v_m$ из векторного пространства $\mathbb{V}$ есть хотя бы одна линейно зависимая подгруппа длины $k$, $k < m$, то вся группа также линейно зависима. 

        \textit{Подсказка: можете начать с ситуации, когда у вас есть линейно зависимая группа длины $k$, и доказать, что если добавить к этой группе любой другой вектор из $\mathbb{V}$, новая группа остается линейно зависимой.}

        \item Докажите, что если $v_1, \ldots, v_m$, $m \geq 2$, является линейно независимой группой векторов из векторного пространства $\mathbb{V}$, то после удаления из нее любого вектора она остается линейно независимой.
\end{enumerate}
\end{document}
