\documentclass{article}
\usepackage[utf8]{inputenc}
\usepackage[english, russian]{babel}
\usepackage[top=1cm,bottom=1cm,left=2cm,right=2cm]{geometry}
\usepackage{graphicx}
\usepackage{amsmath}
\usepackage{amsfonts}
\title{ФКН ВШЭ, МНаД. \\ Линейная алгебра. \\ Лист задач 2. Базисы и координаты.}
\date{}
\author{}
\graphicspath{{data/}}  
% Custom command for column vector
\newcommand{\vtwo}[2]{\begin{pmatrix} #1 \\ #2 \end{pmatrix}}

\begin{document}
\maketitle
% \begin{center}
%     Deadline: before seminar starts
% \end{center}

\begin{enumerate}


    \item Найдите координатный вектор для:
    \begin{enumerate}
        \item Вектора $x = \vtwo{1}{1}$ в базисе $\mathcal{B} = \{b_1, b_2\}$ пространства $\mathbb{R}^2$, если $b_1 = \vtwo{2}{-4}$, $b_2 = \vtwo{3}{8}$.
        \item Вектора $p(x) = 4 + 5 x$ в базисе $U = \{u_1, u_2\}$ пространства $\mathbb{R}[x,1]$, если $u_1 = 1 + 2x$, $u_2 = -2 - 3x$.
    \end{enumerate}

    \textit{Hint: нужно составить и решить две системы линейных уравнений:}
    $$
    P_{B \to S} [x]_B = [x]_S \quad \text{и} \quad P_{U \to S} [p(x)]_U = [p(x)]_S
    $$

\item В пространстве $\mathbb{R}^2$ задан базис $B = \{b_1, b_2\}$, где $b_1 = \vtwo{1}{1}$, $b_2 = \vtwo{2}{1}$
\begin{enumerate}
    \item Найдите матрицу перехода $P_{B \to S}$ от базиса $B$ к стандартному базису $S$ пространства $\mathbb{R}^2$.
    \item Найдите матрицу перехода $P_{S \to B}$ от стандартного базиса $S$ к базису $B$.
    \item Убедитесь, что эти матрицы являются обратными друг для друга.
\end{enumerate}

\item В пространстве $\mathbb{R}^2$ задан базис $\mathcal{B} = \{b_1, b_2\}$, где $b_1 = \vtwo{1}{3}$, $b_2 = \vtwo{1}{0}$,
а также базис $\mathcal{C} = \{c_1, c_2\}$, где $c_1 = \vtwo{1}{1}$, $c_2 = \vtwo{-2}{1}$.

Найдите:

\begin{enumerate}
    \item Найдите матрицы перехода $P_{B \to S}$ и $P_{S \to B}$ от базиса $\mathcal{B}$ к стандартному базису и обратно.
    \item Найдите матрицы перехода $P_{C \to S}$ и $P_{S \to C}$ от базиса $\mathcal{C}$ к стандартному базису и обратно.
\end{enumerate}

Далее, воспользуемся знанием, что $P_{B \to S} [x]_B = [x]_S = P_{C \to S} [x]_C$ (на этом моменте спросите себя, понимаете ли вы, почему это так).
Примените трюк с обратной матрицей и найдите матрицы перехода от базиса $\mathcal{B}$ к базису $\mathcal{C}$ и обратно, то есть $P_{B \to C}$ и $P_{C \to B}$.
Приведите пример, как это работает, на конкретно взятом векторе.

\item Пусть $\mathbb{V}$ — множество всех верхнетреугольных (у которых элементы под главной диагональю всегда равны нулю) матриц размера $2 \times 2$. Множество $\mathbb{V}$ является векторным пространством, и у него есть, например, стандартный базис:

    $$
    \mathcal{A}=\left(\left[\begin{array}{ll}
    1 & 0 \\
    0 & 0
    \end{array}\right],\left[\begin{array}{ll}
    0 & 1 \\
    0 & 0
    \end{array}\right],\left[\begin{array}{ll}
    0 & 0 \\
    0 & 1
    \end{array}\right]\right)
    $$

    Найдите координаты элемента $D=\left[\begin{array}{rr}2 & -1 \\ 0 & 1\end{array}\right]$ относительно стандартного базиса $\mathcal{A}$. Покажите, что это работает в виде линейной комбинации базисных векторов и координат, \textit{т.е.} как мы обсуждали, что если $\{ v_1, \ldots, v_n \}$ — базис, то $\forall x$ из векторного пространства $x = \alpha_1 v_1 + \ldots + \alpha_n v_n$.

    Но также могут быть и другие базисы! Пусть:

    $$
    \mathcal{B}=\left(\left[\begin{array}{ll}
    1 & 0 \\
    0 & 1
    \end{array}\right],\left[\begin{array}{ll}
    0 & 1 \\
    0 & 1
    \end{array}\right],\left[\begin{array}{ll}
    1 & 0 \\
    0 & 2
    \end{array}\right]\right)
    $$
является другим базисом для $\mathbb{V}$. Постройте матрицу $P_{B \to A}$ перехода от базиса $\mathcal{B}$ к базису $\mathcal{A}$. Найдите координатный столбец $[D]_{\mathcal{B}}$, т.е. координаты элемента $D$ относительно базиса $\mathcal{B}$.

\end{enumerate}

\end{document}
