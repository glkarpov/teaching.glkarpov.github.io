\documentclass{article}
\usepackage[utf8]{inputenc}
\usepackage[english, russian]{babel}
\usepackage[top=1cm,bottom=1cm,left=2cm,right=2cm]{geometry}
\usepackage{graphicx}
\usepackage{amsmath}
\usepackage{amsfonts}
\title{ФКН ВШЭ, МНаД. \\ Линейная алгебра. \\ Информация про экзамен.}
\date{}
\author{}
\graphicspath{{data/}}  

\begin{document}
\maketitle

\section{Информация про экзамен}

Нас ожидает письменная работа с асинхронным прокторингом (запись мы будем отсматривать при проверке), с двумя камерами (на компьютере и на телефоне) и записью экрана.

В SmartLMS будет специальный раздел, со свободной навигацией между заданиями. На последней странице будет опция загрузки файлов.
Когда время на решение закончится, можно отключить камеру на телефоне (но не отключать остальные элементы прокторинга) и использовать телефон для сканирования / фотографирования ваших решений.


Всего будет 10-14 задач разного уровня, в целом похожих на те, что мы решали на занятиях и которые были в домашних заданиях, контрольных и квизах.
Для оценки, близкой к максимальной, не обязательно решать всё, количество задач специально будет сделано с некоторым запасом.
Тем не менее, если вы претендуете на высокую оценку по курсу, то нужно постараться успеть решить как можно больше.

\begin{itemize}
  \item Основное время на выполнение задач: 140 минут. Следите за временем самостоятельно, не допускайте превышения.
  \item Дополнительное время на подготовку и загрузку документов: 30 минут. Выключите на телефоне режимы из разряда Super Mega Ultra HD Quality :)
  Фотографии и последующий pdf документ не должны весить, условно, более 50 мегабайт;
  \item Оформление: на бумаге, от руки;
  \item Можно пользоваться:
    \begin{itemize}
    \item лекции и записи лектора, открытые в отдельной вкладке или в программе просмотра pdf документов. Скачать актуальные версии можно по ссылкам в разделе "Материалы" на сайте курса;
    \item калькулятор из операционной системы, открытый в отдельном окне, или калькулятор, специально доступный в оболочке прокторинга;
    \end{itemize}
  \item Нельзя использовать:
    \begin{itemize}
    \item Наушники;
    \item Поисковые системы, LLM-агенты;
    \item Специальные калькуляторы матриц;
    \item Смартфоны;
    \item Другие материалы, отличные от разрешенных;
    \item Языки программирования.
    \end{itemize}

\item Учитывая длительность экзамена, выход в уборную допускается, однако просим ограничиваться 5–7 минутами в час.
В последние 20 минут экзамена выход запрещён. Перед выходом, пожалуйста, заранее проговорите вслух, что вы планируете воспользоваться правом на выход.
Если вы будете отсутствовать продолжительнее, это вызовет вопросы и может привести к ощутимому снижению оценки.
\item Можно закончить раньше. Тогда нужно проговорить вслух, что вы закончили, отключить камеру на телефоне (но не отключать остальные элементы прокторинга) и приступить к подготовке решений к отправке.
\item Если не получается загрузить в SmartLMS - отправляйте ассистентам (\texttt{@pro\_sha\_q}, \texttt{@yu\_igb}).
\end{itemize}

\textbf{Про оформление и проверку.} Я придерживаюсь принципа, что в работе по математике сам по себе ответ не так важен, как ход решения и последовательность изложения ваших мыслей.
С этой точки зрения, если в работе студента написано условие и сразу свалившийся с неба ответ, то это не очень хорошо и скорее всего такое решение не будет зачтено.

Относитесь к оформлению так, как будто вы рассказываете историю или делаете презентацию. Вашим слушателям (проверяющим) проще оценить вашу работу, если вы последовательно и логично рассказываете о том, как вы решали задачу.
Тем не менее, не всегда есть время на подробное изложение, поэтому найдите некий свой баланс. Слово "ответ" можно не писать, но обязательно нужно как-то визуально выделить финальные результаты.

За арифметические ошибки не ставится автоматически ноль. Я проинструктирую ассистентов, и размер штрафа за арифметическую ошибку будет зависеть от места, где она была допущена.
Например, если в самом-самом конце решения, то снижение будет минимальным (если вообще будет), а если в самом начале, тем самым создав лавину неправильных дальнейших результатов, то штраф будет весомым, но тоже не 100\%.

\newpage

\section{Список тем задач на экзамене}
\begin{itemize}
    \item Базовые операции с векторами и матрицами, вычисление скалярного произведения и норм векторов; 
    \item Разложение вектора по разным базисам (понимать идею базиса и координат);
    \item Построение матрицы пересчета координат вектора при переходе между разными базисами;
    \item Построение матрицы линейного отображения;
    \item Изменение матрицы линейного отображения при переходе в другие базисы в domain/target space;
    \item Решение СЛУ методом Гаусса, работа с аугментированной матрицей СЛУ и приведение её к эшелонному (ступенчатому) виду;
    \item Анализ наборов векторов с использованием ступенчатой формы: линейная независимость, полнота линейной оболочки, базис;
    \item Нахождение обратной матрицы $2 \times 2$ по формуле, или $3 \times 3$ с помощью строчных преобразований (может пригодиться максимум в паре задач), или любым другим способом;
    \item Ортогонализация и ортонормирование базиса (процесс Грама-Шмидта).
\end{itemize}

\newpage

\section{Примерный состав варианта}

Вводится четыре уровня сложности задач: I, II, III, IV. Вариант будет единым для обоих потоков.
Возможно, мы рассмотрим некую чуть более щадящую нормировку для потока 2, если обнаружим сильные статистические различия по среднему баллу.

Нужно быть готовыми к тому, что на получение самых высоких оценок будет работать некий блокирующий механизм: например, разблокировка оценки 8 за экзамен будет возможна при правильном решении хотя бы одной задачи уровня III или IV, оценки 9 и выше - при правильном решении хотя бы двух задач уровня III или IV.
Иными словами, нельзя ожидать, что вы сможете получить за экзамен оценку 10, просто решив все задачи уровня I и II и тем самым набрав большую сумму.

В некоторых пунктах приведено несколько примеров.
\begin{enumerate}
    \item (I.) Даны три вектора: $\mathbf{a}(1,-1,1), \mathbf{b}(4,-2,1), \mathbf{c}(0,3,-3)$. Вычислить:

    % src: bekl 2.10
    \begin{enumerate}
        \item $\mathbf{b}(\mathbf{a}, \mathbf{c}) - \mathbf{c}(\mathbf{a}, \mathbf{b})$,
        \item $\lVert\mathbf{a}\rVert_2^2+\lVert\mathbf{c}\rVert_2^2-(\mathbf{a}, \mathbf{b}) \cdot(\mathbf{b}, \mathbf{c})$,
        \item $(\mathbf{a}, \mathbf{c}) \cdot(\mathbf{a}, \mathbf{b})-\lVert \mathbf{a}\rVert_2^2(\mathbf{b}, \mathbf{c})$.
    \end{enumerate}

    \item (II.) Проведите ортогонализацию следующей группы векторов координатного пространства $\mathbb{R}^3$ со стандартным скалярным произведением: 

    $$
        x_1 = (1,1,1)^{\top}, \; x_2 = (1,1,2)^{\top}, \; x_3 = (2,1,1)^{\top}.
    $$

    Покажите ортогональность полученной системы. Как сделать эту систему также ортонормальной? Проведите нужные операции.

    \item (II.) Проанализируйте следующие наборы векторов:
    \begin{enumerate}
        \item $a_1=(2,-3,1), a_2=(3,-1,5), a_3=(1,-4,3)$
        \item $1-3 t+5 t^2,-3+5 t-7 t^2,-4+5 t-6 t^2, 1-t^2$
        \item $5 t+t^2, 1-8 t-2 t^2,-3+4 t+2 t^2, 2-3 t$
    \end{enumerate}

    Сделайте выводы о линейной зависимости/независимости, линейной оболочке предложенных наборов векторов. Какие из них образуют базис в своем пространстве?

    \item (II.)Решите систему линейных уравнений, используя метод Гаусса и приведение к ступенчатой форме:
    \begin{enumerate}
        \item
        $\left\{\begin{array}{l}5 x_1+3 x_2+5 x_3+12 x_4  =8 \\
            2 x_1+2 x_2+3 x_3+5 x_4  =4 \\
            x_1+7 x_2+9 x_3+4 x_4  =8
        \end{array}\right.$
        \item
        $\left\{\begin{array}{l}2 x_1-x_2-x_3=4 \\ 3 x_1+4 x_2-2 x_3=11\end{array}\right.$
    \end{enumerate}

    \item (II.) Нахождение координат вектора в разных базисах.
    \begin{enumerate}
        \item Список задач 2: номер 1.
        \item КР 1: номер 2.
        \item Найдите координаты вектора $f = 2 x^2-x+6$ относительно базиса $1, x, x^2, x^3$ и относительно базиса $1, x,(x-1)^2$. 
    \end{enumerate}

    \item (II.) Построение матрицы пересчета координат между базисом и стандартным базисом.    
    \begin{enumerate}
        \item Список задач 2: номер 2, 4.
        \item КР 1: номер 3.
    \end{enumerate}

    \item (III.) Построение матрицы пересчета координат между разными базисами.
    \begin{enumerate}
        \item Список задач 2: номер 3.
        \item КР 1: номер 3.
    \end{enumerate}

    \item (II.) Построение матрицы линейного отображения в паре зафиксированных базисов.    
    \begin{enumerate}
        \item Список задач 3: номер 1, 2а.
        \item КР 2: номер 1.
        \item Список задач 3: номер 4, 5а (для второго потока такие скорее всего будут уровня III).
    \end{enumerate}
    
    \item (III.) Изменение матрицы линейного отображения при переходе в другие базисы.
    \begin{enumerate}
        \item Список задач 3: номер 2, 5b.
        \item КР 2: номер 2.
    \end{enumerate}

    \item (II-III.) Разные задачи на линейные отображения.
    \begin{enumerate}
        \item Список задач 3: номер 3.
        \item КР 2: номер 3.
    \end{enumerate}
\end{enumerate}
\end{document}
